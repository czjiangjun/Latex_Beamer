\documentclass[cjk,slidestop,compress,mathserif,blue]{beamer}
%dvipdfm选项是关键,否则编译统统通不过
%beamer的颜色选项定义的是导航条和标题的颜色(即关键词structure的颜色)

%----------------------------------------------  PPT Preamble  ----------------------------------------------------------------------------------
%%%%%%%%%%%%%%%%仅限于XeTeX可使用的宏包%%%%%%%%%%%%%%%%%%%%%%%%%%%%
\usepackage{fontspec,xunicode,xltxtra,beamerthemesplit}
%\usepackage{beamerthemesplit}
%\usepackage{handoutWithNotes}		%(讲义)在打印PPT的时候会留出给每一页做注释的部分
\usepackage[dvipdfmx]{movie15_dvipdfmx} %插入视频与(讲义)打印有冲突
\usepackage{xeCJK}
%\setCJKmainfont[BoldFont=黑体, ItalicFont=楷体, BoldItalicFont=仿宋]{黑体}
%\setsansfont[Mapping=tex-text]{Adobe 黑体 Std}
%如果装了Adobe Acrobat,可在font.conf中配置Adobe字体的路径以使用其中文字体
%也可直接使用系统中的中文字体如SimSun,SimHei,微软雅黑 等
%原来beamer用的字体是sans family;注意Mapping的大小写,不能写错

\usepackage{listings} 
\lstset{language=Matlab}%代码语言使用的是matlab 
\lstset{breaklines}%自动将长的代码行换行排版 
\lstset{extendedchars=false}%解决代码跨页时,章节标\dots

%%%%%%%%   确定标题和导航条结构的框架     %%%%%%%%%%%%
\usepackage{beamerthemeshadow}                       %
%\usepackage{beamerthemeclassic}%导航条色与背景色一致%
%\usepackage{authblk}				     %作者地址和E-mail
%%%%%%%%%%%%%%%%%%%%%%%%%%%%%%%%%%%%%%%%%%%%%%%%%%%%%%
\setbeamerfont{roman title}{size={}}
%\usepackage{CJK} % CJK 中文支持                                  %
%\usepackage[version=3]{mhchem}		%化学公式
\usepackage{chemformula}
\usepackage{chemfig}		%化学公式

\usepackage{amsmath,amsthm,amsfonts,amssymb,bm}
\usepackage{bbding}
\usepackage{mathrsfs}
\usepackage{xcolor}                                        %使用默认允许使用颜色
\usepackage{hyperref} 
\usepackage{graphicx}
\usepackage{float}               %将图片定死在某一个位置用(主要支持[htbp!]中的h)
\usepackage{subfigure}           %图片跨页
\usepackage[controls]{animate}	 %插入动画
\usepackage{tikz}		 %绘图工具
\usetikzlibrary{%
    arrows,shapes,chains,shapes,arrows.meta,matrix,
    graphs, decorations, 
    decorations.markings, 
    decorations.pathmorphing, 
%    graphdrawing,                                       % requires lualatex
    shapes.geometric, snakes
}
%\usegdlibrary{trees,force, layered}                     % requires lualatex
\usepackage{adjustbox}                                   %绘制跨页流程图形
\newsavebox{\mysavebox}                                  %绘制跨页流程图形
\newlength{\myrest}                                      %绘制跨页流程图形

\usepackage{pgfplots}
%\pgfplotsset{width=10cm,compat=1.9}                     %每个pgfplot图形的大小更改为10cm
%\usepgfplotslibrary{external}                           %以将图形导出为单独的PDF文件,然后将其导入文档中
%\tikzexternalize
\usepackage{caption}
\captionsetup{font=footnotesize}

\usepackage{verbatim}			%Verbatim 宏包重新实现了 Verbatim 环境,并且提供一个命令可以导入一个 ASCII 文件到文档中
\usepackage{multirow}
\usepackage{makecell}		%允许单元格内换行

%\pgfpagesuselayout{1 on 1 with notes landscape}[a4paper,border shrink=5mm]

\usepackage{booktabs}           %修改表格线段的粗细,可以自定义修改线段粗细
%\toprule[2pt]                   %表格顶端线粗细设置
%\midrule[1pt]                   %表格中间线粗细设置
%\bottomrule[1.8pt]              %表格底端线粗细设置

%%%%%%%%%%%%%%%%%%%%%%BIBTEX 引用参考文献%%%%%%%%%%%%%%%%%%%%%%%%%%%%%%%%%%%%%%%%%%%%%%%%
%\usepackage{filecontents}
%\begin{filecontents*}{main.bib}
%@techreport{2012FracfocusChemical,
%  author = {FracFocus,},
%  howpublished = {\url{http://fracfocus.org/water-protection/drilling-usage}},
%  institution = {The Ground Water Protection Council and Interstate Oil and Gas
%  Compact Commission},
%  month = {feb},
%  title = {{Chemical Use In Hydraulic Fracturing}},
%  year = {2012}
%}
%\end{filecontents*}
%\usepackage[backend=bibtex,sorting=none]{biblatex}
%%\usepackage[backend=biber,style=authoryear]{biblatex}
%\addbibresource{main.bib} %BibTeX数据文件及位置

%\usepackage[numbers,sort&compress]{natbib} %紧密排列             %
\usepackage[sectionbib]{chapterbib}        %每章节单独参考文献   %
\usepackage{hypernat}                                                                         %
\setbeamertemplate{bibliography item}[text] %参考文献前标注[]
%\usepackage[dvipdfm,bookmarksopen=true,pdfstartview=FitH,CJKbookmarks]{hyperref}		%
\hypersetup{bookmarksnumbered,colorlinks,linkcolor=brown,citecolor=blue,urlcolor=red}         %
%参考文献含有超链接引用时需要下列宏包,注意与natbib有冲突        %
%\usepackage[dvipdfm]{hyperref}                                  %
%\usepackage{hypernat}                                           %
\newcommand{\upcite}[1]{\hspace{0ex}\textsuperscript{\cite{#1}}} %

%\usepackage{marvosym} %插入各种符号

%\useoutertheme{smoothbars}
\useinnertheme[shadow=true]{rounded}

% Beamer Settings
\usetheme{Berkeley}                                          %主题式样
%\usetheme{Luebeck}
%\usetheme{Warsaw}

\usecolortheme{lily}                                        %颜色主题式样

\usefonttheme{professionalfonts}                           %字体主题样式宏包

%\beamertemplatetransparentcoveredhigh                      %使所有被隐藏的文本高度透明
\beamertemplatetransparentcovereddynamicmedium             %使所有被隐藏的文本完全透明,动态,动态的范围很小
\mode<presentation>
%\beamersetaveragebackground{gray}                          %设置背景颜色(单一色) 
\beamertemplateshadingbackground{green!10}{red!5}         %设置背景颜色(渐变色)

\graphicspath{{$PATHPWD/Figures/}}                           %$PATH用于脚本指定图片路径
%\graphicspath{{/home/jun-jiang/Documents/Latex_Beamer/Figures/}}   %直接指定图片绝对路径
%\graphicspath{{/home/jiangjun/Documents/Latex_Beamer/Figures/}}   %直接指定图片绝对路径

%i放置单位logo
%\logo{\includegraphics[width=1.6cm,height=0.35cm]{Figures/BCC_logo-1.png}}	%简单设置logo

%\pgfdeclareimage[width=3.5cm]{logoname}{Figures/BCC_logo-1.png}		%logo置于左侧微调
%\logo{\pgfuseimage{logoname}{\vspace{0.2cm}\hspace*{-2.0cm}}}

%在指定位置精确放置logo
\usepackage{beamerfoils}
\usepackage{pgf}
\logo{\pgfputat{\pgfxy(11.68,0.15)}{\includegraphics[height=1.01cm,viewport=0 0 140 120,clip]{Figures/BCC_logo-1.png}}\pgfputat{\pgfxy(10.502,-0.218)}{\includegraphics[height=0.369cm,viewport=140 0 540 120,clip]{Figures/BCC_logo-1.png}}}
%\logo{\pgfputat{\pgfxy(11.68,0.15)}{\includegraphics[height=0.95cm,viewport=0 0 510 360,clip]{Figures/Logo_Gainstrong.png}}\pgfputat{\pgfxy(10.333,-0.195)}{\includegraphics[height=0.35cm,viewport=530 70 1100 218,clip]{Figures/Logo_Gainstrong.png}}}
%\logo{\pgfputat{\pgfxy(10.28,0.00)}{\includegraphics[height=0.95cm,viewport=0 0 1100 360,clip]{Figures/Logo_Gainstrong.png}}}
%\logo{\pgfputat{\pgfxy(11.68,0.15)}{\includegraphics[height=0.95cm,viewport=0 0 510 360,clip]{Figures/Logo_Gainstrong.png}}\pgfputat{\pgfxy(10.333,-0.195)}{\includegraphics[height=0.35cm,viewport=530 70 1100 218,clip]{Figures/Logo_Gainstrong.png}}}
%\logo{\pgfputat{\pgfxy(10.68,0.00)}{\includegraphics[height=1.20cm,viewport=0 15 400 430,clip]{Figures/seal_Jiang-2.jpg}}
      %\pgfputat{\pgfxy(10.817,-0.218)}{\includegraphics[height=0.47cm,viewport=20 0 670 350,clip]{Figures/signature_Jiang_new.jpg}}
%}
%\logo{\pgfputat{\pgfxy(11.68,0.252)}{\includegraphics[height=0.89cm,viewport=0 15 810 800,clip]{Figures/seal_Jiang-new.jpg}}\pgfputat{\pgfxy(10.762,-0.218)}{\includegraphics[height=0.49cm,viewport=20 0 670 350,clip]{Figures/signature_Jiang_new.jpg}}}
%\logo{\pgfputat{\pgfxy(11.68,0.252)}{\includegraphics[height=0.90cm,viewport=0 15 400 430,clip]{Figures/seal_Jiang-2.jpg}}\pgfputat{\pgfxy(10.817,-0.218)}{\includegraphics[height=0.47cm,viewport=20 0 670 350,clip]{Figures/signature_Jiang_new.jpg}}}
%\MyLogo{
%	\pgfputat{\pgfxy(-50,-50)}{\pgfbox[right,base]{\includegraphics[height=1cm]{Figures/BCC_logo-1.png}}}

%logo作为背景放置
%\setbeamertemplate{background}{
%	\pgfputat{\pgfxy(6.5,-0.5)}{\pgfbox[left,top]{\pgfimage[height=1.1cm]{Figures/BCC_logo-1.png}}}}

%\logo{}									%不显示logo

%-----------------------------------------------------------------------------

%-----------------------------------------------------------------------------------------------------------------------------------------------%

%----------------------------------------------  PPT Comments  ----------------------------------------------------------------------------------
\include{Beamer_Template/Beamer_Template-Comment}
%-----------------------------------------------------------------------------------------------------------------------------------------------%

%----------------------------------------------  PPT Fonts  ----------------------------------------------------------------------------------
\include{Beamer_Template/Beamer_Template-Fonts}
%-----------------------------------------------------------------------------------------------------------------------------------------------%

\begin{document}
%---------------------------------------------- PPT FrameWorks ----------------------------------------------------------------------------------
%\begin{CJK*}{GBK}{song}
%\begin{CJK*}{GBK}{kai}
%beamer下不能用\songyi、\zihao等命令!
%\graphicspath{Figures/}

%\renewcommand{\figurename}{\tiny\CJKfamily{hei} 图.}
\renewcommand{\figurename}{\tiny{\bf Fig}.}
%\renewcommand{\tablename}{\tiny\CJKfamily{hei} 表.}
\renewcommand{\tablename}{\tiny{\bf Tab}.}
%\renewcommand{\tablename}{\tiny\CJKfamily{hei} 表.}
%\renewcommand{\thesubfigure}{\roman{subfigure}}  %\makeatletter 子图标记罗马字母
\renewcommand{\thesubfigure}{\tiny(\alph{subfigure})}  %\makeatletter 子图标记英文字母
%\renewcommand{\thesubfigure}{}  \makeatletter %子图无标记

%%%%%%%%%%%%%%%%%%%%%%%%%%%%%%% Latex 的 tikz 绘图 %%%%%%%%%%%%%%%%%%%%%%%%%%%%%%%%%%%%%%%%%%%
%\begin{tikzpicture}
%    % 引入图片
%    \node[anchor=south west,inner sep=0] (image) at (0,0) {\includegraphics[width=0.9\textwidth]{Mycena_interrupta.jpg}};
%
%    \begin{scope}[x={(image.south east)},y={(image.north west)}]
%        % 建立相对坐标系
%        \draw[help lines,xstep=.1,ystep=.1] (0,0) grid (1,1);
%        \foreach \x in {0,1,...,9} { \node [anchor=north] at (\x/10,0) {0.\x}; }
%        \foreach \y in {0,1,...,9} { \node [anchor=east] at (0,\y/10) {0.\y}; }
%        % 作图命令
%        \draw[red, ultra thick, rounded corners] (0.62,0.65) rectangle (0.78,0.75);
%    \end{scope}
%\end{tikzpicture}
%指定位置插入图片
%\begin{tikzpicture}[remember picture,overlay]
%    \node<1->[xshift=-3cm,yshift=-1cm] at (current page.center) {\includegraphics[height=3cm]{Figures/Huang.jpg}};
%    \node<2->[xshift=0cm,yshift=0cm] at (current page.east) {\includegraphics[height=3cm]{Figures/Huang.jpg}};
%    \node<3->[xshift=3cm,yshift=1cm] at (current page.north) {\includegraphics[height=3cm]{Figures/Huang.jpg}};
%\end{tikzpicture}
%\begin{tikzpicture}[remember picture,overlay]
%	\node[xshift=-1.8cm,yshift=1.43cm] at (current page.east) {\includegraphics[height=2.5cm, width=2.22cm, viewport=360 0 1350 1100,clip]{Figures/Liu_Zengfu.jpg}};
%\end{tikzpicture}
%%%%%%%%%%%%%%%%%%%%%%%%%%%%%%%%%%%%%%%%%%%%%%%%%%%%%%%%%%%%%%%%%%%%%%%%%%%%%%%%%%%%%%%%%%%%%%


%-----------------------------------------------------------------------------------------------------------------------------------------------%

%-----------------------------------------------Autho_AND_Ttile----------------------------------------------------------------------------------
%%%%%%%%%%%%%%%%%%%%%%%%%%%%%%%%%%%%%%%%%%%  不使用 authblk 包制作标题  %%%%%%%%%%%%%%%%%%%%%%%%%%%%%%%%%%%%%%%%%%%%%%
%-------------------------------PPT Title-------------------------------------
\title{基于\rm{DFT}的第一原理计算方法简介}
%-----------------------------------------------------------------------------

%----------------------------Author & Date------------------------------------
\author[\textrm{Jun\_Jiang}]{姜\;\;骏\inst{}} %[]{} (optional, use only with lots of authors)
%% - Give the names in the same order as the appear in the paper.
%% - Use the \inst{?} command only if the authors have different
%%   affiliation.
\institute[BCC]{\inst{}%
 \vskip -20pt 北京市计算中心}
\date[\today] % (optional, should be abbreviation of conference name)
{	{\fontsize{6.2pt}{4.2pt}\selectfont{\textcolor{blue}{E-mail:~}\url{jiangjun@bcc.ac.cn}}}
\vskip 45 pt {\fontsize{8.2pt}{6.2pt}\selectfont{%清华大学\;\;物理系% 报告地点
	\vskip 5 pt \textrm{2022.06}}}
}

%% - Either use conference name or its abbreviation
%% - Not really information to the audience, more for people (including
%%   yourself) who are reading the slides onlin%%   yourself) who are reading the slides onlin%%   yourself) who are reading the slides onlineee
%%%%%%%%%%%%%%%%%%%%%%%%%%%%%%%%%%%%%%%%%%%%%%%%%%%%%%%%%%%%%%%%%%%%%%%%%%%%%%%%%%%%%%%%%%%%%%%%%%%%%%%%%%%%%%%%%%%%%

\subject{}
% This is only inserted into the PDF information catalog. Can be left
% out.
%\maketitle
\frame
{
%	\frametitle{\fontsize{9.5pt}{5.2pt}\selectfont{\textcolor{orange}{“高通量并发式材料计算算法与软件”年度检查}}}
\titlepage
}
%-----------------------------------------------------------------------------

%------------------------------------------------------------------------------列出全文 outline ---------------------------------------------------------------------------------
\section*{}
\frame[allowframebreaks]
{
  \frametitle{Outline}
%  \frametitle{\textcolor{mycolor}{\secname}}
  \tableofcontents%[current,currentsection,currentsubsection]
}
%在每个section之前列出全部Outline
%类似的在每个subsection之前列出全部Outline是\AtBeginSubsection[]
\AtBeginSection[]
{
  \frame<handout:0>%[allowframebreaks]
  {
    \frametitle{Outline}
%全部Outline中,本部分加亮
    \tableofcontents[current,currentsection]
  }
}


%-----------------------------------------------------------------------------------------------------------------------------------------------%

%-----------------------------------------------PPT main Body------------------------------------------------------------------------------------
%Main_File
%%%%%%%%%%  2024-09-04   %%%%%%%%%%  
%%%%%%%%%%%%%%%%%%%%%%%%%%%%%%%%%%%%%%%%%%  不使用 authblk 包制作标题  %%%%%%%%%%%%%%%%%%%%%%%%%%%%%%%%%%%%%%%%%%%%%%
%-------------------------------PPT Title-------------------------------------
\title{\rm{\ch{Ni}}表面\rm{\ch{CO2}}的吸附}
%-----------------------------------------------------------------------------

%----------------------------Author & Date------------------------------------
%\author[\textrm{Jun\_Jiang}]{姜\;\;骏\inst{}} %[]{} (optional, use only with lots of authors)
%% - Give the names in the same order as the appear in the paper.
%% - Use the \inst{?} command only if the authors have different
%%   affiliation.
\institute[BCC]{\inst{}%
%\institute[Gain~Strong]{\inst{}%
\vskip 0pt 北京市计算中心有限公司~材料计算团队}
%\vskip -20pt {\large 格致斯创~科技}}
\date[\today] % (optional, should be abbreviation of conference name)
{	%{\fontsize{6.2pt}{4.2pt}\selectfont{\textcolor{blue}{E-mail:~}\url{jiangjun@bcc.ac.cn}}}
\vskip 45 pt {\fontsize{8.2pt}{6.2pt}\selectfont{%北京科技大学% 报告地点
	\vskip 5 pt \textrm{2024.09.04}}}
}

%% - Either use conference name or its abbreviation
%% - Not really information to the audience, more for people (including
%%   yourself) who are reading the slides onlin%%   yourself) who are reading the slides onlin%%   yourself) who are reading the slides onlineee
%%%%%%%%%%%%%%%%%%%%%%%%%%%%%%%%%%%%%%%%%%%%%%%%%%%%%%%%%%%%%%%%%%%%%%%%%%%%%%%%%%%%%%%%%%%%%%%%%%%%%%%%%%%%%%%%%%%%%
\subject{}
% This is only inserted into the PDF information catalog. Can be left
% out.
%\maketitle
\frame
{
%	\frametitle{\fontsize{9.5pt}{5.2pt}\selectfont{\textcolor{orange}{“高通量并发式材料计算算法与软件”年度检查}}}
\titlepage
}
%-----------------------------------------------------------------------------

%------------------------------------------------------------------------------列出全文 outline ---------------------------------------------------------------------------------
%\section*{}
%\frame[allowframebreaks]
%{
%  \frametitle{}
%%  \frametitle{\textcolor{mycolor}{\secname}}
%  \tableofcontents%[current,currentsection,currentsubsection]
%}
%在每个section之前列出全部Outline
%类似的在每个subsection之前列出全部Outline是\AtBeginSubsection[]
%\AtBeginSection[]
%{
%  \frame<handout:0>%[allowframebreaks]
%  {
%    \frametitle{Outline}
%%全部Outline中,本部分加亮
%    \tableofcontents[current,currentsection]
%  }
%}

%-----------------------------------------------PPT main Body------------------------------------------------------------------------------------
\small
\begin{frame}[allowframebreaks]
	\frametitle{前期计算结果}
	\begin{itemize}
		\item 文献和我们的计算表明,甲醛基~\textrm{(\ch{HCO}$\cdot$)}的生成应该在\textrm{\ch{CO}}产生之后;~初始反应时,甲醛基~\textrm{(\ch{HCO}$\cdot$)}的生成比甲酸基~\textrm{(\ch{HCOO}$\cdot$)}和羧酸基~\textrm{($\cdot$\ch{COOH})}困难
		\item 
			\begin{enumerate}
				\item 基底吸附\textrm{\ch{CO2}}的基态能量 $\mathrm{E}_{sub}^{\mathrm{CO_2}}=-1035.594648~\mathrm{eV}$
				\item 反应中间体的基态能量 $\mathrm{E}_{sub}^{\mathrm{\chemfig{C(=[1,0.7]O)-[,0.7]OH}}}=-1037.7071~\mathrm{eV}$
				\item 反应中间体的基态能量 $\mathrm{E}_{sub}^{\mathrm{\chemfig{HC(=[1,0.7]O)-[,0.7]O}}}=-1038.5303~\mathrm{eV}$
				\item 基底吸附\textrm{CO}的基态能量 $\mathrm{E}_{sub}^{\mathrm{CO}}=-1028.222503~\mathrm{eV}$
			\end{enumerate}
		\item \textcolor{blue}{在\textrm{\ch{CO}}生成及后续可能的反应通道}\\
	计算:~\textcolor{red}{实验关心的反应历程}
	\end{itemize}
\end{frame}

\begin{frame}[allowframebreaks]
	\frametitle{计算验证}
%	{\huge
%		\setchemfig{atom sep=2em, bond style={line width=1pt, red, dash pattern=on 2pt off 2pt}}
%		\chemname{\chemfig{H-C(-[2]H)(-[6]H)-C(=[1]O)-[7]H}}{Acetaldehyde}}
	为了检验\textrm{\ch{CO}}与\textrm{H}的反应历程,设计方案
\begin{itemize}
	\item \textrm{\ch{CO}}加\textrm{\ch{H}}的能力:\\加在\textrm{\ch{C}}端(生成\textrm{\chemfig{H-[,0.3]C(=[1,0.7]O)-[6,0.3]}})%\chemfig{R-[:30]*-[:180]H}\\
		~\textrm{\textcolor{blue}{vs}}~加在\textrm{\ch{O}}端(生成\textrm{\chemfig{C(=[1,0.7]O-[,0.3]H)(-[5,0.3])-[7,0.3]}})
	\item 类比\textrm{\ch{CN}}和\textrm{\ch{NO}}的两端加\textrm{\ch{H}}能力
	\item 系统类比加\textrm{\ch{H}}引起体系能量和电荷密度的变化
\end{itemize}
\begin{figure}[h!]
\centering
\begin{tikzpicture}[
%    box/.style={rectangle,draw,node distance=1cm,text width=15em,text centered,rounded corners,minimum height=2em,thick},  %文字居中
    box/.style={rectangle,draw,node distance=1cm,text width=18em,anchor=west,rounded corners,minimum height=5em,thick},   %文字左对齐
    arrow/.style={draw,-latex', red, line width=2pt},
]
\node [box](box){};
\node [anchor=west, text width=1em] (H1) {\textrm{\ch{H}}};  %% anchor=west 文字左对齐
\node [right=2 of H1] (CO) {\textrm{\chemfig{C~O}}};
\path [arrow, draw=red, line width =0.5pt] (10.5em,0.2em) -- (8.5em, 0.2em);
\node [right=5.5 of H1] (H2) {\textrm{\ch{H}}};
\node [above=0.05 of CO] (CN) {\textrm{\chemfig{C(~N)}}};
\draw [draw=red] (7.6em,1.6em) circle [radius=0.1em];
\node [below=0.05 of CO] (NO) {\textrm{\chemfig{N(=O)}}};
\filldraw [fill=red, draw=red] (9.6em,-1.2em) circle [radius=0.1em];
\filldraw [fill=red, draw=red] (11.7em,-1.4em) circle [radius=0.1em];
\filldraw [fill=red, draw=red] (11.7em,-1.8em) circle [radius=0.1em];
% \path [arrow] (Method) -- (Softwares);
  \path [arrow, dotted] (CO) -- (H1);
  \path [arrow, draw=blue, dotted] (CO) -- (H2);
\end{tikzpicture}
\caption{\tiny{\textrm{\ch{CO}、\ch{CN}、\ch{NO}}分子两端加\textrm{\ch{H}}的示意,不同距离的\textrm{X-H~(X=C、N、O)},示意加\textrm{H}的动力学过程}}
\label{Molecules}
\end{figure}
反应模型:~活化的\textrm{\ch{CO}}与\textrm{\ch{H}}
\begin{figure}[h!]
\centering
%\vspace*{-0.10in}
\includegraphics[height=2.00in,width=1.9in, viewport=1870 350 2950 1500, clip]{/home/jun-jiang/BCC/2023-NICE/Ni-CO2/图片/能量测试结构/活化C-O---H.png}
\includegraphics[height=2.00in,width=1.9in, viewport=1870 350 2950 1500, clip]{/home/jun-jiang/BCC/2023-NICE/Ni-CO2/图片/能量测试结构/活化H---C-O.png}
\caption{\tiny \textrm{The front view of model for \ch{CO}-activated compounded with \ch{H} by \ch{O}-end (left) and \ch{C}-end (right).}}%(与文献\cite{EPJB33-47_2003}图1对比)
\label{Model:CO-H}
\end{figure}
反应模型:~活化的\textrm{\ch{CN}}与\textrm{\ch{H}}
\begin{figure}[h!]
\centering
%\vspace*{-0.10in}
\includegraphics[height=2.00in,width=1.9in, viewport=1870 350 2950 1500, clip]{/home/jun-jiang/BCC/2023-NICE/Ni-CO2/图片/能量测试结构/活化C-N---H.png}
%\includegraphics[height=2.00in,width=1.9in, viewport=1870 350 2950 1500, clip]{/home/jun-jiang/BCC/2023-NICE/Ni-CO2/图片/能量测试结构/活化H---C-O.png}
\caption{\tiny \textrm{The front view of model for \ch{CN}-activated compounded with \ch{H} by \ch{N}-end.}}%(与文献\cite{EPJB33-47_2003}图1对比)
\label{Model:CN-H}
\end{figure}
反应模型:~活化的\textrm{\ch{NO}}与\textrm{\ch{H}}
\begin{figure}[h!]
\centering
%\vspace*{-0.10in}
\includegraphics[height=2.00in,width=1.9in, viewport=1870 350 2950 1500, clip]{/home/jun-jiang/BCC/2023-NICE/Ni-CO2/图片/能量测试结构/活化N-O---H.png}
\includegraphics[height=2.00in,width=1.9in, viewport=1870 350 2950 1500, clip]{/home/jun-jiang/BCC/2023-NICE/Ni-CO2/图片/能量测试结构/活化H---N-O.png}
\caption{\tiny \textrm{The front view of model for \ch{NO}-activated compounded with \ch{H} by \ch{O}-end (left) and \ch{N}-end (right).}}%(与文献\cite{EPJB33-47_2003}图1对比)
\label{Model:NO-H}
\end{figure}

差分电荷表示:~活化的\textrm{\ch{CO}}与\textrm{\ch{H}}反应
\begin{figure}[h!]
\centering
%\vspace*{-0.10in}
\includegraphics[height=2.00in,width=1.9in, viewport=1870 850 2950 1880, clip]{/home/jun-jiang/BCC/2023-NICE/Ni-CO2/图片/差分电荷/俯视活化C-O---H.png}
\includegraphics[height=2.00in,width=1.9in, viewport=1870 480 2950 1780, clip]{/home/jun-jiang/BCC/2023-NICE/Ni-CO2/图片/差分电荷/正视活化C-O---H.png}
\caption{\tiny \textrm{The top-view (left) and front-view (right) of charge-density difference for \ch{CO}-activated compounded with \ch{H}.}}%(与文献\cite{EPJB33-47_2003}图1对比)
\label{Charge-density_difference:CO}
\end{figure}
差分电荷表示:~活化的\textrm{\ch{NO}}与\textrm{\ch{H}}反应
\begin{figure}[h!]
\centering
%\vspace*{-0.10in}
\includegraphics[height=2.00in,width=1.9in, viewport=1870 850 2950 1880, clip]{/home/jun-jiang/BCC/2023-NICE/Ni-CO2/图片/差分电荷/俯视活化N-O---H.png}
\includegraphics[height=2.00in,width=1.9in, viewport=1870 480 2950 1780, clip]{/home/jun-jiang/BCC/2023-NICE/Ni-CO2/图片/差分电荷/正视活化N-O---H.png}
\caption{\tiny \textrm{The top-view (left) and front-view (right) of charge-density difference for \ch{NO}-activated compounded with \ch{H}.}}%(与文献\cite{EPJB33-47_2003}图1对比)
\label{Charge-density_difference:NO}
\end{figure}
能量-键长呈现的反应动力学可能性
\begin{figure}[h!]
\centering
%\vspace*{-0.10in}
\includegraphics[height=2.10in,width=4.0in, viewport=0 0 360 200, clip]{/home/jun-jiang/BCC/2023-NICE/Ni-CO2/X-H_E.eps}
\caption{\tiny 金属表面吸附的\textrm{\ch{CO}、\ch{CN}、\ch{NO}}分子与\textrm{\ch{H}}相互作用的能量随\textrm{X-H~(X=C、N、O)}间距的变化}%\textrm{The energy of  of \ch{CO2} activation over \ch{Ni}.}}%(与文献\cite{EPJB33-47_2003}图1对比)
\label{X-H_E}
\end{figure}
\newpage
对比金属表面吸附的\textrm{\ch{CN}}、\textrm{\ch{CO}}、\textrm{\ch{NO}}
\begin{itemize}
	\item \textrm{\ch{CO}}更明显地倾向\textrm{\ch{C}}端与\textrm{\ch{H}}优先反应
%	\item \textrm{\ce{CO2->[+H2]COOH}\ce{->[-H2O]CO}\ce{->[+H2]COH}}
	\item \textrm{\ce{CO2 ->[\ce{+H2}] COOH ->[\ce{-H2O}] CO ->[\ce{+H2}] \chemfig{H-[,0.3]C(=[1,0.7]O)-[6,0.3]} ->[\ce{+H2}] $\cdots$ -> CH4}}
\end{itemize}

\end{frame}


%-----------------------------------------------------------------------------------------------------------------------------------------------%

%-----------------------------------------Reference----------------------------------------------------------------------------------------------
%\begin{thebibliography}{99}
%-----------------------------------------------------------------------------------------------------------------------------------------------%
%\frame
%{
%\frametitle{主要参考文献}
%{\small
%\bibitem{Singh_Book}\textrm{D. J. Singh. \textit{Plane Wave, PseudoPotential and the LAPW method} (Kluwer Academic, Boston,USA, 1994)}					%
%  \nocite{*}																				%
%}
%\end{thebibliography}

%-------------------------------------------------Biblography------------------------------------------------------------------------------------
%%%%%%%%% 参考文献必须用 INPUT 模式,如果用 INCLUDE 会导致引用标号出问题 %%%%%%%%%%%%%%
%\appendix
%------------------------------------------------------------------------Reference----------------------------------------------------------------------------------------------
%-----------------------------------------------------------Beamer下不建议使用bib,因为涉及分页--------------------------------------------------------------------------%
%\begin{thebibliography}{99}								      %
\frame[allowframebreaks]
{
\frametitle{主要参考文献}
{\tiny\textrm{
%\setlength{\bibsep}{0em}  %%设置参考文献间距
%--------%%%%%%%%%%%%%%%%%%%%%%%%%%%%%%%%%%%%%%%%%%%%%--------%
%%\bibitem{PRL58-65_1987}H.Feil, C. Haas, {\it Phys. Rev. Lett.} {\bf 58}, 65 (1987).         %
%\bibitem{kp-method} \textrm{Zhenxi Pan, Yong Pan, Jun Jiang$^{\ast}$, Liutao Zhao}, \textrm{High-Throughput Electronic Band Structure Calculations for Hexaborides}, \textit{Intelligent Computing}, \textbf{Springer}, \textbf{P.386-395}, (2019).              %
%\bibitem{QCQC_2014} \textrm{姜骏},\textrm{PAW原子数据集的构造与检验}, \textit{中国化学会第十二届全国量子化学会议论文摘要集},\textbf{太原},(2014).
%											      %

%\phantomsection\addcontentsline{toc}{section}{Bibliography}	 %直接调用\addcontentsline命令可能导致超链指向不准确,一般需要在之前调用一次\phantomsection命令加以修正	%
%\phantomsection\addcontentsline{toc}{section}{主要参考文献}	 %直接调用\addcontentsline命令可能导致超链指向不准确,一般需要在之前调用一次\phantomsection命令加以修正	%
	\bibliography{$PATHPWD/Myref}				%%$PATH用于脚本指定图片路径
%	\bibliography{../ref/Myref}%
\bibliographystyle{$PATHPWD/mybib}				%%$PATH用于脚本指定图片路径
%\bibliographystyle{../ref/mybib}%
}}
%\printbibliography[heading=bibintoc, title=参考文献]    %%在目录列表中列出参考文献,title可自定义

%\nocite{*}%
}
%\end{thebibliography}								              %
%-----------------------------------------------------------------------------

%-----------------------------------------------------------------------------------------------------------------------------------------------%

%-------------------------------------------------Acknowledge------------------------------------------------------------------------------------
%------------------------------------------------- Acknowledge ----------------------------------------------------------------------------
%\section{致谢}
%\frame
%{
%\frametitle{致$\quad$谢}
%\begin{itemize}
%    \setlength{\itemsep}{20pt}
%  \item 感谢本团队高兴誉、吴泉生、宋红州等各位老师参与的讨论
%  \item 感谢莫所长、宋主任以及软件中心各位老师和同事
%  \item 感谢王崇愚先生的帮助
%\end{itemize}
%}

\logo{}									%不显示logo
\frame
{
\vskip 60 pt
%\hskip 10pt \textcolor{blue}{\Huge 感谢答辩委员会各位老师\,\textrm{!}}\\
\vskip 35 pt
\hskip 60pt \textcolor{blue}{\Huge 谢谢大家\:!}
%\vskip 15 pt
%\hskip 40pt \textcolor{blue}{\Huge \textrm{for your attention\:!}}
}

%\frame
%{
%\begin{figure}[h!]
%\centering
%\animategraphics[autoplay, loop, height=2.1in]{1}{Figures/Prof_Liu-}{06}{11}
%\label{Prof_Liu}
%\end{figure}
%}
%

%-----------------------------------------------------------------------------------------------------------------------------------------------%

\clearpage
%\end{CJK*}
\end{document}
