%%%%%%%%%%%%%%%%%%%%%%%%%%%%%%%%%%%%%%%%%%  不使用 authblk 包制作标题  %%%%%%%%%%%%%%%%%%%%%%%%%%%%%%%%%%%%%%%%%%%%%%
%-------------------------------PPT Title-------------------------------------
\title{模型微调简介}
%-----------------------------------------------------------------------------

%----------------------------Author & Date------------------------------------
\author[]{\vskip +10pt 姜\;\;骏\inst{}} %[]{} (optional, use only with lots of authors)
%% - Give the names in the same order as the appear in the paper.
%% - Use the \inst{?} command only if the authors have different
%%   affiliation.
\institute[BCC]{\inst{}%
%\institute[Gain~Strong]{\inst{}%
\vskip -15pt 北京市计算中心}
%\vskip -20pt {\large 格致斯创~科技}}
\date[\today] % (optional, should be abbreviation of conference name)
{	{\fontsize{6.2pt}{4.2pt}\selectfont{\textcolor{blue}{E-mail:~}\url{jiangjun@bcc.ac.cn}}}
\vskip 45 pt {\fontsize{8.2pt}{6.2pt}\selectfont{%清华大学\;\;物理系% 报告地点
	\vskip 5 pt \textrm{2025.02}}}
}

%% - Either use conference name or its abbreviation
%% - Not really information to the audience, more for people (including
%%   yourself) who are reading the slides onlin%%   yourself) who are reading the slides onlin%%   yourself) who are reading the slides onlineee
%%%%%%%%%%%%%%%%%%%%%%%%%%%%%%%%%%%%%%%%%%%%%%%%%%%%%%%%%%%%%%%%%%%%%%%%%%%%%%%%%%%%%%%%%%%%%%%%%%%%%%%%%%%%%%%%%%%%%

\subject{}
% This is only inserted into the PDF information catalog. Can be left
% out.
%\maketitle
\frame
{
%	\frametitle{\fontsize{9.5pt}{5.2pt}\selectfont{\textcolor{orange}{“高通量并发式材料计算算法与软件”年度检查}}}
\titlepage
}
%-----------------------------------------------------------------------------

%------------------------------------------------------------------------------列出全文 outline ---------------------------------------------------------------------------------
\section*{}
\frame[allowframebreaks]
{
	\frametitle{\textrm{Outline}}
%  \frametitle{\textcolor{mycolor}{\secname}}
  \tableofcontents%[current,currentsection,currentsubsection]
}
%在每个section之前列出全部Outline
%类似的在每个subsection之前列出全部Outline是\AtBeginSubsection[]
%\AtBeginSection[]
%{
%  \frame<handout:0>%[allowframebreaks]
%  {
%    \frametitle{Outline}
%%全部Outline中,本部分加亮
%    \tableofcontents[current,currentsection]
%  }
%}

%-----------------------------------------------PPT main Body------------------------------------------------------------------------------------
\small
\documentclass{beamer}
\usepackage[utf8]{inputenc}
\usepackage{graphicx}
\usepackage{listings}
\usepackage{xcolor}

% 设置代码高亮
\lstdefinestyle{pythonstyle}{
    language=Python,
    basicstyle=\ttfamily\footnotesize,
    keywordstyle=\color{blue},
    commentstyle=\color{green!50!black},
    stringstyle=\color{red},
    numbers=left,
    numberstyle=\tiny\color{gray},
    stepnumber=1,
    numbersep=5pt,
    backgroundcolor=\color{gray!10},
    showspaces=false,
    showstringspaces=false,
    showtabs=false,
    tabsize=2
}

\usetheme{Madrid}
\title{Python 与 OpenAI API 的使用说明}
\author{Your Name}
\date{\today}

\begin{document}

\begin{frame}
    \titlepage
\end{frame}

\begin{frame}{目录}
    \tableofcontents
\end{frame}

\section{引言}
\begin{frame}{引言}
    \begin{itemize}
        \item 介绍 Python 和 OpenAI API 的重要性
        \item 说明本演示的目标和范围
    \end{itemize}
\end{frame}

\section{OpenAI API 概述}
\begin{frame}{OpenAI API 概述}
    \begin{itemize}
        \item 什么是 OpenAI API
        \item 主要功能和应用场景
        \item 不同模型的特点(如 GPT - 3.5, GPT - 4 等)
    \end{itemize}
\end{frame}

\section{Python 环境准备}
\begin{frame}{Python 环境准备}
    \begin{itemize}
        \item 安装 Python(建议 Python 3.7 及以上版本)
        \item 安装 OpenAI Python 库
    \end{itemize}
    \begin{lstlisting}[style=pythonstyle]
pip install openai
    \end{lstlisting}
\end{frame}

\section{API 密钥设置}
\begin{frame}{API 密钥设置}
    \begin{itemize}
        \item 在 OpenAI 平台获取 API 密钥
        \item 在 Python 代码中设置 API 密钥
    \end{itemize}
    \begin{lstlisting}[style=pythonstyle]
import openai

openai.api_key = "YOUR_API_KEY"
    \end{lstlisting}
\end{frame}

\section{基本请求示例}
\begin{frame}{基本请求示例}
    以下是一个简单的使用 OpenAI API 生成文本的示例:
    \begin{lstlisting}[style=pythonstyle]
import openai

openai.api_key = "YOUR_API_KEY"

response = openai.Completion.create(
    engine="text-davinci-003",
    prompt="Once upon a time",
    max_tokens=50
)

print(response.choices[0].text)
    \end{lstlisting}
\end{frame}

\section{高级使用技巧}
\begin{frame}{高级使用技巧 - 调整参数}
    \begin{itemize}
        \item \textbf{temperature}: 控制生成文本的随机性。取值范围从 0 到 1,值越大,生成的文本越随机、越有创意;值越小,生成的文本越确定、越保守。
        \begin{lstlisting}[style=pythonstyle]
import openai

openai.api_key = "YOUR_API_KEY"

# 低 temperature,生成文本较保守
response_low_temp = openai.Completion.create(
    engine="text-davinci-003",
    prompt="Write a short story about a cat",
    max_tokens=100,
    temperature=0.2
)

# 高 temperature,生成文本较随机
response_high_temp = openai.Completion.create(
    engine="text-davinci-003",
    prompt="Write a short story about a cat",
    max_tokens=100,
    temperature=0.8
)

print("Low temperature result:", response_low_temp.choices[0].text)
print("High temperature result:", response_high_temp.choices[0].text)
        \end{lstlisting}
    \end{itemize}
\end{frame}

\begin{frame}{高级使用技巧 - 调整参数(续)}
    \begin{itemize}
        \item \textbf{top_p}: 也称为核采样,是另一种控制生成文本随机性的方法。它会从累积概率超过指定值(如 0.9)的词中进行采样。
        \begin{lstlisting}[style=pythonstyle]
import openai

openai.api_key = "YOUR_API_KEY"

# 低 top_p,生成文本较集中
response_low_top_p = openai.Completion.create(
    engine="text-davinci-003",
    prompt="Describe a beautiful sunset",
    max_tokens=100,
    top_p=0.2
)

# 高 top_p,生成文本较分散
response_high_top_p = openai.Completion.create(
    engine="text-davinci-003",
    prompt="Describe a beautiful sunset",
    max_tokens=100,
    top_p=0.8
)

print("Low top_p result:", response_low_top_p.choices[0].text)
print("High top_p result:", response_high_top_p.choices[0].text)
        \end{lstlisting}
    \end{itemize}
\end{frame}

\section{错误处理和最佳实践}
\begin{frame}{错误处理和最佳实践}
    \begin{itemize}
        \item 处理 API 请求中的错误(如网络错误、权限错误等)
        \item 遵循 OpenAI 的使用政策和最佳实践
    \end{itemize}
\end{frame}

\section{总结}
\begin{frame}{总结}
    \begin{itemize}
        \item 回顾 Python 与 OpenAI API 的使用要点
        \item 鼓励进一步探索和实践
    \end{itemize}
\end{frame}

\end{document}
