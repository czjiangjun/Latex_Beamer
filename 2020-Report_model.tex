\documentclass[10pt, oneside, a4paper]{article}      % Specifies the document class
%\documentclass[10pt, twoside, a4paper]{article}      % Specifies the document class

%%%%%%%%%%%%%%%%% CJK 中文版面控制  %%%%%%%%%%%%%%%%%%%%%%%%%%%%%%
%\usepackage{CJK} % CTEX-CJK 中文支持                            %
\usepackage{xeCJK} % seperate the english and chinese		 %
\usepackage{CJKutf8} % Texlive 中文支持                         %
\usepackage{CJKnumb} %中文序号                                   %
\usepackage{indentfirst} % 中文段落首行缩进                      %
%\setlength\parindent{22pt}       % 段落起始缩进量               %
\renewcommand{\baselinestretch}{1.2} % 中文行间距调整            %
\setlength{\textwidth}{16cm}                                     %
\setlength{\textheight}{24cm}                                    %
\setlength{\topmargin}{-1cm}                                     %
\setlength{\oddsidemargin}{0.1cm}                                %
\setlength{\evensidemargin}{\oddsidemargin}                      %
\usepackage{fancyhdr}           %使用页眉-页脚                   %
%%%%%%%%%%%%%%%%%%%%%%%%%%%%%%%%%%%%%%%%%%%%%%%%%%%%%%%%%%%%%%%%%%

\usepackage{amsmath,amsthm,amsfonts,amssymb,bm}          %数学公式
\usepackage{mathrsfs}                                    %英文花体
\usepackage{xcolor}                                        %使用默认允许使用颜色
%\usepackage{hyperref} 
\usepackage{graphicx}
\usepackage{subfigure}           %图片跨页
\usepackage{animate}		 %插入动画
\usepackage{caption}
\captionsetup{font=footnotesize}

%\usepackage[version=3]{mhchem}		%化学公式
\usepackage{chemformula}
\usepackage{chemfig}		%化学公式

\usepackage{fontspec} % use to set font
\setCJKmainfont{SimSun}
\XeTeXlinebreaklocale "zh"  % Auto linebreak for chinese
\XeTeXlinebreakskip = 0pt plus 1pt % Auto linebreak for chinese

\usepackage{longtable}                                   %使用长表格
\usepackage{multirow}
\usepackage{makecell}		%允许单元格内换行

%%%%%%%%%%%%%%%%%%%%%%%%%  参考文献引用 %%%%%%%%%%%%%%%%%%%%%%%%%%%
%%尽量使用 BibTeX(含有超链接,数据库的条目URL即可)                %
%%%%%%%%%%%%%%%%%%%%%%%%%%%%%%%%%%%%%%%%%%%%%%%%%%%%%%%%%%%%%%%%%%%

\usepackage[numbers,sort&compress]{natbib} %紧密排列             %
\usepackage[sectionbib]{chapterbib}        %每章节单独参考文献   %
%\usepackage{footbib}			   %脚注列出参考文献    %
\usepackage{hypernat}                                                                         %
\usepackage[bookmarksopen=true,pdfstartview=FitH,CJKbookmarks]{hyperref}              %
\hypersetup{bookmarksnumbered,colorlinks,linkcolor=green,citecolor=blue,urlcolor=red}         %
%参考文献含有超链接引用时需要下列宏包,注意与natbib有冲突        %
%\usepackage[dvipdfm]{hyperref}                                  %
%\usepackage{hypernat}                                           %
\newcommand{\upcite}[1]{\hspace{0ex}\textsuperscript{\cite{#1}}} %

%%%%%%%%%%%%%%%%%%%%%%%%%%%%%%%%%%%%%%%%%%%%%%%%%%%%%%%%%%%%%%%%%%%%%%%%%%%%%%%%%%%%%%%%%%%%%%%
%\AtBeginDvi{\special{pdf:tounicode GBK-EUC-UCS2}} %CTEX用dvipdfmx的话,用该命令可以解决      %
%						   %pdf书签的中文乱码问题		      %
%%%%%%%%%%%%%%%%%%%%%%%%%%%%%%%%%%%%%%%%%%%%%%%%%%%%%%%%%%%%%%%%%%%%%%%%%%%%%%%%%%%%%%%%%%%%%%%

%%%%%%%%%%%%%%%%%%%%%  % 插图使用位置  %%%%%%%%%%%%%%%%%%%%%%%%%%%
\graphicspath{{Presentation_Beamer/}}                            %
%%%%%%%%%%%%%%%%%%%%%%%%%%%%%%%%%%%%%%%%%%%%%%%%%%%%%%%%%%%%%%%%%%

\begin{document}
%%%%%%%%%%%%%%%%%%%%%  % 页眉-页脚设计  %%%%%%%%%%%%%%%%%%%%%%%%%%%
\pagestyle{fancy}
%\lhead{\bfseries Result} %页眉左边位置内容,并加粗 
\chead{} % 页眉中间位置内容
\rhead{\includegraphics[scale=0.20]{Figures/BCC_logo-1.png}}%在此处插入logo.pdf图片 图片靠右
%\lfoot{}  %页脚
\cfoot{}
%\rfoot{}
%%%%%%%%%%%%%%%%%  % pagestyleR常用格式  %%%%%%%%%%%%%%%%%%%%%%%%%
%% empty 无页眉页脚
%% plain 无页眉,页脚为居中页码
%% headings 页眉为章节标题,无页脚
%% myheadings 页眉内容可自定义,无页脚
%%%%%%%%%%%%%%%%%%%%%%%%%%%%%%%%%%%%%%%%%%%%%%%%%%%%%%%%%%%%%%%%%%

%\begin{CJK}{UTF8}{gbsn} %针对文字编码为unix %CJK自带的utf-8简体字体有gbsn(宋体)和gkai(楷体)
%\begin{CJK}{GBK}{hei}	%针对文字编码为doc
%\CJKindent     %在CJK环境中,中文段落起始缩进2个中文字符
%\indent
%
\renewcommand{\abstractname}{\small{\CJKfamily{hei} 摘\quad 要}} %\CJKfamily{hei} 设置中文字体,字号用\big \small来设
\renewcommand{\refname}{\centering\CJKfamily{hei} 参考文献}
%\renewcommand{\figurename}{\CJKfamily{hei} 图.}
\renewcommand{\figurename}{{\bf Fig}.}
%\renewcommand{\tablename}{\CJKfamily{hei} 表.}
\renewcommand{\tablename}{{\bf Tab}.}
%\renewcommand{\thesubfigure}{\roman{subfigure}}  \makeatletter %子图标记罗马字母
%\renewcommand{\thesubfigure}{\tiny(\alph{subfigure})}  \makeatletter %子图标记英文字母
%\renewcommand{\thesubfigure}{}  \makeatletter %子图无标记

%将图表的Caption写成 图(表) Num. 格式
\makeatletter
\long\def\@makecaption#1#2{%
  \vskip\abovecaptionskip
  \sbox\@tempboxa{#1. #2}%
  \ifdim \wd\@tempboxa >\hsize
    #1. #2\par
  \else
    \global \@minipagefalse
    \hb@xt@\hsize{\hfil\box\@tempboxa\hfil}%
  \fi
  \vskip\belowcaptionskip}
\makeatother

\newcommand{\keywords}[1]{{\hspace{0\ccwd}\small{\CJKfamily{hei} 关键词:}{\hspace{2ex}{#1}}\bigskip}}

%%%%%%%%%%%%%%%%%%中文字体设置%%%%%%%%%%%%%%%%%%%%%%%%%%%
%默认字体 defalut fonts \TeX 是一种排版工具 \\		%
%{\bfseries 粗体 bold \TeX 是一种排版工具} \\		%
%{\CJKfamily{song}宋体 songti \TeX 是一种排版工具} \\	%
%{\CJKfamily{hei} 黑体 heiti \TeX 是一种排版工具} \\	%
%{\CJKfamily{kai} 楷书 kaishu \TeX 是一种排版工具} \\	%
%{\CJKfamily{fs} 仿宋 fangsong \TeX 是一种排版工具} \\	%
%%%%%%%%%%%%%%%%%%%%%%%%%%%%%%%%%%%%%%%%%%%%%%%%%%%%%%%%%

%\addcontentsline{toc}{section}{Bibliography}

%-------------------------------The Title of The Paper-----------------------------------------%
%报告题目
\title{报告题目:~}
%-----------------------------------------------------------------------------

%----------------------The Authors and the address of The Paper--------------------------------%
\author{
\small
%Author1, Author2, Author3\footnote{Communication author's E-mail} \\    %Authors' Names	       %
\small
%(The Address,City Post code)						%Address	       %
}
%\affil[$\dagger$]{清华大学~材料加工研究所~A213}
\date{}					%if necessary					       %
%----------------------------------------------------------------------------------------------%
\maketitle
\thispagestyle{fancy}   % 首页插入页眉页脚 

%-------------------------------------------------------------------------------The Abstract and the keywords of The Paper----------------------------------------------------------------------------%
\begin{abstract}
摘要内容
%2020-11-24:~中国科学院-力学研究所
%密度泛函理论自1960年代中旬确立以来,已经成为物理、化学和材料领域电子结构计算的重要工具。相比于传统的波函数理论,密度泛函理论以电子密度作为基本变量,有着更直观、明确的物理内涵,计算效率也更高。通过对密度泛函理论和Kohn-Sham方程的讨论,特别是交换-相关能泛函的改进和发展历史梳理,回顾密度泛函理论的概貌,简单介绍主要软件的基本框架。
%2020-11-28:~北京理工大学
%基于密度泛函理论(\textrm{DFT})发展的第一原理计算软件已经成为物理、化学和材料领域电子结构计算的重要工具。\textrm{LAPW}方法和\textrm{PAW}方法是目前材料电子计算的主要方法,围绕两种方法各自的基组构造、势能与总能量计算等问题,讨论基于密度泛函理论的相关软件(如\textrm{WIEN2k}、\textrm{VASP}等)的算法实现和基本框架。
\end{abstract}

%关键词
%\keywords{\textrm{DFT}, \textrm{LAPW}, \textrm{PAW}, 电子计算}

%----------------------------------------------------------------------------------------------------------------------------------------------------------------------------------------------------%

%------------------------------------------------------------------------------ main Body------------------------------------------------------------------------------------
%Introduction
%\section{Introduction}
%报告人简介
%1.~个人简历
\textbf{姜骏的个人简历}

\begin{minipage}[b]{0.75\textwidth}
{\fontsize{11.0pt}{10.0pt}\selectfont{姜骏,北京市计算中心~~副研究员}}
\vskip 4pt
2001年毕业于中国纺织大学~(现~东华大学)~染整专业,2008年~北京大学物理化学专业博士研究生毕业。\\
2008-2012年~北京大学博士后。\\
2013-2016年~中物院高性能数值模拟软件中心助理研究员,2016年4月起,北京市计算中心副研究员。
\end{minipage}
\hskip 15pt
%2.~个人近照
\begin{minipage}[b]{0.20\textwidth}
	\vspace{17pt}
\includegraphics[height=1.1in]{Figures/Person_Photo.JPG}
\end{minipage}
\vskip 12pt
%3.~个人研究内容与兴趣
姜骏长年从事第一原理计算方法和软件研究,熟悉密度泛函理论~(\textrm{Density Functional Theory, DFT})~和第一原理计算方法,特别是对赝势~(\textrm{Pseudo-Potential, PP})~方法、线性缀加平面波~(\textrm{Linearised Augmeneted Plane-Wave, LAPW})~方法和投影子缀加波~(\textrm{Projected~Augmented~Wave, PAW})~方法有系统深入的理解。2017年7月起,作为骨干研究人员承担科技部“材料基因工程关键技术与支撑平台”重点专项课题1“研发多尺度集成化高通量计算方法与计算软件”研究任务,从事高通量跨尺度并发式集成计算主体算法与软件开发。近年来的主要工作包括第一原理-分子动力学材料计算软件平台的设计与开发、晶体空间群软件开发和VASP软件的赝势重构等。

%4. 个人名章
%\hskip 0.75\textwidth
%\begin{minipage}[b]{0.20\textwidth}
%		\includegraphics[scale=0.07]{Figures/signature-seal_Jiang-1.png}
%\end{minipage}

%\section{正文章节}
%参考文献的引用方式1\upcite{QCQC_2014}

%-------------------The Figure Of The Paper------------------
%\begin{figure}[h!]
%\centering
%\includegraphics[height=3.35in,width=2.85in,viewport=0 0 400 475,clip]{PbTe_Band_SO.eps}
%\hspace{0.5in}
%\includegraphics[height=3.35in,width=2.85in,viewport=0 0 400 475,clip]{EuTe_Band_SO.eps}
%\caption{\small Band Structure of PbTe (a) and EuTe (b).}%(与文献\cite{EPJB33-47_2003}图1对比)
%\label{Pb:EuTe-Band_struct}
%\end{figure}

%-------------------The Equation Of The Paper-----------------
%\begin{equation}
%\varepsilon_1(\omega)=1+\frac2{\pi}\mathscr P\int_0^{+\infty}\frac{\omega'\varepsilon_2(\omega')}{\omega'^2-\omega^2}d\omega'
%\label{eq:magno-1}
%\end{equation}

%\begin{equation} 
%\begin{split}
%\varepsilon_2(\omega)&=\frac{e^2}{2\pi m^2\omega^2}\sum_{c,v}\int_{BZ}d{\vec k}\left|\vec e\cdot\vec M_{cv}(\vec k)\right|^2\delta [E_{cv}(\vec k)-\hbar\omega] \\
% &= \frac{e^2}{2\pi m^2\omega^2}\sum_{c,v}\int_{E_{cv}(\vec k=\hbar\omega)}\left|\vec e\cdot\vec M_{cv}(\vec k)\right|^2\dfrac{dS}{\nabla_{\vec k}E_{cv}(\vec k)}
% \end{split}
%\label{eq:magno-2}
%\end{equation}

%-------------------The Table Of The Paper----------------------
%\begin{table}[!h]
%\tabcolsep 0pt \vspace*{-12pt}
%%\caption{The representative $\vec k$ points contributing to $\sigma_2^{xy}$ of interband transition in EuTe around 2.5 eV.}
%\label{Table-EuTe_Sigma}
%\begin{minipage}{\textwidth}
%%\begin{center}
%\centering
%\def\temptablewidth{0.84\textwidth}
%\rule{\temptablewidth}{1pt}
%\begin{tabular*} {\temptablewidth}{|@{\extracolsep{\fill}}c|@{\extracolsep{\fill}}c|@{\extracolsep{\fill}}l|}

%-------------------------------------------------------------------------------------------------------------------------
%&Peak (eV)  & {$\vec k$}-point            &Band{$_v$} to Band{$_c$}  &Transition Orbital
%Components\footnote{波函数主要成分后的括号中,$5s$、$5p$和$5p$、$4f$、$5d$分别指碲和铕的原子轨道。} &Gap (eV)   \\ \hline
%-------------------------------------------------------------------------------------------------------------------------
%&2.35       &(0,0,0)         &33$\rightarrow$34    &$4f$(31.58)$5p$(38.69)$\rightarrow$$5p$      &2.142   \\% \cline{3-7}
%&       &(0,0,0)         &33$\rightarrow$34    &$4f$(31.58)$5p$(38.69)$\rightarrow$$5p$      &2.142   \\% \cline{3-7}
%-------------------------------------------------------------------------------------------------------------------------
%\end{tabular*}
%\rule{\temptablewidth}{1pt}
%\end{minipage}{\textwidth}
%\end{table}

%-------------------The Long Table Of The Paper--------------------
%\begin{small}
%%\begin{minipage}{\textwidth}
%%\begin{longtable}[l]{|c|c|cc|c|c|} %[c]指定长表格对齐方式
%\begin{longtable}[c]{|c|c|p{1.9cm}p{4.6cm}|c|c|}
%\caption{Assignment for the peaks of EuB$_6$}
%\label{tab:EuB6-1}\\ %\\长表格的caption中换行不可少
%\hline
%%
%--------------------------------------------------------------------------------------------------------------------------------
%\multicolumn{2}{|c|}{\bfseries$\sigma_1(\omega)$谱峰}&\multicolumn{4}{c|}{\bfseries部分重要能带间电子跃迁\footnotemark}\\ \hline
%\endfirsthead
%--------------------------------------------------------------------------------------------------------------------------------
%%
%\multicolumn{6}{r}{\it 续表}\\
%\hline
%--------------------------------------------------------------------------------------------------------------------------------
%标记 &峰位(eV) &\multicolumn{2}{c|}{有关电子跃迁} &gap(eV)  &\multicolumn{1}{c|}{经验指认} \\ \hline
%\endhead
%--------------------------------------------------------------------------------------------------------------------------------
%%
%\multicolumn{6}{r}{\it 续下页}\\
%\endfoot
%\hline
%--------------------------------------------------------------------------------------------------------------------------------
%%
%%\hlinewd{0.5$p$t}
%\endlastfoot
%--------------------------------------------------------------------------------------------------------------------------------
%%
%% Stuff from here to \endlastfoot goes at bottom of last page.
%%
%--------------------------------------------------------------------------------------------------------------------------------
%标记 &峰位(eV)\footnotetext{见正文说明。} &\multicolumn{2}{c|}{有关电子跃迁\footnotemark} &gap(eV) &\multicolumn{1}{c|}{经验指认\upcite{PRB46-12196_1992}}\\ \hline
%--------------------------------------------------------------------------------------------------------------------------------
%
%     &0.07 &\multicolumn{2}{c|}{电子群体激发$\uparrow$} &--- &电子群\\ \cline{2-5}
%\raisebox{2.3ex}[0pt]{$\omega_f$} &0.1 &\multicolumn{2}{c|}{电子群体激发$\downarrow$} &--- &体激发\\ \hline
%--------------------------------------------------------------------------------------------------------------------------------
%
%     &1.50 &\raisebox{-2ex}[0pt][0pt]{20-22(0,1,4)} &2$p$(10.4)4$f$(74.9)$\rightarrow$ &\raisebox{-2ex}[0pt][0pt]{1.47} &\\%\cline{3-5}
%     &1.50$^\ast$ & &2$p$(17.5)5$d_{\mathrm E}$(14.0)$\uparrow$ & &4$f$$\rightarrow$5$d_{\mathrm E}$\\ \cline{3-5}
%     \raisebox{2.3ex}[0pt][0pt]{$a$} &(1.0$^\dagger$) &\raisebox{-2ex}[0pt][0pt]{20-22(1,2,6)} &\raisebox{-2ex}[0pt][0pt]{4$f$(89.9)$\rightarrow$2$p$(18.7)5$d_{\mathrm E}$(13.9)$\uparrow$}\footnotetext{波函数主要成分后的括号中,2$s$、2$p$和5$p$、4$f$、5$d$、6$s$分别指硼和铕的原子轨道;5$d_{\mathrm E}$、5$d_{\mathrm T}$分别指铕的(5$d_{z^2}$,5$d_{x^2-y^2}$和5$d_{xy}$,5$d_{xz}$,5$d_{yz}$)轨道,5$d_{\mathrm{ET}}$(或5$d_{\mathrm{TE}}$)则指5个5$d$轨道成分都有,成分大的用脚标的第一个字母标示;2$ps$(或2$sp$)表示同时含有硼2$s$、2$p$轨道成分,成分大的用第一个字母标示。$\uparrow$和$\downarrow$分别标示$\alpha$和$\beta$自旋电子跃迁。} &\raisebox{-2ex}[0pt][0pt]{1.56} &激子跃迁。 \\%\cline{3-5}
%     &(1.3$^\dagger$) & & & &\\ \hline
%--------------------------------------------------------------------------------------------------------------------------------

%     & &\raisebox{-2ex}[0pt][0pt]{19-22(0,0,1)} &2$p$(37.6)5$d_{\mathrm T}$(4.5)4$f$(6.7)$\rightarrow$ & & \\\nopagebreak %\cline{3-5}
%     & & &2$p$(24.2)5$d_{\mathrm E}$(10.8)4$f$(5.1)$\uparrow$ &\raisebox{2ex}[0pt][0pt]{2.78} &a、b、c峰可能 \\ \cline{3-5}
%     & &\raisebox{-2ex}[0pt][0pt]{20-29(0,1,1)} &2$p$(35.7)5$d_{\mathrm T}$(4.8)4$f$(10.0)$\rightarrow$ & &包含有复杂的\\ \nopagebreak%\cline{3-5}
%     &2.90 & &2$p$(23.2)5$d_{\mathrm E}$(13.2)4$f$(3.8)$\uparrow$ &\raisebox{2ex}[0pt][0pt]{2.92} &强激子峰。$^{\ast\ast}$\\ \cline{3-5}
%$b$  &2.90$^\ast$ &\raisebox{-2ex}[0pt][0pt]{19-22(0,1,1)} &2$p$(33.9)4$f$(15.5)$\rightarrow$ & &B2$s$-2$p$的价带 \\ \nopagebreak%\cline{3-5}
%     &3.0 & &2$p$(23.2)5$d_{\mathrm E}$(13.2)4$f$(4.8)$\uparrow$ &\raisebox{2ex}[0pt][0pt]{2.94} &顶$\rightarrow$B2$s$-2$p$导\\ \cline{3-5}
%     & &12-15(0,1,2) &2$p$(39.3)$\rightarrow$2$p$(25.2)5$d_{\mathrm E}$(8.6)$\downarrow$ &2.83 &带底跃迁。\\ \cline{3-5}
%     & &14-15(1,1,1) &2$p$(42.5)$\rightarrow$2$p$(29.1)5$d_{\mathrm E}$(7.0)$\downarrow$ &2.96 & \\\cline{3-5}
%     & &13-15(0,1,1) &2$p$(40.4)$\rightarrow$2$p$(28.9)5$d_{\mathrm E}$(6.6)$\downarrow$ &2.98 & \\ \hline
%--------------------------------------------------------------------------------------------------------------------------------
%%\hline
%%\hlinewd{0.5$p$t}
%\end{longtable}
%%\end{minipage}{\textwidth}
%%\setlength{\unitlength}{1cm}
%%\begin{picture}(0.5,2.0)
%%  \put(-0.02,1.93){$^{1)}$}
%%  \put(-0.02,1.43){$^{2)}$}
%%\put(0.25,1.0){\parbox[h]{14.2cm}{\small{\\}}
%%\put(-0.25,2.3){\line(1,0){15}}
%%\end{picture}
%\end{small}

%-----------------------------------------------------------------------------------------------------------------------------------------------------------------------------------------------------%

%-------------------------------------------------------------------------Thanks------------------------------------------------------------------------------------------------
%\section{致谢}
%\newpage %%
%\thispagestyle{fancy}   % 首页插入页眉页脚 

%--------------------------------------------------------------------------The Biblography of The Paper-----------------------------------------------------------------%
%\newpage																				%
%-----------------------------------------------------------------------------------------------------------------------------------------------------------------------%
%\begin{thebibliography}{99}																		%
%%\bibitem{PRL58-65_1987}H.Feil, C. Haas, {\it Phys. Rev. Lett.} {\bf 58}, 65 (1987).											%
%	\bibitem{kp-method} \textrm{Zhenxi Pan, Yong Pan, Jun Jiang$^{\ast}$, Liutao Zhao}, \textrm{High-Throughput Electronic Band Structure Calculations for Hexaborides}, \textit{Intelligent Computing}, \textbf{Springer}, \textbf{P.386-395}, (2019).%
%	\bibitem{PAW-dataset} \textrm{姜骏},\textrm{PAW原子数据集的构造与检验}, \textit{中国化学会第十二届全国量子化学会议论文摘要集},\textbf{太原},(2014).
%\end{thebibliography}																			%
%-----------------------------------------------------------------------------------------------------------------------------------------------------------------------%
%\phantomsection\addcontentsline{toc}{section}{Bibliography} %直接调用\addcontentsline命令可能导致超链指向不准确,一般需要在之前调用一次\phantomsection命令加以修正%
%\bibliography{ref/Myref_HT}   %
%\bibliographystyle{ref/mybib} %% 接近ieeert样式
%%%%%%%%%%%%%%%%%%%%%%%%%%%%      \bibliographystyle         %%%%%%%%%%%%%%%%%%%%%%%%%%%%%%%%%%
%%%%%%      LaTeX 参考文献标准选项及其样式共有以下8种:                                %%%%%%%%
% plain,按字母的顺序排列,比较次序为作者、年度和标题.
% unsrt,样式同plain,只是按照引用的先后排序.
% alpha,用作者名首字母+年份后两位作标号,以字母顺序排序.
% abbrv,类似plain,将月份全拼改为缩写,更显紧凑.
% ieeetr,国际电气电子工程师协会期刊样式.
% acm,美国计算机学会期刊样式.
% siam,美国工业和应用数学学会期刊样式.
% apalike,美国心理学学会期刊样式.
%%%%%%%%%%%%%%%%%%%%%%%%%%%%%%%%%%%%%%%%%%%%%%%%%%%%%%%%%%%%%%%%%%%%%%%%%%%%%%%%%%%%%%%%%%%%%%%
%%  \nocite{*}																				%
%-----------------------------------------------------------------------------------------------------------------------------------------------------------------------%

\clearpage     %\end{CJK} 前加上\clearpage是CJK的要求
%\end{CJK*}
\end{document}
