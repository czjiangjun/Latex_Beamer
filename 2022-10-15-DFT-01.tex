%%%%%%%%%%%%%%%%%%%%%%%%%%%%%%%%%%%%%%%%%%  不使用 authblk 包制作标题  %%%%%%%%%%%%%%%%%%%%%%%%%%%%%%%%%%%%%%%%%%%%%%
%-------------------------------PPT Title-------------------------------------
\title{01-量子力学基础}
%-----------------------------------------------------------------------------

%----------------------------Author & Date------------------------------------
%\author[\textrm{Jun\_Jiang}]{姜\;\;骏\inst{}} %[]{} (optional, use only with lots of authors)
%% - Give the names in the same order as the appear in the paper.
%% - Use the \inst{?} command only if the authors have different
%%   affiliation.
%\institute[BCC]{\inst{}%
\institute[Gain~Strong]{\inst{}%
% \vskip -20pt 北京市计算中心}
\vskip -20pt {\large 格致斯创~科技}}
\date[\today] % (optional, should be abbreviation of conference name)
{%	{\fontsize{6.2pt}{4.2pt}\selectfont{\textcolor{blue}{E-mail:~}\url{jiangjun@bcc.ac.cn}}}
\vskip 45 pt {\fontsize{8.2pt}{6.2pt}\selectfont{%清华大学\;\;物理系% 报告地点
	\vskip 5 pt \textrm{2022.10.15}}}
}

%% - Either use conference name or its abbreviation
%% - Not really information to the audience, more for people (including
%%   yourself) who are reading the slides onlin%%   yourself) who are reading the slides onlin%%   yourself) who are reading the slides onlineee
%%%%%%%%%%%%%%%%%%%%%%%%%%%%%%%%%%%%%%%%%%%%%%%%%%%%%%%%%%%%%%%%%%%%%%%%%%%%%%%%%%%%%%%%%%%%%%%%%%%%%%%%%%%%%%%%%%%%%

\subject{}
% This is only inserted into the PDF information catalog. Can be left
% out.
%\maketitle
\frame
{
%	\frametitle{\fontsize{9.5pt}{5.2pt}\selectfont{\textcolor{orange}{“高通量并发式材料计算算法与软件”年度检查}}}
\titlepage
}
%-----------------------------------------------------------------------------

%------------------------------------------------------------------------------列出全文 outline ---------------------------------------------------------------------------------
%\section*{}
%\frame[allowframebreaks]
%{
%  \frametitle{Outline}
%%  \frametitle{\textcolor{mycolor}{\secname}}
%  \tableofcontents%[current,currentsection,currentsubsection]
%}
%%在每个section之前列出全部Outline
%%类似的在每个subsection之前列出全部Outline是\AtBeginSubsection[]
%\AtBeginSection[]
%{
%  \frame<handout:0>%[allowframebreaks]
%  {
%    \frametitle{Outline}
%%全部Outline中,本部分加亮
%    \tableofcontents[current,currentsection]
%  }
%}

%-----------------------------------------------PPT main Body------------------------------------------------------------------------------------
\small
\section{引言}
\frame
{
	\frametitle{材料模拟的作用与地位}
\begin{figure}[h!]
\vspace*{-0.18in}
\centering
\includegraphics[height=2.55in,width=4.05in]{Figures/MGE.png}
%\caption{\tiny \textrm{Pseudopotential for metallic sodium, based on the empty core model and screened by the Thomas-Fermi dielectric function.}}%(与文献\cite{EPJB33-47_2003}图1对比)
%\caption{\tiny \textrm{Pseudopotential for metallic sodium, based on the empty core model and screened by the Thomas-Fermi dielectric function.}}%(与文献\cite{EPJB33-47_2003}图1对比)
\label{MGE}
\end{figure}
}

\frame
{
	\frametitle{材料模拟的基本思想和方法}
\begin{figure}[h!]
\vspace*{-0.25in}
\centering
\includegraphics[height=0.80in,width=4.05in]{Figures/MGE-2.png}
%\caption{\tiny \textrm{Pseudopotential for metallic sodium, based on the empty core model and screened by the Thomas-Fermi dielectric function.}}%(与文献\cite{EPJB33-47_2003}图1对比)
\vskip 0.05pt
\includegraphics[height=2.20in,width=3.45in]{Figures/Multi-Scale-6.png}
%\caption{\tiny \textrm{Pseudopotential for metallic sodium, based on the empty core model and screened by the Thomas-Fermi dielectric function.}}%(与文献\cite{EPJB33-47_2003}图1对比)
\label{Multi-Scale}
\end{figure}
}

\frame
{
	\frametitle{\rm{I~Have~A~Dream}}
\begin{figure}[h!]
\vspace*{-0.18in}
\centering
\includegraphics[height=2.55in,width=4.05in]{Figures/Schematic_Material-Design.png}
%\caption{\tiny \textrm{Pseudopotential for metallic sodium, based on the empty core model and screened by the Thomas-Fermi dielectric function.}}%(与文献\cite{EPJB33-47_2003}图1对比)
%\caption{\tiny \textrm{Pseudopotential for metallic sodium, based on the empty core model and screened by the Thomas-Fermi dielectric function.}}%(与文献\cite{EPJB33-47_2003}图1对比)
\label{Schematic_Material-Design}
\end{figure}
}

\section{量子力学基础}
\subsection{能量量子化}
\frame
{
	\frametitle{经典物理学的成功}
\begin{figure}[h!]
\vspace*{-0.18in}
\centering
\includegraphics[height=1.90in,width=3.00in,viewport=0 0 1150 690,clip]{Figures/Albert_Michelson-Quotes.jpg}
\includegraphics[height=1.10in,width=2.55in,viewport=0 0 880 400,clip]{Figures/Quote-there-is-nothing-new-to-be-discovered-in-physics-now-all-that-remains-is-more-and-more-lord-kelvin-57-38-79.jpg}
%\includegraphics[height=2.50in,width=4.05in,viewport=0 20 735 470,clip]{Figures/Two-dark-cloud-in-physics-3.jpg}
%\includegraphics[height=2.40in,width=4.05in,viewport=0 50 735 470,clip]{Figures/Two-dark-cloud-in-physics-2.jpg}
%\includegraphics[height=2.40in,width=4.05in,viewport=0 0 580 325,clip]{Figures/Two-dark-cloud-in-physics-1.jpg}
\label{two_Dark_Clouds}
\end{figure}
}

\frame
{
	\frametitle{经典物理学天空的“两朵乌云”\textrm{(Dark Clouds)}}
\begin{figure}[h!]
\vspace*{-0.18in}
\centering
%\includegraphics[height=2.90in,width=2.80in,viewport=0 0 1000 1100,clip]{Figures/Baron_Kelvin-Lecture.jpeg}
\includegraphics[height=0.35in,width=3.35in,viewport=0 900 1020 1030,clip]{Figures/Baron_Kelvin-Lecture.jpeg}
\includegraphics[height=0.80in,width=3.35in,viewport=0 50 1020 350,clip]{Figures/Baron_Kelvin-Lecture.jpeg}
\includegraphics[height=1.85in,width=4.05in,viewport=0 50 735 370,clip]{Figures/Two-dark-cloud-in-physics-2.jpg}
\label{two_Dark_Clouds_2}
\end{figure}
}

%\frame
%{
%	\frametitle{经典物理学天空的“两朵乌云”\textrm{(Dark Clouds)}}
%\begin{figure}[h!]
%\vspace*{-0.18in}
%\centering
%\includegraphics[height=2.50in,width=4.05in,viewport=0 20 735 470,clip]{Figures/Two-dark-cloud-in-physics-3.jpg}
%\includegraphics[height=2.40in,width=4.05in,viewport=0 50 735 470,clip]{Figures/Two-dark-cloud-in-physics-2.jpg}
%\includegraphics[height=2.40in,width=4.05in,viewport=0 0 580 325,clip]{Figures/Two-dark-cloud-in-physics-1.jpg}
%\label{two_Dark_Clouds_3}
%\end{figure}
%}
%
%\begin{frame}[allowframebreaks]
%	\frametitle{黑体辐射与能量量子化}
%	黑体辐射\textrm{(black-body radiation)}\footnote{\fontsize{6.2pt}{4.2pt}\selectfont{物质达到平衡,可以用温度来描述物质的状态;~而光和物质的交互作用很强,因此光和光之间也可以用一个温度来描述(光和光之间本身不会有交互作用,但光和物质相互作用很强)}}是光和物质达到平衡时表现出的热辐射(本质是电磁辐射)
%\begin{figure}[h!]
%\centering
%\vspace{-5.5pt}
%\includegraphics[height=1.22in,width=1.22in,viewport=0 0 136 136,clip]{Figures/Black_box.jpg}
%%\hskip 1pt
%%\includegraphics[height=1.32in,width=2.25in,viewport=0 0 390 215,clip]{Figures/Black_box_curve.png}
%\caption{\tiny \textrm{The black-body radiation and the curve}}
%\label{Black_box}
%\end{figure}
%	{\fontsize{8.0pt}{4.2pt}\selectfont{
%\textrm{1896}年,\textrm{Wien}从热力学角度提出,将组成黑体空腔腔壁的粒子看作带电的线性谐振子,假设谐振子能谱分布与\textrm{Maxwell}分子速率分布类似,得到黑体辐射强度的近似表示:
%\begin{displaymath}
%	\rho(\lambda, T)=c\cdot\dfrac1{\lambda^5}\cdot\mathrm{e}^{-c_1/\lambda T}
%\end{displaymath}
%%\textrm{Wien}近似
%%\begin{itemize}
%%	\item 在温度较低或波长较短时,与实验一致性非常好
%%	\item 在$\lambda T$较大时,会与实验存在偏差,并随$\lambda T$增大而增加
%%\end{itemize}
%%		随后\textrm{Rayeigh}将\textrm{Boltzmann}统计物理的能量(按自由度)均分定理应用到电磁辐射上,则当$\lambda T$较大时,\textrm{Wien}近似中指数$\mathrm{e}^{-c_1/\lambda T}\rightarrow1$,此时辐射强度将与温度成正比

%\textrm{Rayleigh-Jeans}公式\footnote{\fontsize{6.2pt}{4.2pt}\selectfont{\textrm{Rayleigh}提出,经\textrm{Jeans}作小的修正(即式中的系数8)}}给出的黑体辐射
%	\begin{displaymath}
%		\rho(\nu, T)\mathrm{d}\nu=\dfrac{8\pi k_{\mathrm{B}}T\nu^2}{c^2}\mathrm{d}\nu
%	\end{displaymath}
%	\textrm{Rayleigh-Jeans}公式是发散的
%	\begin{displaymath}
%		\int_0^{\infty}\rho(\nu, T)\mathrm{d}\nu=\int_0^{\infty}\dfrac{8\pi k_{\mathrm{B}}T\nu^2}{c^2}\mathrm{d}\nu\longrightarrow\infty
%	\end{displaymath}}}
%\end{frame}

\frame
{
	\frametitle{黑体辐射与能量量子化}
	{\fontsize{9.0pt}{4.2pt}\selectfont{\textrm{1900}年,为了解释黑体辐射\textrm{(black-body radiation)}\footnote{\fontsize{6.2pt}{4.2pt}\selectfont{黑体辐射是光和物质达到平衡时表现出的现象。物质达到平衡,可以用温度来描述物质的状态;~而光和物质的交互作用很强,因此光和光之间也可以用一个温度来描述(光和光之间本身不会有交互作用,但光和物质相互作用很强)}}的能量密度与电磁辐射频率的关系,\textrm{M.~Planck}%\textcolor{blue}{放弃能量均分定理}\textrm{(the equipartition theorem)},
	引入\textcolor{red}{能量量子化}\textrm{(quantization of energy)}的假设,利用统计物理推导出与实验符合得非常好的黑体辐射\textrm{Planck~}公式:~
	\begin{displaymath}
		\rho_{\nu}\mathrm{d}{\nu}=\dfrac{8{\pi}h{\nu}^3}{c^2}\bigg(\textcolor{blue}{\dfrac1{\mathrm{e}^{h\nu/kT}-1}}\bigg)\mathrm{d}\nu
	\end{displaymath}}}
\begin{figure}[h!]
\centering
\vspace{-10.5pt}
\includegraphics[height=1.32in,width=1.32in,viewport=0 0 136 136,clip]{Figures/Black_box.jpg}
\hskip 1pt
\includegraphics[height=1.32in,width=2.25in,viewport=0 0 390 215,clip]{Figures/Black_box_curve.png}
\caption{\tiny \textrm{The black-body radiation and the curve}}
\label{Black_box}
\end{figure}
}

%\begin{frame}[allowframebreaks]
%	\frametitle{黑体辐射与能量量子化}
%\textrm{Planck}认为,辐射的黑体是由大量带电的谐振子组成的,它们可以发射和吸收辐射能,根据{Boltzmann}统计物理和\textrm{Maxwell}电动力学,提出:
%
%	{\fontsize{9.0pt}{4.2pt}\selectfont{
%在频率(或波长)一定,振子的平均能量$\bar{\varepsilon}(\nu)$与单个振子体系的基元能量$\varepsilon_0(\nu)$的关系为\footnote{\fontsize{6.2pt}{4.2pt}\selectfont{\textcolor{red}{这是\textrm{Planck}放弃了能量均分原理,凭物理直觉猜出的辐射体系的能量密度的量子化分布}%,但是要推导出量子化分布的结果,会碰到\textrm{Laplace}逆变换和广义函数问题。\textrm{Laplace}逆变换具有不适定性(病态)、解的唯一性和稳定性等都遇到难题,任何近似的数值试探很容易陷入泥坑不能自拔;~广义函数差不多\textrm{40}年后才正式进入数学界的厅堂,并很快获得第二次世界大战后第一个\textrm{Fields}奖}}
%\begin{displaymath}
%	\bar{\varepsilon}(\nu)=\dfrac{\varepsilon_0(\nu)}{\mathrm{e}^{\varepsilon_0(nu)/kT}-1}
%\end{displaymath}
%对于不同频率的振子,发现$\varepsilon_0(\nu)$与$\nu$存在正比关系
%\begin{displaymath}
%	\varepsilon_0(\nu)\propto\nu
%\end{displaymath}
%因此定义了比例常数$h$。因此对于频率一定的振子体系而言,体系最低基元能量为
%\begin{displaymath}
%	\varepsilon_{\nu}=h\nu
%\end{displaymath}
%这就是能量量子化的概念}}
%
%$\ast$\textrm{Planck}对比了\textrm{Rayleigh}考虑黑体辐射的过程,将公式变形
%\begin{displaymath}
%	\rho_{\nu}\mathrm{d\nu}=\dfrac{c_2\nuk_{\mathrm{B}}}{\mathrm{e}^{c_2\nu k_{\mathrm{B}}/Tk_{\mathrm{B}}}-1}\dfrac{c_1}{c_2k_{\mathrm{B}}}\nu^2\mathrm{d}\nu
%\end{displaymath}
%注意到\textrm{Rayleigh}公式推导时,空腔中电磁波允许的振动模式的态密度函数数目为$N$\sim$\nu^2\mathrm{d}\nu$,因此
%\begin{displaymath}
%	\bar{\varepsilon}_{\nu}\equiv\dfrac{c_2\nu k_{\mathrm{B}}}{\mathrm{e}^{c_2\nu k_{\mathrm{B}}/Tk_{\mathrm{B}}}-1}
%\end{displaymath}
%被解释为频率$\nu$的允许的电磁波振动模的平均能量,并有
%\begin{displaymath}
%	\bar{\varepsilon}_{\nu}\equiv\dfrac{c_2\nu k_{\mathrm{B}}}{\mathrm{e}^{c_2\nu k_{\mathrm{B}}/Tk_{\mathrm{B}}}-1}=\dfrac{\mathrm{d}}{\mathrm{d}\beta}\ln\big(\textcolor{purple}{1-\mathrm{e}^{-\beta k_{\mathrm{B}}c_2\nu}}\big)
%\end{displaymath}
%这里$\beta=1/k_{\mathrm{B}}T$,考虑配分函数\textrm{(partition function)}:
%\begin{displaymath}
%	Z=\sum_{\varepsilon}\mathrm{e}^{-\beta\varepsilon}
%\end{displaymath}
%满足
%\begin{displaymath}
%	\bar{E}=-\dfrac{\mathrm{d}}{\mathrm{d}\beta}\ln Z=\dfrac{\mathrm{d}}{\mathrm{d}\beta}\ln\big(\textcolor{purple}{\sum_{\varepsilon}\mathrm{e}^{-\beta\varepsilon}}\big)=\dfrac{\sum_{\varepsilon}\varepsilon\mathrm{e}^{-\beta\varepsilon}}{\sum_{\varepsilon}\mathrm{e}^{-\beta\varepsilon}}
%\end{displaymath}
%因此令
%\begin{displaymath}
%	\textcolor{purple}{\sum_{\varepsilon}\mathrm{e}^{-\beta\varepsilon}=1-\mathrm{e}^{-\beta k_{\mathrm{B}}c_2\nu}}
%\end{displaymath}
%将等式左侧用等比级数展开,有
%\begin{displaymath}
%	\sum_{\varepsilon}\mathrm{e}^{-\beta\varepsilon}=1+\mathrm{e}^{-\beta k_{\mathrm{B}}c_2\nu}++\mathrm{e}^{-2\beta k_{\mathrm{B}}c_2\nu}++\mathrm{e}^{-3\beta k_{\mathrm{B}}c_2\nu}+$\cdots$
%\end{displaymath}
%由此可得递推关系
%\begin{displaymath}
%	\begin{aligned}
%		\varepsilon_0=&0\\
%		\varepsilon_1=&k_{\mathrm{B}}c_2\nu\\
%		\varepsilon_2=&2k_{\mathrm{B}}c_2\nu\\
%		&\cdots\\
%		\varepsilon_n=&nk_{\mathrm{B}}c_2\nu
%	\end{aligned}
%\end{displaymath}
%令$h=k_{\mathrm{B}}c_2$,因此有能量关系
%\begin{displaymath}
%	\varepsilon_n=nh\nu
%\end{displaymath}
%即\textrm{Planck}公式中,黑体辐射的空腔中,电磁波的能量是不连续的
%`
%	\textrm{Rayleigh-Jeans}公式是\textrm{Planck}黑体辐射公式低频下的近似结果
%
%	{\fontsize{9.0pt}{4.2pt}\selectfont{当$h\nu/k_{\mathrm{B}}T\ll1$时,可有
%	\begin{displaymath}
%		\begin{aligned}
%			\rho(\nu, T)\mathrm{d}\nu=&\dfrac{8\pi h}{c^3}\dfrac{\nu^3}{\mathrm{e}^{h\nu/k_{\mathrm{B}}T}-1}\mathrm{d}\nu\approx\dfrac{8\pi h}{c^3}\dfrac{\nu^3}{1+h\nu/k_{\mathrm{B}}T-1}\mathrm{d}\nu\\
%		=&\textcolor{blue}{\dfrac{8\pi h}{c^3}\dfrac{\nu^3}{h\nu/k_{\mathrm{B}}T}\mathrm{d}\nu=\dfrac{8\pi k_{\mathrm{B}}T\nu^2}{c^3}\mathrm{d}\nu}
%		\end{aligned}
%	\end{displaymath} }}
%
%	\textrm{Lorentz}对\textrm{Planck}黑体辐射公式的物理图像作了阐释:\\
%	\textcolor{magenta}{假设每个频率的能量是不连续的,而是分成正比与频率的小份,根据\textrm{Boltzmann}分布},有
%	\begin{displaymath}
%		\begin{aligned}
%			\rho(\nu, T)\mathrm{d}\nu=\textcolor{red}{\dfrac{\bar{E}N}{L^3}}\mathrm{d}\nu=&\dfrac{\sum\limits_0^{\infty}\mathrm{e}^{-nh\nu/k_{\mathrm{B}}T}nh\nu}{\sum\limits_0^{\infty}\mathrm{e}^{-nh\nu/k_{\mathrm{B}}T}}(2\times4\pi|\vec n|^2\mathrm{d}|\vec n|)\mathrm{d}\nu\\
%		=&\dfrac{8\pi h}{c^3}\dfrac{\nu^3}{\mathrm{e}^{h\nu/k_{\mathrm{B}}T}-1}\mathrm{d}\nu
%		\end{aligned}
%	\end{displaymath}
%\end{frame}
%
%\begin{frame}
%	\frametitle{黑体辐射与能量量子化}
%	\textrm{Wien}近似是\textrm{Planck}黑体辐射公式高频下的近似结果
%
%	{\fontsize{9.0pt}{4.2pt}\selectfont{当$h\nu/k_{\mathrm{B}}T\gg1$时,可有
%	\begin{displaymath}
%		\begin{aligned}
%			\rho(\nu, T)\mathrm{d}\nu=&\dfrac{8\pi h}{c^3}\dfrac{\nu^3}{\mathrm{e}^{h\nu/k_{\mathrm{B}}T}-1}\mathrm{d}\nu\approx\dfrac{8\pi h}{c^3}\dfrac{\nu^3}{\mathrm{e}^{h\nu/k_{\mathrm{B}}T}}\mathrm{d}\nu\\
%			=&\textcolor{blue}{\dfrac{8\pi h\cdot\nu^3}{c^3}\mathrm{e}^{-h\nu/k_{\mathrm{B}}T}\mathrm{d}\nu}
%		\end{aligned}
%	\end{displaymath}
%	最初\textrm{Wien}提出近似时并不包含\textrm{Planck}常数,他选择黑体辐射的波长和原子的\textrm{Maxwell-Boltzmann}分布%\textcolor{blue}{满足能量均分定理}
%	来表示该常数 }} 
%
%	\textrm{Wien}近似曲线的极值的确定(曲线函数的导数为零)
%	\begin{displaymath}
%		\begin{aligned}
%			\lambda_{\max}=&\dfrac{hc}{5k_{\mathrm{B}}T}\approx\dfrac{0.2878~\mathrm{cm}\cdot \textrm{K}}{T}\\
%			\nu_{\max}=&\dfrac{3k_{\mathrm{B}}T}{h}\approx6.25\times10^{10}\dfrac{\mathrm{Hz}}{\mathrm{K}}\cdot{T}
%		\end{aligned}
%	\end{displaymath}
%	这就是\textrm{Wien}位移定律
%\end{frame}
%
\frame
{
	\frametitle{波-粒二象性与光电效应}
\begin{figure}[h!]
\centering
\vspace{-15.5pt}
\includegraphics[height=1.35in,width=2.70in,viewport=0 0 536 280,clip]{Figures/wave-particle_duality.png}
\vskip 1pt
\includegraphics[height=1.32in,width=2.05in,viewport=0 0 620 455,clip]{Figures/Photoelectic_effect.png}
\caption{\textrm{The wave-particle duality and Photoelectric effect}}
\label{wave_and_particle}
\end{figure}
}

\frame
{
	\frametitle{电子衍射、\textrm{Compton~effect}与\textrm{H}原子光谱}
\begin{figure}[h!]
\centering
\vspace{-15.5pt}
\includegraphics[height=1.35in,width=1.80in,viewport=0 0 680 600,clip]{Figures/Single_Slit_Diffraction.png}
\includegraphics[height=1.20in,width=2.10in,viewport=0 0 1600 950,clip]{Figures/Compton_effect.jpg}\\
\includegraphics[height=1.65in,width=1.75in,viewport=0 0 620 600,clip]{Figures/Hydrogen_spectrum-3.png}
\includegraphics[height=1.55in,width=1.75in,viewport=0 0 500 380,clip]{Figures/Hydrogen_spectrum-2.png}
%\caption{\textrm{The wave-particle duality and Photoelectric effect}}
\label{electron:wave_and_particle}
\end{figure}
}

\subsection{\textrm{Schr\"odinger}方程与量子力学的建立}
\frame
{
	\frametitle{\textrm{De Broglie}物质波}
\begin{minipage}{0.53\textwidth}
\begin{figure}[h!]
\centering
\vspace{-15.5pt}
\includegraphics[height=1.3in,width=2.1in,viewport=0 0 500 280,clip]{Figures/De-Broglie-waves.jpg}
%\caption{\textrm{The wave-particle duality and Photoelectric effect}}
\label{Matter_wave}
\end{figure}
经典的观念:
\begin{itemize}
	\item \textcolor{red}{粒子}:~\textcolor{blue}{物质存在的形式}
	\item \textcolor{red}{波动}:~\textcolor{blue}{能量传递的形式}
\end{itemize}
\end{minipage}
\begin{minipage}{0.45\textwidth}
\begin{figure}[h!]
\centering
\vspace{-15.5pt}
\includegraphics[height=2.80in,width=1.90in,viewport=0 0 430 650,clip]{Figures/De_Broglie-dissertation_Cover.jpg}
%\caption{\textrm{The wave-particle duality and Photoelectric effect}}
\label{De_Broglie-dissertation}
\end{figure}
\end{minipage}
}

\frame
{
	\frametitle{\textrm{驻波}}
\begin{figure}[h!]
\centering
\vspace{-15.5pt}
\includegraphics[height=0.40\textwidth,width=0.8\textwidth,viewport=0 0 900 450,clip]{Figures/Guitar-string.png}
\vskip 0.1pt
\includegraphics[height=0.35\textwidth,width=0.8\textwidth,viewport=0 0 122 48,clip]{Figures/string-standing-wave.jpg}
%\caption{\textrm{ABINIT}的Si.in}
\label{Standing_Wave_0}
\end{figure}
}

\frame
{
	\frametitle{驻波方程与势阱}
\begin{figure}[h!]
\centering
\vspace{-12.5pt}
\includegraphics[height=0.32\textwidth,width=0.7\textwidth,viewport=0 0 125 75,clip]{Figures/Standing_wave.jpeg}
\vskip 2pt
\includegraphics[height=0.40\textwidth,width=0.9\textwidth,viewport=0 0 1200 550,clip]{Figures/Standing_wave-energy.jpg}
%\caption{\textrm{ABINIT}的Si.in}
\label{Standing_Wave_1}
\end{figure}
}

\frame
{
	\frametitle{驻波方程与势阱}
\begin{figure}[h!]
\centering
\vspace{-5.5pt}
\includegraphics[height=0.55\textwidth,width=1.0\textwidth,viewport=0 0 720 400,clip]{Figures/Standing_wave-energy_1-5.jpg}
%\caption{\textrm{ABINIT}的Si.in}
\label{Standing_Wave_2}
\end{figure}
}

\frame
{
	\frametitle{驻波方程与势阱}
\begin{figure}[h!]
\centering
\vspace{-0.5pt}
\includegraphics[height=0.46\textwidth,width=1.0\textwidth,viewport=0 0 650 390,clip]{Figures/Standing_wave_Energy.jpeg}
%\caption{\textrm{ABINIT}的Si.in}
\label{Standing_Wave_3}
\end{figure}
}

\frame
{
	\frametitle{原子中电子的驻波方程}
\begin{figure}[h!]
	\vspace{-10.5pt}
\centering
\includegraphics[height=0.38\textwidth,width=0.74\textwidth,viewport=0 0 840 440,clip]{Figures/Standing_wave-atom.png}
\vskip 2pt
\animategraphics[autoplay, loop, height=1.3in]{1}{Figures/Standing_wave_circle_}{1}{25}
\label{Atomic-electron_Standing_wave}
\end{figure}
}

\frame
{
	\frametitle{原子中的电子轨道和能量}
\begin{minipage}{0.43\textwidth}
\begin{figure}[h!]
%	\vspace{-14.8pt}
	\vspace{-4.8pt}
\centering
\includegraphics[height=0.57\textwidth,width=1.00\textwidth,viewport=0 50 1680 1000,clip]{Figures/electron-theory-Bohr-point-mass-energy-levels.jpg}
%\includegraphics[height=1.23\textwidth,width=1.00\textwidth,viewport=0 10 1250 1500,clip]{Figures/wave_function.png}
\includegraphics[height=0.95\textwidth,width=1.00\textwidth,viewport=0 0 630 650,clip]{Figures/wave_function-2.jpeg}
\label{Atomic-electron_wave}
\end{figure}
\end{minipage}
\begin{minipage}{0.55\textwidth}
\begin{figure}[h!]
	\vspace{-16.5pt}
\centering
\includegraphics[height=1.10\textwidth,width=1.00\textwidth,viewport=0 0 560 600,clip]{Figures/Electron_orbital-energy.png}
\label{Atomic-electron_wave-energy}
\end{figure}
\end{minipage}
}

%\frame
%{
%	\frametitle{什么是“方程”}
%\begin{minipage}{0.45\textwidth}
%\begin{figure}[h!]
%	\vspace{-3.8pt}
%\centering
%\includegraphics[height=0.60in,width=0.60in,viewport=0 0 1130 1130,clip]{Figures/Chinese_charater-equation-1.png}
%\includegraphics[height=0.60in,width=0.60in,viewport=0 0 1130 1130,clip]{Figures/Chinese_charater-equation-2.png}
%\label{equation_1}
%\end{figure}
%\begin{figure}[h!]
%	\vspace{-15.8pt}
%\centering
%\includegraphics[height=2.20in,width=1.60in,viewport=0 0 520 700,clip]{Figures/Jiuzhang-8-equation_v1-1.png}
%\label{equation_2}
%\end{figure}
%\end{minipage}
%\begin{minipage}{0.53\textwidth}
%	{\fontsize{6.5pt}{5.2pt}\selectfont{
%	\vspace{-22.8pt}
%	\begin{itemize}
%		\item 方的本义是并,将两条船并起来,船头拴在一起,谓之方\\
%			\textcolor{purple}{\fontsize{5.5pt}{4.2pt}\selectfont{《说文》:~方,并船也。象两舟总头形。}}
%		\item 程的本义为称量谷物,并用作度量衡的总名\\
%			\textcolor{purple}{\fontsize{5.5pt}{4.2pt}\selectfont{《说文》:~程,品也。十髮爲程,十程爲分,十分爲寸。从禾呈聲。}}
%	\end{itemize}}}
%	\begin{itemize}
%		\item “\textcolor{blue}{课程}”{\fontsize{8.5pt}{6.2pt}\selectfont{指按不同物品的数量关系列出的计算式}}
%		\item “\textcolor{blue}{实}”{\fontsize{8.5pt}{6.2pt}\selectfont{指计算式中的常数项}}
%		\item “\textcolor{blue}{令每行为率}”,{\fontsize{8.5pt}{6.2pt}\selectfont{是由一个条件列一行计算式:~横列代表一个未知量}}
%		\item “\textcolor{blue}{如物数程之}”,{\fontsize{8.5pt}{6.2pt}\selectfont{就是有几个未知数就必须列出几个等式}}
%	\end{itemize}
%	\textcolor{red}{故而列出的一系列计算式称“方程”}
%\end{minipage}
%}
%
%\frame
%{
%	\frametitle{关于科学}
%\begin{figure}[h!]
%	\vspace{-3.8pt}
%\centering
%\includegraphics[height=0.60in,width=0.60in,viewport=0 0 1130 1130,clip]{Figures/Chinese_charater-Science.png}
%\includegraphics[height=0.60in,width=0.60in,viewport=0 0 1130 1130,clip]{Figures/Chinese_charater-equation-2.png}
%\label{Science_1}
%\end{figure}
%	\begin{itemize}
%		\item 科的本义是品类,等级\\
%			\textcolor{purple}{\fontsize{5.5pt}{4.2pt}\selectfont{《说文》:~科,程也。从禾从斗。斗者,量也。}}
%		\item 程的本义为称量谷物,并用作度量衡的总名\\
%			\textcolor{purple}{\fontsize{5.5pt}{4.2pt}\selectfont{《说文》:~程,品也。十髮爲程,十程爲分,十分爲寸。从禾呈聲。}}
%	\end{itemize}
%	\begin{itemize}
%		\item “科学”一词由近代日本学界率先使用,初用于对译英文中的\textcolor{red}{\textrm{Science}}及其它欧洲语言中的相应词汇,欧洲语言中该词来源于拉丁文\textcolor{red}{\textrm{Scientia}},\textcolor{blue}{意为“知识”与“学问”},在近代侧重关于自然的学问。在日本幕府末期到明治时期,“科学”是专门的“个别学问”,有的在以“分科的学问”的意义被使用
%	\end{itemize}
%	“\textcolor{blue}{科学}”一词经日本传入中国,命名其取\textcolor{purple}{分类、测量之学问}之义
%}
%
\frame
{
	\frametitle{\textrm{Schr\"odinger}~方程}
\begin{minipage}{0.49\textwidth}
\begin{figure}[h!]
\centering
%
\vspace{-25.5pt}
\includegraphics[height=1.80in,width=2.00in,viewport=180 0 1380 1100,clip]{Figures/Schrodinger_article.png}
\includegraphics[height=1.20in,width=2.00in,viewport=0 0 600 350,clip]{Figures/Schrodinger_Equation.png}
\label{Schrodinger_Equation}
\end{figure}
\end{minipage}
\begin{minipage}{0.49\textwidth}
\begin{figure}[h!]
\centering
%
\vspace{-15.5pt}
\includegraphics[height=2.85in,width=2.00in,viewport=0 0 780 1100,clip]{Figures/Great_Equation.jpg}
\label{Great_Equation}
\end{figure}
\end{minipage}
}

\frame
{
	\frametitle{量子力学的奠基人}
\begin{figure}[h!]
\centering
%\vspace{-25.5pt}
%\hspace*{-15.5pt}
%\includegraphics[height=0.57\textwidth,width=1.1\textwidth,viewport=0 0 2150 1050,clip]{Figures/Solvay_Conference-5-fine.jpg}
\vspace{-14.5pt}
\hspace*{-15.5pt}
\includegraphics[height=0.50\textwidth,width=0.70\textwidth,viewport=150 105 850 710,clip]{Figures/Solvay_Conference-5.jpg}
\caption{\fontsize{7.5pt}{6.2pt}\selectfont{\textrm{The Fifth Solvay International Conference, Brussels, Belgium, Oct. 1927}}}
\label{Solvay Conference-5-fine}
\end{figure}
\vspace{-11.5pt}
\fontsize{4.1pt}{3.9pt}\selectfont{\textrm{\textcolor{blue}{前排左起}:~I.Langmuir(\textcolor{blue}{朗缪尔}) M.Planck(\textcolor{blue}{普朗克}) Marie Curie(\textcolor{blue}{居里夫人}) H.Lorentz(\textcolor{blue}{洛仑兹}) A.Einstein(\textcolor{blue}{爱因斯坦}) P.Langevin(\textcolor{blue}{朗之万}) Ch.E.Guye(\textcolor{blue}{古伊}) C.T.R.Wilson(\textcolor{blue}{威尔逊}) O.W.Richardson(\textcolor{blue}{理查森})\\
\textcolor{blue}{中排左起}:~P.Debye(\textcolor{blue}{德拜}) M.Knudsen(\textcolor{blue}{克努森}) W.L.Bragg(\textcolor{blue}{布拉格}) H.A.Kramers(\textcolor{blue}{克莱默}) P.A.M.Dirac(\textcolor{blue}{狄拉克}) A.H.Compton(\textcolor{blue}{康普顿}) L.de Broglie(\textcolor{blue}{德布罗意}) M.Born(\textcolor{blue}{玻恩}) N.Bohr(\textcolor{blue}{玻尔})\\
\textcolor{blue}{后排左起}:~A.Piccard(\textcolor{blue}{皮卡尔德}) E.Henriot(\textcolor{blue}{亨利厄特}) P.Ehrenfest(\textcolor{blue}{埃伦费斯特}) Ed.Herzen(\textcolor{blue}{赫尔岑}) Th.de Donder(\textcolor{blue}{德唐德}) E.Schr\"odinger(\textcolor{blue}{薛定谔}) E.Verschaffelt(\textcolor{blue}{费尔夏费尔特}) W.Pauli(\textcolor{blue}{泡利}) W.Heisenberg(\textcolor{blue}{海森堡}) R.H.Fowler(\textcolor{blue}{富勒}) L.Brillouin(\textcolor{blue}{布里渊})}}
}

\frame
{
	\frametitle{态叠加原理:~\textrm{Schr\"odinger's cat}}
\begin{figure}[h!]
\centering
\vspace{-10.5pt}
\includegraphics[height=0.70\textwidth,width=0.48\textwidth,viewport=0 0 550 750,clip]{Figures/Schrodinger-cat.jpg}
\includegraphics[height=0.70\textwidth,width=0.50\textwidth,viewport=0 0 720 930,clip]{Figures/Schrodinger_book.jpg}
%\caption{\textrm{ABINIT}的Si.in}
\label{Schrodinger-cat}
\end{figure}
}

\frame
{
	\frametitle{因果倒置:~\textrm{Delayed Choice Experiment}}
\begin{figure}[h!]
\centering
\vspace{-10.5pt}
\includegraphics[height=0.55\textwidth,width=1.0\textwidth,viewport=0 0 690 370,clip]{Figures/Schematic-diagram-of-delayed_choice-experiment.png}
\caption{\fontsize{5.2pt}{3.9pt}\selectfont{\textrm{Schematic diagram of Wheeler's delayed choice experiment with A Mach-Zehnder Interferometer.}}}
\label{Delayed_Choice-Experiment}
\end{figure}
}

\begin{frame}
%	\frametitle{\textrm{EPR佯谬}}
	\frametitle{\textrm{Einstein}与\textrm{Bohr}的辩论}
            \begin{figure}
		    %\vskip -0.5in
        \centering
                \includegraphics[height=2.6in, width=4.0in, viewport=0 0 2119 1460,clip]{Figures/Einstein-and-Bohr-debate.jpg}
%%		\caption{\tiny{\textrm{Schematic diagram of the entanglement of two qubits.}}}
		\label{Fig:Einstein-Bohr}
            \end{figure}
\end{frame}

\begin{frame}
	\frametitle{\textrm{EPR佯谬}}
	{\fontsize{8.5pt}{5.2pt}\selectfont{\textrm{EPR提出者}根据定域实在性主张:~粒子状态在测量前已确定(\textcolor{blue}{存在隐变量}),\underline{\textcolor{purple}{量子力学的不确定性仅源于理论不完备}}}}
	\vskip 5pt
	{\fontsize{7.5pt}{5.2pt}\selectfont{根据量子力学,假设两个处于纠缠态\textrm{(entanglement)}的粒子(如自旋纠缠的粒子对)被分离到极远距离(如银河系两端)
            \begin{figure}
	    \vskip -5pt
	\centering
               % \includegraphics[height=1.2in, width=1.9in, viewport=0 0 250 160,clip]{Figures/Leveraging_principles-of-quantum_mechanics-to-securely-transmit_messages-has-promised-a-revolution-in-encryption_keeping-sensitive-information_secure.jpg}
%		\caption{\tiny{\textrm{Leveraging principles of quantum mechanics to securely transmit messages has promised a revolution in encryption keeping sensitive information secure.}}}
		%\label{Fig:Leveraging_principles-of-quantum_mechanics-to-securely-transmit_messages-has-promised-a-revolution-in-encryption_keeping-sensitive-information_secure}
                \includegraphics[height=1.0in, width=3.42in, viewport=0 0 1463 428,clip]{Figures/Quantum_entanglement-measurements-of-the-spin_state-of-one-particle-influence-the-spin_state-of-another-arbitrarily_distant_particle-with-which-the-first-is-entangled.png}
		\caption{\tiny{\textrm{Quantum entanglement measurements of the spin state of one particle influence the spin state of another arbitrarily distant particle with which the first is entangled.}}}
		\label{Fig:Quantum_entanglement-measurements-of-the-spin_state-of-one-particle-influence-the-spin_state-of-another-arbitrarily_distant_particle-with-which-the-first-is-entangled}
            \end{figure}
	    \vskip -8pt
		    \begin{itemize}
			    \item 未测量前的粒子处于叠加态(如自旋:~同时为``$\uparrow$''和``$\downarrow$''的叠加)
			    \item 测量其中一个粒子会\textcolor{red}{瞬时}确定另一个粒子的状态,\textcolor{blue}{即使它们间距达到光年或更远}
		    \end{itemize}}}
	    {\fontsize{8.5pt}{5.2pt}\selectfont{\textrm{Bohr}等基于\textrm{Copenhagen }诠释认为:~\underline{\textcolor{red}{量子态本质是概率性的}},测量导致波函数坍缩,纠缠粒子属于同一系统,无需超距作用}}
\end{frame}
%
\begin{frame}
	\frametitle{隐变量与\textrm{Bell}不等式}
%	    {\fontsize{8.5pt}{5.2pt}\selectfont{考虑两个自旋为\textrm{1/2}的粒子\textcolor{blue}{A}和\textcolor{blue}{B}构成的一个体系,在一定时刻后,使\textcolor{blue}{A}和\textcolor{blue}{B}完全分离,不再相互作用:\\
%	    \textcolor{red}{当观察者测得\textcolor{blue}{A}自旋的某一分量后,根据角动量守恒,就能确定地预言\textcolor{blue}{B}在相应\textrm{EPR}佯谬方向上的自旋值}}}
	\begin{itemize}
    \item \textrm{Bohm}隐变量理论
		    \begin{enumerate}
			    \item {\fontsize{7.5pt}{5.2pt}\selectfont{微观粒子可以有确定的位置和动量,因此可以用明确的径迹(\textrm{trajectory})来描述其运动\\
				    \textcolor{blue}{但对于粒子位置和速度的测量,依然必须遵守不确定性原理}}}
			    \item {\fontsize{7.5pt}{5.2pt}\selectfont{粒子运动接受波函数的引导,并通过与量子势\textrm{(Quantum potential)}的交互作用,表现出非局域的整体性}}
			    \item {\fontsize{7.5pt}{5.2pt}\selectfont{波函数遵守\textrm{Schr\"odiner}波动方程演化,从不坍缩}} 
		    \end{enumerate}
		    {\fontsize{8.5pt}{5.2pt}\selectfont{隐变量理论可以完全重现与传统统计性量子力学的相同的实验结果}}
     \item \textrm{Bell}不等式:~{\fontsize{8.5pt}{5.2pt}\selectfont{由\textrm{J.S. Bell}于\textrm{1964}年提出}}
	\begin{displaymath}
		\big|\mathbf{P}_{xy}-\mathbf{P}_{zy}\big|\leqslant 1+\mathbf{P}_{xy}
	\end{displaymath}
	{\fontsize{8.5pt}{5.2pt}\selectfont{\textcolor{magenta}{在定域性和实在性的双重假设下,对于两个分离的粒子同时被测量时,其结果的可能的关联程度给出严格的限制}}}
	\end{itemize}
\end{frame}

\begin{frame}
    \frametitle{量子纠缠原理}
    量子纠缠\textrm{(Quantum entanglement)}是指多个量子比特之间存在一种特殊的关联,使一个量子比特的状态会瞬间影响其他纠缠量子比特的状态,无论它们相距多远
            \begin{figure}
        \centering
                \includegraphics[height=1.5in, width=0.8in, viewport=460 0 645 380,clip]{Figures/Illustration-of-a-bit_and_qubit.png}
                \includegraphics[height=1.5in, width=2.05in, viewport=0 0 980 775,clip]{Figures/Quantum_Entanglement_chart.png}
%                \includegraphics[width=0.95\textwidth]{Figures_History/Collection_23-1.jpeg}
		\caption{\tiny{\textrm{Schematic diagram of the entanglement of two qubits.}}}
		\label{Fig:Illustration-of-a-qubit-entanglement}
            \end{figure}
	    \vskip -10pt
	    {\fontsize{7.5pt}{5.2pt}\selectfont{利用量子纠缠可以实现高效的信息传输和处理,是量子通信和量子计算的重要资源}}
\end{frame}

\frame
{
	\frametitle{量子力学量力学}
\begin{figure}[h!]
\centering
\vspace{-13.5pt}
\includegraphics[height=0.75\textwidth,width=0.55\textwidth,viewport=0 0 500 650,clip]{Figures/Quote-Niels_Bohr-on-Quantum_mechanics.png}
\caption{\fontsize{5.2pt}{3.9pt}\selectfont{\textrm{A quote of Niels Bohr on Quantum mechanics.}}}
\label{Quote-Niels_Bohr}
\end{figure}
}

\frame
{
	\frametitle{量子力学的解释}
\begin{figure}[h!]
\centering
\vspace{-12.5pt}
\includegraphics[height=0.745\textwidth,width=0.65\textwidth,viewport=300 150 1800 1870,clip]{Figures/Interpretation-QM.png}
\caption{\fontsize{5.2pt}{3.9pt}\selectfont{\textrm{Interpretation of Quantum mechanics.}}}
\label{Interpretation-QM}
\end{figure}
}

\frame
{
	\frametitle{几何原本:~公理体系的源头}
\begin{figure}[h!]
\centering
\vspace{-13pt}
\includegraphics[height=0.38\textwidth,width=0.65\textwidth,viewport=0 0 680 500,clip]{Figures/Element_Geometry_1.jpg}\\
\vspace{1pt}
\includegraphics[height=0.36\textwidth,width=0.65\textwidth,viewport=0 0 810 500,clip]{Figures/Element_Geometry_2.jpg}
%\caption{\textrm{ABINIT}的Si.in}
\label{Element_Geometru}
\end{figure}
}

%\begin{frame}
%	\frametitle{数的信仰}
%\begin{figure}[h!]
%\centering
%\vspace{-15.5pt}
%\includegraphics[height=1.30in,width=2.60in,viewport=0 0 600 290,clip]{Figures/pythagorean_numerology.png}
%\includegraphics[height=1.20in,width=2.60in,viewport=0 0 850 400,clip]{Figures/quote_pythagoras.jpg}
%\caption{\tiny 万物皆数\textrm{(All is number)}信仰}
%\label{Pythagoras_numerology}
%\end{figure}
%\end{frame}
%
%\begin{frame}
%	\frametitle{无理数的挑战}
%	\begin{itemize}
%		\item \textcolor{red}{有理数}:~\textrm{rational number} ~~~ {\fontsize{7.5pt}{6.0pt}\selectfont{\textcolor{magenta}{\textrm{rate}} $\Longrightarrow$ \textrm{ration} $\longrightarrow$ \textcolor{violet}{\textrm{rational}}}}
%		\item \textcolor{blue}{无理数}:~\textrm{irrational number}
%	\end{itemize}
%\begin{figure}[h!]
%\centering
%\vspace{-10.5pt}
%\includegraphics[height=1.80in,width=1.80in,viewport=0 0 110 110,clip]{Figures/Pi_sqrt2.jpeg}
%\includegraphics[height=1.80in,width=1.80in,viewport=0 0 317 316,clip]{Figures/coordinate-plane.png}
%\caption{\tiny \textrm{$\pi$、$\sqrt 2$、$\sqrt 3$ from a circle and a coordinate plane.}}
%\label{Pi_sqrt2}
%\end{figure}
%\end{frame}
%
%\begin{frame}
%	\frametitle{复数的物理意义}
%	\begin{displaymath}
%		\textcolor{blue}{z}=a+b\textcolor{red}{\mathbf{i}}
%	\end{displaymath}
%\begin{figure}[h!]
%\centering
%\vspace{-5.5pt}
%\includegraphics[height=1.90in,width=4.00in,viewport=0 0 1052 566,clip]{Figures/RLC_series_circuit.png}
%\caption{\tiny \textrm{The imaginary in RLC series circuit.}}
%\label{complex_imaginary}
%\end{figure}
%\end{frame}
%
\frame
{
	\frametitle{\textcolor{red}{公理体系}:~现代科学的逻辑起点}
\begin{figure}[h!]
\centering
\vspace{-10.5pt}
\includegraphics[height=0.68\textwidth,width=1.0\textwidth,viewport=0 0 770 500,clip]{Figures/Philp_Nature_Math-2.png}
%\caption{\textrm{ABINIT}的Si.in}
\label{Philp_Nature}
\end{figure}
}

\frame
{
	\frametitle{经典力学\textrm{Classical Mechanics}}
\begin{figure}[h!]
\vspace*{-0.18in}
\centering
\includegraphics[height=2.65in,width=4.05in,viewport=0 0 715 495,clip]{Figures/Classical_Mechanics.jpg}
%\includegraphics[height=2.50in,width=4.05in,viewport=0 20 735 470,clip]{Figures/Two-dark-cloud-in-physics-3.jpg}
%\includegraphics[height=2.40in,width=4.05in,viewport=0 50 735 470,clip]{Figures/Two-dark-cloud-in-physics-2.jpg}
%\includegraphics[height=2.40in,width=4.05in,viewport=0 0 580 325,clip]{Figures/Two-dark-cloud-in-physics-1.jpg}
\label{Classical_Mechanics}
\end{figure}
}

\frame
{
	\frametitle{\textrm{\small Newtonian, Lagrangian and Hamiltonian Mechanics}}
	\begin{itemize}
   		\setlength{\itemsep}{10pt}
		\item \textrm{\textcolor{blue}{Newtonian~Mechanics}}\\
		牛顿运动定律体系是以力、加速度、动量这些矢量为基本量来描述力学系统在欧氏空间的运动~(用几何方程表述约束)
	\item \textrm{\textcolor{blue}{Lagrangian~Mechanics}}\\
		拉格朗日力学是关于研究对象在其对应的约束系统下的运动形式,大大压缩牛顿方程描述需要的约束个数。不需要在另外设未知数目
	\item \textrm{\textcolor{blue}{Hamiltonian~Mechanics}}\\
		哈密度力学由拉格朗日力学演变而来,把位置和动量彻底分开,成为两种独立变量,由此诞生\textcolor{blue}{相空间}。把广义动量和广义坐标放在等同的位置上(正则配对,方程降阶)
		\vskip 6pt
		拉格朗日力学和哈密顿力学的基本量是\textcolor{blue}{系统的能量}等标量,通过变分原理建立系统的动力学方程,所以拉格朗日力学和哈密顿力学合称\textcolor{magenta}{分析力学}
	\end{itemize}
}

\frame
{
	\frametitle{\textrm{Invariante Variationsprobleme}}
\begin{figure}[h!]
\centering
%
\vspace{-10.5pt}
\includegraphics[height=0.52\textwidth,width=0.42\textwidth,viewport=0 0 450 580,clip]{Figures/Noether_theorem-1st_page.png}
\label{Noether_theorem}
\end{figure}
\begin{itemize}
\centering
	\item \textcolor{red}{能量守恒}~$\Longleftrightarrow$~\textcolor{magenta}{时间平移对称性}
	\item \textcolor{red}{动量守恒}~$\Longleftrightarrow$~\textcolor{magenta}{空间平移对称性}
	\item \textcolor{red}{角动量守恒}~$\Longleftrightarrow$~\textcolor{magenta}{空间旋转对称性}
\end{itemize}
}

\frame[allowframebreaks]
{
	\frametitle{量子力学基本假设(\textcolor{red}{公理体系})}
	\begin{itemize}
		\item 全同粒子假设\\
			\textcolor{blue}{全同粒子组成的体系中,两个全同粒子相互调换不改变体系的状态}\\ 
			全同粒子是指\textcolor{red}{内禀性质完全相同的一类微观粒子}:\\例如,所有的电子是全同粒子 
		\item 波函数假设\\
			\textcolor{blue}{微观体系的运动状态可由波函数$\Psi$完全描述,波函数包含体系的所有性质}\\
			波函数$\Psi$一般要求满足\textcolor{red}{连续}、\textcolor{red}{有限}和\textcolor{red}{单值}三个条件
		\item 微观体系的运动状态\textcolor{blue}{波函数随时间变化的规律}:\\\textcolor{red}{遵从\textrm{Schr\"odinger}方程}
			$$\mathrm{i}\hbar\dfrac{\mathrm{d}}{\mathrm{d}t}|\Psi\rangle=\hat{\mathbf H}|\Psi\rangle$$
		\item 态叠加原理\\
			如果$\Psi_1$是体系的一个本征态,对应的本征值为$A_1$,$\Psi_2$也是体系的一个本征态,对应的本征值为$A_2$,则\textcolor{blue}{$$\Psi=C_1\Psi_1+C_2\Psi_2$$}\textcolor{red}{也是体系一个可能的存在状态}
            \begin{figure}
        \centering
                \includegraphics[height=1.0in, width=1.45in, viewport=0 0 460 330,clip]{Figures/Illustration-of-a-bit_and_qubit.png}
		\hspace{0.1in}
                \includegraphics[height=1.0in, width=0.5in, viewport=0 0 85 170,clip]{Figures/American_Eagle_gold_coin_2019_2.jpg}
                \includegraphics[height=1.0in, width=1.0in, viewport=0 0 150 151,clip]{Figures/Coin-spin.jpg}
%                \includegraphics[width=0.95\textwidth]{Figures_History/Collection_23-1.jpeg}
		\caption{\tiny{\textrm{Schematic diagram of superposition/coherence of particle.}}}
		\label{Fig:Schematic-Coherence-of-particle}
            \end{figure}
		\item 力学量算符假设\\
			\textcolor{blue}{经典力学的物理量对应到量子力学中,要用线性~\textrm{Hermite}算符表示}(\textcolor{red}{\textrm{Hermite~}算符的本征函数构成完备空间})\\
			如动量算符 ~~~ $\hat{\mathbf{p}}=-\mathrm{i}\hbar\nabla$\\
			~~~位置算符 ~~~ $\hat{\mathbf r}=r$\\
			力学量算符之间有确定的对易关系(\textcolor{brown}{量子条件})
			$$[\hat{\mathbf F},\hat{\mathbf G}]=\hat{\mathbf F}\hat{\mathbf G}-\hat{\mathbf G}\hat{\mathbf F}$$ 
%\frame
%{
%	\frametitle{力学量算符的数学表示:~矩阵}
			\textcolor{magenta}{矩阵}:~力学量算符的数学表示
\begin{figure}[h!]
\centering
\vspace{-1.5pt}
%\hspace*{-0.12in}
\includegraphics[height=0.25\textwidth,width=0.65\textwidth]{Figures/Matrix_Rotation.png}
%\caption{\textrm{ABINIT}的Si.in}
\label{Matrix-Rotation}
\end{figure}
	\end{itemize}
}

\frame
{
	\frametitle{量子化学学科创立}
	\begin{itemize}
		\item \textrm{1927}年,\textrm{\o.~Burrau}应用量子力学原理,完成\textrm{\ch{H2+}}离子的计算
		\item 同年,\textrm{Walter~Heitlery}和\textrm{Fritz~W.~London}对\textrm{\ch{H2}}分子的计算,标志着量子化学这一学科正式创立
	\end{itemize}
\begin{figure}[h!]
\centering
\vspace{-1.5pt}
\hspace*{-0.12in}
\includegraphics[height=0.48\textwidth,width=0.80\textwidth,viewport=0 10 260 175,clip]{Figures/Walter-Heitlery_Fritz-W-London.jpeg}
\caption{\textrm{W.~Heitlery (left) and F.~W.~London(right).}}
\label{Heitlery_London}
\end{figure}
}

%\frame
%{
%	\frametitle{\textrm{Paul Adrian Maurice Dirac's Commandments}}
%	\textrm{The underlying laws necessary for the mathematical treatment of a large part of physics \textcolor{red}{and the whole of chemistry} are thus completely known, and the difficulty lies only in the fact that application of these laws leads to equations that are \underline{too complex to be solved}.
%\vskip 15pt
%It therefore becomes desirable that approximation practical methods of applying quantum mechanics should be develop $\cdots$ 
%}

%\vskip 15pt
%\textrm{P.A.M Dirac Proc. Roy. Soc. Ser. A, \textbf{123}, 714, (1929)}
%}
%
\frame
{
%	\frametitle{\rm{Paul Dirac's Commandments\upcite{PRSLSA123-714_1929}}}
	\frametitle{\textrm{Paul Dirac's Commandments}}%\upcite{PRSLSA123-714_1929}}}
%	\textrm{\textcolor{purple}{The underlying laws necessary for the mathematical treatment of a large part of physics and the whole of chemistry are thus completely known, and the difficulty lies only in the fact that application of these laws leads to equations that are too complex to be solved.}}
\begin{figure}[h!]
\centering
\vspace{-10.5pt}
\includegraphics[height=0.71\textwidth,width=0.9\textwidth,viewport=0 0 1150 920,clip]{Figures/Dirac_comment.png}
%\caption{\textrm{ABINIT}的Si.in}
\label{Diract_Commandment}
\end{figure}
}

\frame
{
%	\frametitle{\textrm{DFT-SCF}}
\begin{figure}[h!]
\vspace*{-0.25in}
\centering
\includegraphics[height=2.80in,width=4.95in,viewport=5 3 1250 780,clip]{Figures/Method_Procedure.png}
%\caption{\tiny \textrm{Pseudopotential for metallic sodium, based on the empty core model and screened by the Thomas-Fermi dielectric function.}}%(与文献\cite{EPJB33-47_2003}图1对比)
\label{Method-Procedure}
\end{figure}
}

\frame
{
	\frametitle{王守竞先生与量子力学}
	王守竞(\textrm{Shou~Chin~Wang})的工作
	\begin{itemize}
		\item 计算氢分子的电子结构
			\vskip 2.5pt
			王守竞的论文\textcolor{purple}{《新量子力学下的常态氢分子问题》}%THE PROBLEM OF THE NORMAL HYDROGEN MOLECULE IN THE NEW QUANTUM MECHANICS
	\textcolor{blue}{
	\begin{displaymath}
		\Psi=C\left\{ \mathrm{exp}[-Z(r_1+ p_2)/a]+\mathrm{exp}[-Z( r_2+ p_1)/a] \right\}
	\end{displaymath}}
	{\fontsize{7.2pt}{6.5pt}\selectfont{其中$r_1$、$p_1$是第一个电子到两个原子核的距离,$r_2$,$p_2$是第二个电子到两个原子核的距离,$a$是\textrm{Bohr}半径}}\\
	得到的数值结果$Z=1.666$,$E_0=86.6\,\mathrm{kcal}$,$R_0=0.78$\,\textrm{\AA}
		\item 不对称陀螺(不对称转动)的能谱%On the Asymmetrical Top in Quantum Mechanics
			\vskip 2.5pt
			不对称陀螺的能级公式(\textcolor{purple}{“王氏公式”})
	\textcolor{blue}{
	\begin{displaymath}
		E= (hc8\pi)[Aj(j+1)+W]
	\end{displaymath}}
	\end{itemize}
}

\frame
{
	\frametitle{王守竞先生与量子力学}
\begin{figure}[h!]
\centering
\vspace{-10.5pt}
\includegraphics[height=0.66\textwidth,width=0.52\textwidth,viewport=0 0 270 350,clip]{Figures/Wang_Shoujing.jpg}
\caption{王守竞先生(1902-1984)}
\label{Wang_Shoujing}
\end{figure}
}

\frame
{
	\frametitle{王守竞先生与量子力学}
\begin{figure}[h!]
\centering
\vspace{-10.5pt}
\includegraphics[height=0.65\textwidth,width=1.0\textwidth,viewport=0 0 560 350,clip]{Figures/Collect_Wang.jpg}
\caption{\fontsize{7.2pt}{6.5pt}\selectfont{\textrm{左1:~王守竞,左2:~Ralph~Kronig 右1:~I.~I.~Rabi(1944年诺贝尔物理学奖获得者)}}}
\label{Collect_Wang}
\end{figure}
}

\frame
{
	\frametitle{}
\begin{figure}[h!]
\centering
\hspace*{-10.5pt}
\includegraphics[height=0.42\textwidth,width=1.05\textwidth,viewport=0 0 860 350,clip]{Figures/Wang_Family_Suzhou.jpg}
\caption{\fontsize{6.2pt}{5.5pt}\selectfont{\textrm{苏州王氏家族中的著名学者}}}
\label{Wang_Family}
\end{figure}
}

%-----------------------------------------------------------------------------
%------------------------------------------------------------------------Reference----------------------------------------------------------------------------------------------
