%%%%%%%%%%%%%%%%仅限于XeTeX可使用的宏包%%%%%%%%%%%%%%%%%%%%%%%%%%%%
\usepackage{fontspec,xunicode,xltxtra,beamerthemesplit}
%\usepackage{beamerthemesplit}
%\usepackage{handoutWithNotes}		%(讲义)在打印PPT的时候会留出给每一页做注释的部分
\usepackage[dvipdfmx]{movie15_dvipdfmx} %插入视频与(讲义)打印有冲突
\usepackage{xeCJK}
%\setCJKmainfont[BoldFont=黑体, ItalicFont=楷体, BoldItalicFont=仿宋]{黑体}
%\setsansfont[Mapping=tex-text]{Adobe 黑体 Std}
%如果装了Adobe Acrobat,可在font.conf中配置Adobe字体的路径以使用其中文字体
%也可直接使用系统中的中文字体如SimSun,SimHei,微软雅黑 等
%原来beamer用的字体是sans family;注意Mapping的大小写,不能写错

\usepackage{listings} 
\lstset{language=Matlab}%代码语言使用的是matlab 
\lstset{breaklines}%自动将长的代码行换行排版 
\lstset{extendedchars=false}%解决代码跨页时,章节标\dots

%%%%%%%%   确定标题和导航条结构的框架     %%%%%%%%%%%%
\usepackage{beamerthemeshadow}                       %
%\usepackage{beamerthemeclassic}%导航条色与背景色一致%
%\usepackage{authblk}				     %作者地址和E-mail
%%%%%%%%%%%%%%%%%%%%%%%%%%%%%%%%%%%%%%%%%%%%%%%%%%%%%%
\setbeamerfont{roman title}{size={}}
%\usepackage{CJK} % CJK 中文支持                                  %
%\usepackage[version=3]{mhchem}		%化学公式
\usepackage{chemformula}
\usepackage{chemfig}		%化学公式

\usepackage{amsmath,amsthm,amsfonts,amssymb,bm}
\usepackage{bbding}
\usepackage{mathrsfs}
\usepackage{xcolor}                                        %使用默认允许使用颜色
\usepackage{hyperref} 
\usepackage{graphicx}
\usepackage{float}               %将图片定死在某一个位置用(主要支持[htbp!]中的h)
\usepackage{subfigure}           %图片跨页
\usepackage[controls]{animate}	 %插入动画
\usepackage{tikz}		 %绘图工具
\usetikzlibrary{%
    arrows,shapes,chains,shapes,arrows.meta,matrix,
    graphs, decorations, 
    decorations.markings, 
    decorations.pathmorphing, 
%    graphdrawing,                                       % requires lualatex
    shapes.geometric, snakes
}
%\usegdlibrary{trees,force, layered}                     % requires lualatex
\usepackage{adjustbox}                                   %绘制跨页流程图形
\newsavebox{\mysavebox}                                  %绘制跨页流程图形
\newlength{\myrest}                                      %绘制跨页流程图形

\usepackage{pgfplots}
%\pgfplotsset{width=10cm,compat=1.9}                     %每个pgfplot图形的大小更改为10cm
%\usepgfplotslibrary{external}                           %以将图形导出为单独的PDF文件,然后将其导入文档中
%\tikzexternalize
\usepackage{caption}
\captionsetup{font=footnotesize}

\usepackage{verbatim}			%Verbatim 宏包重新实现了 Verbatim 环境,并且提供一个命令可以导入一个 ASCII 文件到文档中
\usepackage{multirow}
\usepackage{makecell}		%允许单元格内换行

%\pgfpagesuselayout{1 on 1 with notes landscape}[a4paper,border shrink=5mm]

\usepackage{booktabs}           %修改表格线段的粗细,可以自定义修改线段粗细
%\toprule[2pt]                   %表格顶端线粗细设置
%\midrule[1pt]                   %表格中间线粗细设置
%\bottomrule[1.8pt]              %表格底端线粗细设置

%%%%%%%%%%%%%%%%%%%%%%BIBTEX 引用参考文献%%%%%%%%%%%%%%%%%%%%%%%%%%%%%%%%%%%%%%%%%%%%%%%%
%\usepackage{filecontents}
%\begin{filecontents*}{main.bib}
%@techreport{2012FracfocusChemical,
%  author = {FracFocus,},
%  howpublished = {\url{http://fracfocus.org/water-protection/drilling-usage}},
%  institution = {The Ground Water Protection Council and Interstate Oil and Gas
%  Compact Commission},
%  month = {feb},
%  title = {{Chemical Use In Hydraulic Fracturing}},
%  year = {2012}
%}
%\end{filecontents*}
%\usepackage[backend=bibtex,sorting=none]{biblatex}
%%\usepackage[backend=biber,style=authoryear]{biblatex}
%\addbibresource{main.bib} %BibTeX数据文件及位置

%\usepackage[numbers,sort&compress]{natbib} %紧密排列             %
\usepackage[sectionbib]{chapterbib}        %每章节单独参考文献   %
\usepackage{hypernat}                                                                         %
\setbeamertemplate{bibliography item}[text] %参考文献前标注[]
%\usepackage[dvipdfm,bookmarksopen=true,pdfstartview=FitH,CJKbookmarks]{hyperref}		%
\hypersetup{bookmarksnumbered,colorlinks,linkcolor=brown,citecolor=blue,urlcolor=red}         %
%参考文献含有超链接引用时需要下列宏包,注意与natbib有冲突        %
%\usepackage[dvipdfm]{hyperref}                                  %
%\usepackage{hypernat}                                           %
\newcommand{\upcite}[1]{\hspace{0ex}\textsuperscript{\cite{#1}}} %

%\usepackage{marvosym} %插入各种符号

%\useoutertheme{smoothbars}
\useinnertheme[shadow=true]{rounded}

% Beamer Settings
\usetheme{Berkeley}                                          %主题式样
%\usetheme{Luebeck}
%\usetheme{Warsaw}

\usecolortheme{lily}                                        %颜色主题式样

\usefonttheme{professionalfonts}                           %字体主题样式宏包

%\beamertemplatetransparentcoveredhigh                      %使所有被隐藏的文本高度透明
\beamertemplatetransparentcovereddynamicmedium             %使所有被隐藏的文本完全透明,动态,动态的范围很小
\mode<presentation>
%\beamersetaveragebackground{gray}                          %设置背景颜色(单一色) 
\beamertemplateshadingbackground{green!10}{red!5}         %设置背景颜色(渐变色)

\graphicspath{{$PATHPWD/Figures/}}                           %$PATH用于脚本指定图片路径
%\graphicspath{{/home/jun-jiang/Documents/Latex_Beamer/Figures/}}   %直接指定图片绝对路径
%\graphicspath{{/home/jiangjun/Documents/Latex_Beamer/Figures/}}   %直接指定图片绝对路径

%i放置单位logo
%\logo{\includegraphics[width=1.6cm,height=0.35cm]{Figures/BCC_logo-1.png}}	%简单设置logo

%\pgfdeclareimage[width=3.5cm]{logoname}{Figures/BCC_logo-1.png}		%logo置于左侧微调
%\logo{\pgfuseimage{logoname}{\vspace{0.2cm}\hspace*{-2.0cm}}}

%在指定位置精确放置logo
\usepackage{beamerfoils}
\usepackage{pgf}
\logo{\pgfputat{\pgfxy(11.68,0.15)}{\includegraphics[height=1.01cm,viewport=0 0 140 120,clip]{Figures/BCC_logo-1.png}}\pgfputat{\pgfxy(10.502,-0.218)}{\includegraphics[height=0.369cm,viewport=140 0 540 120,clip]{Figures/BCC_logo-1.png}}}
%\logo{\pgfputat{\pgfxy(11.68,0.15)}{\includegraphics[height=0.95cm,viewport=0 0 510 360,clip]{Figures/Logo_Gainstrong.png}}\pgfputat{\pgfxy(10.333,-0.195)}{\includegraphics[height=0.35cm,viewport=530 70 1100 218,clip]{Figures/Logo_Gainstrong.png}}}
%\logo{\pgfputat{\pgfxy(10.28,0.00)}{\includegraphics[height=0.95cm,viewport=0 0 1100 360,clip]{Figures/Logo_Gainstrong.png}}}
%\logo{\pgfputat{\pgfxy(11.68,0.15)}{\includegraphics[height=0.95cm,viewport=0 0 510 360,clip]{Figures/Logo_Gainstrong.png}}\pgfputat{\pgfxy(10.333,-0.195)}{\includegraphics[height=0.35cm,viewport=530 70 1100 218,clip]{Figures/Logo_Gainstrong.png}}}
%\logo{\pgfputat{\pgfxy(10.68,0.00)}{\includegraphics[height=1.20cm,viewport=0 15 400 430,clip]{Figures/seal_Jiang-2.jpg}}
      %\pgfputat{\pgfxy(10.817,-0.218)}{\includegraphics[height=0.47cm,viewport=20 0 670 350,clip]{Figures/signature_Jiang_new.jpg}}
%}
%\logo{\pgfputat{\pgfxy(11.68,0.252)}{\includegraphics[height=0.89cm,viewport=0 15 810 800,clip]{Figures/seal_Jiang-new.jpg}}\pgfputat{\pgfxy(10.762,-0.218)}{\includegraphics[height=0.49cm,viewport=20 0 670 350,clip]{Figures/signature_Jiang_new.jpg}}}
%\logo{\pgfputat{\pgfxy(11.68,0.252)}{\includegraphics[height=0.90cm,viewport=0 15 400 430,clip]{Figures/seal_Jiang-2.jpg}}\pgfputat{\pgfxy(10.817,-0.218)}{\includegraphics[height=0.47cm,viewport=20 0 670 350,clip]{Figures/signature_Jiang_new.jpg}}}
%\MyLogo{
%	\pgfputat{\pgfxy(-50,-50)}{\pgfbox[right,base]{\includegraphics[height=1cm]{Figures/BCC_logo-1.png}}}

%logo作为背景放置
%\setbeamertemplate{background}{
%	\pgfputat{\pgfxy(6.5,-0.5)}{\pgfbox[left,top]{\pgfimage[height=1.1cm]{Figures/BCC_logo-1.png}}}}

%\logo{}									%不显示logo

%-----------------------------------------------------------------------------
