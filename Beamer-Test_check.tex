%%%%%%%%%%%%%%%%%%%%%%%%%%%%%%%%%%%%%%%%%%  不使用 authblk 包制作标题  %%%%%%%%%%%%%%%%%%%%%%%%%%%%%%%%%%%%%%%%%%%%%%
%-------------------------------PPT Title-------------------------------------
\title{第一原理计算软件简介与应用}
%-----------------------------------------------------------------------------

%----------------------------Author & Date------------------------------------
%\author[\textrm{Jun\_Jiang}]{姜\;\;骏\inst{}} %[]{} (optional, use only with lots of authors)
%% - Give the names in the same order as the appear in the paper.
%% - Use the \inst{?} command only if the authors have different
%%   affiliation.
\institute[BCC]{\inst{}%
%\institute[Gain~Strong]{\inst{}%
\vskip -20pt 北京市计算中心~云平台事业部~姜骏}
%\vskip -20pt {\large 格致斯创~科技}}
\date[\today] % (optional, should be abbreviation of conference name)
{	{\fontsize{6.2pt}{4.2pt}\selectfont{\textcolor{blue}{E-mail:~}\url{jiangjun@bcc.ac.cn}}}
\vskip 45 pt {\fontsize{8.2pt}{6.2pt}\selectfont{北京科技大学~理化楼-308% 报告地点
	\vskip 5 pt \textrm{2023.07.12}}}
}

%% - Either use conference name or its abbreviation
%% - Not really information to the audience, more for people (including
%%   yourself) who are reading the slides onlin%%   yourself) who are reading the slides onlin%%   yourself) who are reading the slides onlineee
%%%%%%%%%%%%%%%%%%%%%%%%%%%%%%%%%%%%%%%%%%%%%%%%%%%%%%%%%%%%%%%%%%%%%%%%%%%%%%%%%%%%%%%%%%%%%%%%%%%%%%%%%%%%%%%%%%%%%

\subject{}
% This is only inserted into the PDF information catalog. Can be left
% out.
%\maketitle
\frame
{
%	\frametitle{\fontsize{9.5pt}{5.2pt}\selectfont{\textcolor{orange}{“高通量并发式材料计算算法与软件”年度检查}}}
\titlepage
}
%-----------------------------------------------------------------------------

%------------------------------------------------------------------------------列出全文 outline ---------------------------------------------------------------------------------
\section*{}
\frame[allowframebreaks]
{
  \frametitle{Outline}
%  \frametitle{\textcolor{mycolor}{\secname}}
  \tableofcontents%[current,currentsection,currentsubsection]
}
%%在每个section之前列出全部Outline
%%类似的在每个subsection之前列出全部Outline是\AtBeginSubsection[]
%\AtBeginSection[]
%{
%  \frame<handout:0>%[allowframebreaks]
%  {
%    \frametitle{Outline}
%%全部Outline中,本部分加亮
%    \tableofcontents[current,currentsection]
%  }
%}

%-----------------------------------------------PPT main Body------------------------------------------------------------------------------------
\small
\frame
{
	\frametitle{现有高通量计算平台概览}
\begin{table}[!h]
\tabcolsep 0pt \vspace*{-12pt}
%\caption{}
\label{Table-Cost}
\begin{minipage}{0.85\textwidth}
%\begin{center}
\centering
\def\temptablewidth{1.1\textwidth}
\renewcommand\arraystretch{0.8} %表格宽度控制(普通表格宽度的两倍)
\rule{\temptablewidth}{1pt}
\begin{tabular*} {\temptablewidth}{@{\extracolsep{\fill}}c@{\extracolsep{\fill}}c@{\extracolsep{\fill}}c@{\extracolsep{\fill}}c@{\extracolsep{\fill}}c@{\extracolsep{\fill}}c@{\extracolsep{\fill}}c}
%-------------------------------------------------------------------------------------------------------------------------
	&\multirow{2}{*}{\fontsize{7.2pt}{5.2pt}\selectfont{编程语言}}	&\fontsize{7.2pt}{5.2pt}\selectfont{建模} &\multicolumn{2}{|c|}{\fontsize{6.2pt}{5.2pt}\selectfont{任务提交与管理}} &\multirow{2}{*}{\fontsize{7.2pt}{5.2pt}\selectfont{后处理}} &\multirow{2}{*}{\fontsize{6.2pt}{5.2pt}\selectfont{数据组织管理}} \\\cline{4-5}
	&	&\fontsize{7.2pt}{5.2pt}\selectfont{功能} &\multicolumn{1}{|l}{\fontsize{7.2pt}{5.2pt}\selectfont{软件接口}} &\multicolumn{1}{r|}{\fontsize{7.2pt}{5.2pt}\selectfont{运行容错}} & & \\\hline
	\fontsize{7.2pt}{5.2pt}\selectfont{{AFLOW}} &\fontsize{7.2pt}{5.2pt}\selectfont{C++} &\checkmark &$\triangle$ &\FiveStarOpen &\FiveStarOpen &\fontsize{7.2pt}{5.2pt}\selectfont{{Django}} \\
	\fontsize{7.2pt}{5.2pt}\selectfont{{MP}} &\fontsize{7.2pt}{5.2pt}\selectfont{Python} &\checkmark &\checkmark &\FiveStarOpen &\FiveStarOpen &\fontsize{7.2pt}{5.2pt}\selectfont{{MongoDB}} \\
	\multirow{2}{*}{\fontsize{7.2pt}{5.2pt}\selectfont{{QMIP}}} &\fontsize{7.2pt}{5.2pt}\selectfont{JavaScript/SVG} &\multirow{2}{*}{\checkmark} &\multirow{2}{*}{\checkmark} &\multirow{2}{*}{--} &\multirow{2}{*}{\checkmark} &\multirow{2}{*}{--} \\
	&\fontsize{7.2pt}{5.2pt}\selectfont{+html/Python} & & & & & \\
	\fontsize{7.2pt}{5.2pt}\selectfont{{CEP}} &\fontsize{7.2pt}{5.2pt}\selectfont{Python} &\checkmark &\checkmark &-- &\checkmark &\fontsize{7.2pt}{5.2pt}\selectfont{{Django/MySQL}} \\
	\fontsize{7.2pt}{5.2pt}\selectfont{{ASE}} &\fontsize{7.2pt}{5.2pt}\selectfont{Python} &\FiveStarOpen &\FiveStarOpen &-- &$\triangle$ &-- \\
	\multirow{2}{*}{\fontsize{7.2pt}{5.2pt}\selectfont{{MatCloud}}} &\fontsize{7.2pt}{5.2pt}\selectfont{JavaScript} &\multirow{2}{*}{\checkmark} &\multirow{2}{*}{$\triangle$} &\multirow{2}{*}{\checkmark} &\multirow{2}{*}{\checkmark} &\multirow{2}{*}{\fontsize{7.2pt}{5.2pt}\selectfont{{MongoDB}}} \\
	&\fontsize{7.2pt}{5.2pt}\selectfont{+.NETCore} & & & & &
\end{tabular*}
\rule{\temptablewidth}{1pt}
\end{minipage}
%\vskip -15pt
\fontsize{7.2pt}{5.2pt}\selectfont{
\begin{description}
	\item[\FiveStarOpen]~该功能较突出
	\item[\checkmark]~该功能基本满足需求
	\item[$\triangle$]~该功能存在不足
\end{description}}
%\end{center}
\end{table}
\fontsize{8.2pt}{6.2pt}\selectfont{
	\textrm{Lin L. \textit{Materials databases infrastructure constructed by first principles calculations:~a review.} \textbf{Mater. Perform. Character.}, 2015, 4(1):148.}}
}

\frame
{
	\frametitle{}
\begin{figure}[!h]
	\centering
	\begin{tikzpicture}[
    box-rect/.style={rectangle,draw=none,node distance=1cm,text width=15em,text centered,rounded corners,minimum height=2em,thick},
    straightline/.style = {line width = 2pt,-},
    arrow/.style={draw,-latex', line width = 2pt},
		]
		\node [box-rect, text width=2.0em, minimum height=1.5em, very thin] at(-4.0, 2.3 )(Melody) {\fontsize{3.5pt}{2.5pt}\selectfont{唱腔}}; % yshift=5.0em 表示竖直方向,向上移动3.5em
		\node [box-rect, below=0.04 of Melody, fill=green!40, text width=0.75em, minimum height=0.1em, very thin] (Nature) {\fontsize{3.5pt}{2.5pt}\selectfont{自然}};
		\node [box-rect, below=0.04 of Nature, fill=green!40, text width=7.5em, minimum height=0.1em, very thin] (Tech-Motion) {\fontsize{3.5pt}{2.5pt}\selectfont{技:~刚、实、宽、厚\hspace{5em}薄、窄、虚、柔:~情}};
		\node [box-rect, below=0.04 of Tech-Motion, fill=green!40, text width=1.5em, minimum height=0.1em, very thin] (Tan-Xinpei) {\fontsize{3.5pt}{2.5pt}\selectfont{谭鑫培}};
		\node [box-rect, below=0.04 of Tan-Xinpei, fill=green!40, text width=1.5em, minimum height=0.1em, very thin] (Yang-Xiaolou) {\fontsize{3.5pt}{2.5pt}\selectfont{杨小楼}};
		\node [box-rect, below=0.45 of Tan-Xinpei, fill=green!40, text width=1.5em, minimum height=0.1em, very thin] (Tan-Xiaopei) {\fontsize{3.5pt}{2.5pt}\selectfont{谭小培}};
		\node [box-rect, below=0.04 of Tan-Xiaopei, fill=green!40, text width=1.5em, minimum height=0.1em, very thin] (Five-Tan) {\fontsize{3.5pt}{2.5pt}\selectfont{``五坛''}};
		\node [box-rect, below=0.04 of Five-Tan, fill=green!40, text width=1.5em, minimum height=0.1em, very thin] (Wang-Youchen) {\fontsize{3.5pt}{2.5pt}\selectfont{王又宸}};
		\node [box-rect, below=0.04 of Wang-Youchen, fill=green!40, text width=1.5em, minimum height=0.1em, very thin] (Chen-Yanheng) {\fontsize{3.5pt}{2.5pt}\selectfont{陈彦衡}};
		\node [box-rect, below=0.04 of Chen-Yanheng, fill=green!40, text width=3.75em, minimum height=0.1em, very thin] (Guan-Chen) {\fontsize{3.5pt}{2.5pt}\selectfont{贯大元\hspace{2em}陈秀华}};
		\node [box-rect, below=0.04 of Guan-Chen, fill=green!40, text width=1.5em, minimum height=0.1em, very thin] (Yu-Shuyan) {\fontsize{3.5pt}{2.5pt}\selectfont{余叔岩}};
		\node [box-rect, below=0.04 of Yu-Shuyan, fill=green!40, text width=1.5em, minimum height=0.1em, very thin] (Yan-Jupeng) {\fontsize{3.5pt}{2.5pt}\selectfont{言菊朋}};
		\node [box-rect, below=0.04 of Yan-Jupeng, fill=green!40, text width=7.5em, minimum height=0.1em, very thin] (Double-Wang-Li-Zhang) {\fontsize{3.5pt}{2.5pt}\selectfont{王君直\hspace{2em}王荣山\hspace{2em}李适可\hspace{2em}张伯驹}};
		\node [box-rect, below=0.04 of Double-Wang-Li-Zhang, fill=green!40, text width=3.75em, minimum height=0.1em, very thin] (Fan-Han) {\fontsize{3.5pt}{2.5pt}\selectfont{范濂泉\hspace{2em}韩慎先}};
		\node [box-rect, below=0.04 of Fan-Han, fill=green!40, text width=1.5em, minimum height=0.1em, very thin] (Yang-Baosen) {\fontsize{3.5pt}{2.5pt}\selectfont{杨宝森}};
		\node [box-rect, below=0.04 of Yang-Baosen, fill=green!40, text width=1.5em, minimum height=0.1em, very thin] (Meng-Xiaodong) {\fontsize{3.5pt}{2.5pt}\selectfont{孟小冬}};
		\node [box-rect, below=0.04 of Meng-Xiaodong, fill=green!40, text width=1.5em, minimum height=0.1em, very thin] (Xi-Xiaobo) {\fontsize{3.5pt}{2.5pt}\selectfont{奚啸伯}};

\path
(Nature.west)++(-3.75, 0.0) coordinate(Nature-)
(Nature.east)++(3.75, 0.0) coordinate(Nature+)
(Tech-Motion.west)++(-2.55, 0.0) coordinate(Tech-Motion-)
(Tech-Motion.east)++(2.55, 0.0) coordinate(Tech-Motion+)

(Tan-Xinpei.west)++(-3.6, 0.0) coordinate(Tan-Xinpei-)
(Tan-Xinpei.east)++(3.6, 0.0) coordinate(Tan-Xinpei+)
(Yang-Xiaolou.west)++(-3.5, 0.0) coordinate(Yang-Xiaolou-)
(Yang-Xiaolou.east)++(3.5, 0.0) coordinate(Yang-Xiaolou+)
(Tan-Xiaopei.west)++(-2.2, 0.0) coordinate(Tan-Xiaopei-)
(Tan-Xiaopei.east)++(2.2, 0.0) coordinate(Tan-Xiaopei+)
(Five-Tan.west)++(-2.5, 0.0) coordinate(Five-Tan-)
(Five-Tan.east)++(2.5, 0.0) coordinate(Five-Tan+)
(Wang-Youchen.west)++(-2.7, 0.0) coordinate(Wang-Youchen-)
(Wang-Youchen.east)++(2.7, 0.0) coordinate(Wang-Youchen+)
(Chen-Yanheng.west)++(-3.0, 0.0) coordinate(Chen-Yanheng-)
(Chen-Yanheng.east)++(3.0, 0.0) coordinate(Chen-Yanheng+)
(Guan-Chen.west)++(-2.8, 0.0) coordinate(Guan-Chen-)
(Guan-Chen.east)++(2.8, 0.0) coordinate(Guan-Chen+)

(Yu-Shuyan.west)++(-3.8, 0.0) coordinate(Yu-Shuyan-)
(Yu-Shuyan.east)++(3.1, 0.0) coordinate(Yu-Shuyan+)
(Yan-Jupeng.west)++(-3.0, 0.0) coordinate(Yan-Jupeng-)
(Yan-Jupeng.east)++(3.8, 0.0) coordinate(Yan-Jupeng+)
(Double-Wang-Li-Zhang.west)++(-1.2, 0.0) coordinate(Double-Wang-Li-Zhang-)
(Double-Wang-Li-Zhang.east)++(1.2, 0.0) coordinate(Double-Wang-Li-Zhang+)
(Fan-Han.west)++(-2.9, 0.0) coordinate(Fan-Han-)
(Fan-Han.east)++(3.0, 0.0) coordinate(Fan-Han+)
(Yang-Baosen.west)++(-3.45, 0.0) coordinate(Yang-Baosen-)
(Yang-Baosen.east)++(3.45, 0.0) coordinate(Yang-Baosen+)
(Meng-Xiaodong.west)++(-3.70, 0.0) coordinate(Meng-Xiaodong-)
(Meng-Xiaodong.east)++(2.8, 0.0) coordinate(Meng-Xiaodong+)
(Xi-Xiaobo.west)++(-3.0, 0.0) coordinate(Xi-Xiaobo-)
(Xi-Xiaobo.east)++(3.55, 0.0) coordinate(Xi-Xiaobo+);

\draw [dotted, very thick](-8, -4.5) -- (-8, 2.5);
\draw [dotted, very thick](0, -10.0) -- (0, 2.5);
		\draw [straightline, dashed, thin](Nature.west) -- (Nature-);
		\draw [straightline, dashed, thin](Nature.east) -- (Nature+);
		\path [arrow, NavyBlue, thick](Tech-Motion.west) -- (Tech-Motion-);
		\path [arrow, NavyBlue, thick](Tech-Motion.east) -- (Tech-Motion+);

		\draw [solid, ultra thick](-8, 0) -- (-8, -.4);
		\draw [solid, ultra thick](8, 0) -- (8, -0.4);
		\draw [straightline, dash dot, thin](Tan-Xinpei.west) -- (Tan-Xinpei-);
		\draw [straightline, dash dot, thin](Tan-Xinpei.east) -- (Tan-Xinpei+);

		\draw [solid, red, very thick](-7.9, -0.75) -- (-7.9, -1.15);
		\draw [solid, red, very thick](7.9, -0.75) -- (7.9, -1.15);
		\draw [straightline, dash dot dot, thin](Yang-Xiaolou.west) -- (Yang-Xiaolou-);
		\draw [straightline, dash dot dot, thin](Yang-Xiaolou.east) -- (Yang-Xiaolou+);

		\draw [solid, very thick](-5, -1.5) -- (-5, -1.9);
		\draw [solid, very thick](5, -1.5) -- (5, -1.9);
		\draw [straightline, dash dot, thin](Tan-Xiaopei.west) -- (Tan-Xiaopei-);
		\draw [straightline, dash dot, thin](Tan-Xiaopei.east) -- (Tan-Xiaopei+);

		\draw [solid, very thick](-6, -2.20) -- (-6, -2.60);
		\draw [solid, very thick](6, -2.20) -- (6, -2.60);
		\draw [straightline, dash dot, thin](Five-Tan.west) -- (Five-Tan-);
		\draw [straightline, dash dot, thin](Five-Tan.east) -- (Five-Tan+);

		\draw [solid, very thick](-6.5, -2.95) -- (-6.5, -3.35);
		\draw [solid, very thick](6.5, -2.95) -- (6.5, -3.35);
		\draw [straightline, dash dot, thin](Wang-Youchen.west) -- (Wang-Youchen-);
		\draw [straightline, dash dot, thin](Wang-Youchen.east) -- (Wang-Youchen+);

		\draw [solid, very thick](-7.0, -3.65) -- (-7.0, -4.05);
		\draw [solid, very thick](7.0, -3.65) -- (7.0, -4.05);
		\draw [straightline, dash dot, thin](Chen-Yanheng.west) -- (Chen-Yanheng-);
		\draw [straightline, dash dot, thin](Chen-Yanheng.east) -- (Chen-Yanheng+);

		\draw [solid, very thick](-6.8, -4.4) -- (-6.8, -4.8);
		\draw [solid, very thick](6.8, -4.4) -- (6.8, -4.8);
		\draw [straightline, dash dot, thin](Guan-Chen.west) -- (Guan-Chen-);
		\draw [straightline, dash dot, thin](Guan-Chen.east) -- (Guan-Chen+);

		\draw [solid, very thick](-8.5, -5.10) -- (-8.5, -5.50);
		\draw [solid, very thick](6.8, -5.10) -- (6.8, -5.50);
		\draw [straightline, dash dot, thin](Yu-Shuyan.west) -- (Yu-Shuyan-);
		\draw [straightline, dash dot, thin](Yu-Shuyan.east) -- (Yu-Shuyan+);

		\draw [solid, very thick](-6.5, -5.85) -- (-6.5, -6.25);
		\draw [solid, very thick](8.5, -5.85) -- (8.5, -6.25);
		\draw [straightline, dash dot, thin](Yan-Jupeng.west) -- (Yan-Jupeng-);
		\draw [straightline, dash dot, thin](Yan-Jupeng.east) -- (Yan-Jupeng+);

		\draw [solid, very thick](-7.5, -6.60) -- (-7.5, -7.00);
		\draw [solid, very thick](6.4, -6.60) -- (6.4, -7.00);
		\draw [straightline, dash dot, thin](Double-Wang-Li-Zhang.west) -- (Double-Wang-Li-Zhang-);
		\draw [straightline, dash dot, thin](Double-Wang-Li-Zhang.east) -- (Double-Wang-Li-Zhang+);

		\draw [solid, very thick](-6.4, -7.30) -- (-6.4, -7.70);
		\draw [solid, very thick](7.5, -7.30) -- (7.5, -7.70);
		\draw [straightline, dash dot, thin](Fan-Han.west) -- (Fan-Han-);
		\draw [straightline, dash dot, thin](Fan-Han.east) -- (Fan-Han+);

		\draw [solid, very thick](-7.8, -8.05) -- (-7.8, -8.45);
		\draw [solid, very thick](6.5, -8.05) -- (6.5, -8.45);
		\draw [straightline, dash dot, thin](Yang-Baosen.west) -- (Yang-Baosen-);
		\draw [straightline, dash dot, thin](Yang-Baosen.east) -- (Yang-Baosen+);

		\draw [solid, very thick](-8.2, -8.75) -- (-8.2, -9.15);
		\draw [solid, very thick](6.5, -8.75) -- (6.5, -9.15);
		\draw [straightline, dash dot, thin](Meng-Xiaodong.west) -- (Meng-Xiaodong-);
		\draw [straightline, dash dot, thin](Meng-Xiaodong.east) -- (Meng-Xiaodong+);

		\draw [solid, very thick](-6.3, -9.50) -- (-6.3, -9.90);
		\draw [solid, very thick](7.8, -9.50) -- (7.8, -9.90);
		\draw [straightline, dash dot, thin](Xi-Xiaobo.west) -- (Xi-Xiaobo-);
		\draw [straightline, dash dot, thin](Xi-Xiaobo.east) -- (Xi-Xiaobo+);
\end{tikzpicture}
%	\includegraphics{<+file+>}
%	\caption{<+caption text+>}
	\label{fig:Broad_Spectrum-Analysis}
\end{figure}
}
