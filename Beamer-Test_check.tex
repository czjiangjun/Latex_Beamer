%%%%%%%%%%%%%%%%%%%%%%%%%%%%%%%%%%%%%%%%%%  不使用 authblk 包制作标题  %%%%%%%%%%%%%%%%%%%%%%%%%%%%%%%%%%%%%%%%%%%%%%
%-------------------------------PPT Title-------------------------------------
\title{第一原理计算软件简介与应用}
%-----------------------------------------------------------------------------

%----------------------------Author & Date------------------------------------
%\author[\textrm{Jun\_Jiang}]{姜\;\;骏\inst{}} %[]{} (optional, use only with lots of authors)
%% - Give the names in the same order as the appear in the paper.
%% - Use the \inst{?} command only if the authors have different
%%   affiliation.
\institute[BCC]{\inst{}%
%\institute[Gain~Strong]{\inst{}%
\vskip -20pt 北京市计算中心~云平台事业部~姜骏}
%\vskip -20pt {\large 格致斯创~科技}}
\date[\today] % (optional, should be abbreviation of conference name)
{	{\fontsize{6.2pt}{4.2pt}\selectfont{\textcolor{blue}{E-mail:~}\url{jiangjun@bcc.ac.cn}}}
\vskip 45 pt {\fontsize{8.2pt}{6.2pt}\selectfont{北京科技大学~理化楼-308% 报告地点
	\vskip 5 pt \textrm{2023.07.12}}}
}

%% - Either use conference name or its abbreviation
%% - Not really information to the audience, more for people (including
%%   yourself) who are reading the slides onlin%%   yourself) who are reading the slides onlin%%   yourself) who are reading the slides onlineee
%%%%%%%%%%%%%%%%%%%%%%%%%%%%%%%%%%%%%%%%%%%%%%%%%%%%%%%%%%%%%%%%%%%%%%%%%%%%%%%%%%%%%%%%%%%%%%%%%%%%%%%%%%%%%%%%%%%%%

\subject{}
% This is only inserted into the PDF information catalog. Can be left
% out.
%\maketitle
\frame
{
%	\frametitle{\fontsize{9.5pt}{5.2pt}\selectfont{\textcolor{orange}{“高通量并发式材料计算算法与软件”年度检查}}}
\titlepage
}
%-----------------------------------------------------------------------------

%------------------------------------------------------------------------------列出全文 outline ---------------------------------------------------------------------------------
\section*{}
\frame[allowframebreaks]
{
  \frametitle{Outline}
%  \frametitle{\textcolor{mycolor}{\secname}}
  \tableofcontents%[current,currentsection,currentsubsection]
}
%%在每个section之前列出全部Outline
%%类似的在每个subsection之前列出全部Outline是\AtBeginSubsection[]
%\AtBeginSection[]
%{
%  \frame<handout:0>%[allowframebreaks]
%  {
%    \frametitle{Outline}
%%全部Outline中,本部分加亮
%    \tableofcontents[current,currentsection]
%  }
%}

%-----------------------------------------------PPT main Body------------------------------------------------------------------------------------
\small
\frame
{
	\frametitle{现有高通量计算平台概览}
\begin{table}[!h]
\tabcolsep 0pt \vspace*{-12pt}
%\caption{}
\label{Table-Cost}
\begin{minipage}{0.85\textwidth}
%\begin{center}
\centering
\def\temptablewidth{1.1\textwidth}
\renewcommand\arraystretch{0.8} %表格宽度控制(普通表格宽度的两倍)
\rule{\temptablewidth}{1pt}
\begin{tabular*} {\temptablewidth}{@{\extracolsep{\fill}}c@{\extracolsep{\fill}}c@{\extracolsep{\fill}}c@{\extracolsep{\fill}}c@{\extracolsep{\fill}}c@{\extracolsep{\fill}}c@{\extracolsep{\fill}}c}
%-------------------------------------------------------------------------------------------------------------------------
	&\multirow{2}{*}{\fontsize{7.2pt}{5.2pt}\selectfont{编程语言}}	&\fontsize{7.2pt}{5.2pt}\selectfont{建模} &\multicolumn{2}{|c|}{\fontsize{6.2pt}{5.2pt}\selectfont{任务提交与管理}} &\multirow{2}{*}{\fontsize{7.2pt}{5.2pt}\selectfont{后处理}} &\multirow{2}{*}{\fontsize{6.2pt}{5.2pt}\selectfont{数据组织管理}} \\\cline{4-5}
	&	&\fontsize{7.2pt}{5.2pt}\selectfont{功能} &\multicolumn{1}{|l}{\fontsize{7.2pt}{5.2pt}\selectfont{软件接口}} &\multicolumn{1}{r|}{\fontsize{7.2pt}{5.2pt}\selectfont{运行容错}} & & \\\hline
	\fontsize{7.2pt}{5.2pt}\selectfont{{AFLOW}} &\fontsize{7.2pt}{5.2pt}\selectfont{C++} &\checkmark &$\triangle$ &\FiveStarOpen &\FiveStarOpen &\fontsize{7.2pt}{5.2pt}\selectfont{{Django}} \\
	\fontsize{7.2pt}{5.2pt}\selectfont{{MP}} &\fontsize{7.2pt}{5.2pt}\selectfont{Python} &\checkmark &\checkmark &\FiveStarOpen &\FiveStarOpen &\fontsize{7.2pt}{5.2pt}\selectfont{{MongoDB}} \\
	\multirow{2}{*}{\fontsize{7.2pt}{5.2pt}\selectfont{{QMIP}}} &\fontsize{7.2pt}{5.2pt}\selectfont{JavaScript/SVG} &\multirow{2}{*}{\checkmark} &\multirow{2}{*}{\checkmark} &\multirow{2}{*}{--} &\multirow{2}{*}{\checkmark} &\multirow{2}{*}{--} \\
	&\fontsize{7.2pt}{5.2pt}\selectfont{+html/Python} & & & & & \\
	\fontsize{7.2pt}{5.2pt}\selectfont{{CEP}} &\fontsize{7.2pt}{5.2pt}\selectfont{Python} &\checkmark &\checkmark &-- &\checkmark &\fontsize{7.2pt}{5.2pt}\selectfont{{Django/MySQL}} \\
	\fontsize{7.2pt}{5.2pt}\selectfont{{ASE}} &\fontsize{7.2pt}{5.2pt}\selectfont{Python} &\FiveStarOpen &\FiveStarOpen &-- &$\triangle$ &-- \\
	\multirow{2}{*}{\fontsize{7.2pt}{5.2pt}\selectfont{{MatCloud}}} &\fontsize{7.2pt}{5.2pt}\selectfont{JavaScript} &\multirow{2}{*}{\checkmark} &\multirow{2}{*}{$\triangle$} &\multirow{2}{*}{\checkmark} &\multirow{2}{*}{\checkmark} &\multirow{2}{*}{\fontsize{7.2pt}{5.2pt}\selectfont{{MongoDB}}} \\
	&\fontsize{7.2pt}{5.2pt}\selectfont{+.NETCore} & & & & &
\end{tabular*}
\rule{\temptablewidth}{1pt}
\end{minipage}
%\vskip -15pt
\fontsize{7.2pt}{5.2pt}\selectfont{
\begin{description}
	\item[\FiveStarOpen]~该功能较突出
	\item[\checkmark]~该功能基本满足需求
	\item[$\triangle$]~该功能存在不足
\end{description}}
%\end{center}
\end{table}
\fontsize{8.2pt}{6.2pt}\selectfont{
	\textrm{Lin L. \textit{Materials databases infrastructure constructed by first principles calculations:~a review.} \textbf{Mater. Perform. Character.}, 2015, 4(1):148.}}
}

%\begin{frame}
%	\frametitle{}	
%\begin{animateinline}[loop]{20}%
%  \multiframe{360}{rx=0+5}{
%\begin{tikzpicture}[scale=1.5]
%\pgfmathsetmacro{\x}{4+1*sin(\rx)}	
%\coordinate  (P) at (\x,0);	
%\draw [decorate,decoration={coil,segment length= \x pt, pre length=2mm, post length=2mm
%,amplitude=2mm}] (0,0)--(P);
%\shade[ball color=black](P)circle (2mm);
%\fill [pattern = north east lines] (0,0.-0.2) -- ++ (0,0.4) -- ++ (-0.1,0) -- ++(0,-0.4) --cycle
%      (0,-0.2) -- ++(0,-0.1) -- ++ (5.5,0) -- ++ (0,0.1) -- cycle;
%\draw (0,0.2) -- (0,-0.2) -- (5.5,-0.2);
%\end{tikzpicture} 	    }
%\end{animateinline}
%\end{frame}

%\begin{frame}
%	\frametitle{}
%\centering
%\begin{animateinline}[loop]{20}%
%  \multiframe{60  }{rt=-120+1}{
%\begin{tikzpicture}[scale=1.5]
%\filldraw [densely dashed,draw=black,fill=blue!20](0,0) -- (0,-1.5)
%(0,0) -- (-120:1.5) (0,0) -- (-60:1.5) arc (-60:-120:1.5);
%\draw (-2,-2) rectangle (2,1);
%\pgfmathsetmacro{\x}{1.5*cos(\rt)}
%\pgfmathsetmacro{\y}{1.5*sin(\rt)}
%\coordinate  (P) at (\x,\y);
%\draw[blue] (0,0)--(P);
%\shade[ball color=black](P)circle (2pt);
%\fill [pattern = north east lines] (-1,0) -- (1,0) -- (1,0.1) -- (-1,0.1) -- cycle;
%\draw (-1,0) -- (1,0);
%\end{tikzpicture} 	    }
%\newframe
%\multiframe{60  }{rt=-60+-1}{
%\begin{tikzpicture}[scale=1.5]
%\filldraw [densely dashed,draw=black,fill=blue!20](0,0) -- (0,-1.5)
%(0,0) -- (-120:1.5) (0,0) -- (-60:1.5) arc (-60:-120:1.5);
%\draw (-2,-2) rectangle (2,1);
%\pgfmathsetmacro{\x}{1.5*cos(\rt)}
%\pgfmathsetmacro{\y}{1.5*sin(\rt)}
%\coordinate  (P) at (\x,\y);
%\draw (0,0)--(P);
%\shade[ball color=black](P)circle (2pt);
%\fill [pattern = north east lines] (-1,0) -- (1,0) -- (1,0.1) -- (-1,0.1) -- cycle;
%\draw (-1,0) -- (1,0);
%\end{tikzpicture} 	    }
%\end{animateinline}
%\end{frame}
%
%\begin{frame}
%	\frametitle{}
%\begin{animateinline}[loop]{20}%
%  \multiframe{72  }{rt=0+-5}{
%\begin{tikzpicture}[scale=1.5]
%\draw (-2,-2) rectangle (2,1);
%\pgfmathsetmacro{\x}{0.75*cos(\rt)}
%\pgfmathsetmacro{\y}{-1.3+0.3*sin(\rt)}
%\coordinate  (P) at (\x,\y);
%\draw [blue] (0,0)--(P);
%\shade[ball color=black] (P) circle (2pt);
%\fill [pattern = north east lines] (-1,0) -- (1,0) -- (1,0.1) -- (-1,0.1) -- cycle;
%\draw (-1,0) -- (1,0);
%\draw[red,densely dashed] (0,-1.3)ellipse (0.75 and 0.3);
%\fill[opacity=0.1,red] (0,0) -- (0.75,-1.3)
%arc (0:\rt:0.75 and 0.3) -- cycle;
%\draw[-Stealth] (-0.5,-0.5) -- ++ (0,-0.3)node[right]{$g$};
%\draw [densely dashed](0,0) -- (0,-1.3) -- node[above]{$r$}
%(P) node[above right]{$m$};
%\end{tikzpicture} 	    }
%\end{animateinline}
%\end{frame}

%\begin{frame}
%	\frametitle{}
%	\begin{animateinline}[loop]{12}%
%  \multiframe{72}{ix = 0 + 5}{
%   \begin{tikzpicture}
%    \draw [line width = 0pt] (-3,-2) rectangle (2,2);
%    \begin{scope}[xshift=-0.5cm]
%      \draw[blue] (0,0) circle (0.5);
%      \draw[green] (\ix:1) circle (0.5);
%    \end{scope}
%    \node at (0.5,1.7) {$r=1-\cos\theta$};
%    \draw[domain=0:\ix,samples=500,red] plot ({(1-cos(\x))*cos(\x)},{(1-cos(\x))*sin(\x)});
%  \end{tikzpicture}
%  }
%\end{animateinline}
%\end{frame}

%\begin{frame}
%	\frametitle{}	
%\begin{animateinline}[loop]{10}%
%  \multiframe{60}{rx = 0 + pi/30}{
%  \begin{tikzpicture}
%    \draw [->] (-1,0) -- (0,0) node[below left]{$O$}
%               -- (2*pi,0) node[below]{$2\pi$}
%               -- (7,0) node[below]{$x$};
%    \draw [->] (0,-2) -- (0,2.4) node[left]{$y$};
%    \node at (3,-1) {$\begin{cases}
%      x = t - \sin t \\
%      y = 1 - \cos t
%    \end{cases}$};
%    \draw [blue] (\rx,1) circle (1);
%    \draw [domain=0:\rx,red] plot({\x-sin(\x r)},{1-cos(\x r)});
%  \end{tikzpicture}
%  }
%\end{animateinline}
%\end{frame}

%\begin{frame}
%	\frametitle{}
%	\begin{animateinline}[loop]{10}%
%  \multiframe{60}{rx = 0 + 6}{
%  \begin{tikzpicture}
%    \draw [->] (-5,0) -- (0,0) node[below left]{$O$}
%               -- (5,0)  node[below]{$x$};
%    \draw [->] (0,-5) -- (0,5) node[left]{$y$};
%    \draw (0,0) circle (4) ;
%    \draw [blue] (\rx:3) circle (1);
%    \node at (4,4) {$\begin{cases}
%      x = a\cos ^3t\\
%      y = a\sin ^3 t
%    \end{cases}$};
%    \draw [domain=0:\rx,red,samples=200] plot({4*(cos(\x))^3},{4*(sin(\x))^3});
%  \end{tikzpicture}
%  }
%\end{animateinline}
%\end{frame}

%\begin{frame}
%	\frametitle{}
%	\foreach \x in {0,6,...,354}{
%\begin{tikzpicture}[scale=1.3]
%\filldraw[fill=gray!20](0,0)circle(4.1cm);
%\filldraw[fill=white](0,0)circle(4cm);
%\foreach \x in {6,12,...,360}
%\draw(\x:4)--(\x:3.8);
%\foreach \x in {0,30,...,330}
%\draw[ultra thick](\x:4)--(\x:3.8);
%\node at (0:3.3){$[\ln\mathrm e^\uppi]$};
%\node at (30:3.3){$\sum_{i=0}^\infty\frac1{2^i}$};
%\node at (60:3.3){$\lim_{x\to0}\frac x{\sin x}$};
%\node at (90:3.3){$\mathrm C_{(16)}$};
%\node at (120:3.3){$\begin{vmatrix}
%3&1\\1&4
%\end{vmatrix}$};
%\node at (150:3.3){$\mathrm C_5^2$};
%\node at (180.7:3){$\frac1{(\arctan 2\sqrt2)'}$};
%\node at (210:3.3){$2^{\log_327}$};
%\node at (240:3.2){$\int_1^23x^2\,\mathrm dx$};
%\node at (270:3.3){$3!$};
%\node at (300:3.2){$\frac\uppi{\arccos\frac12}{+}\frac\uppi{\arcsin1}$};
%\node at (-30:3){$\min\left\{\left|4x+\frac1x\right|\right\}$};
%\draw (0,0)circle(0.15);
%\draw[line width=4.5pt,rotate=-\x/720](90:2)--(-90:0.7);
%\draw[line width=2.5pt,rotate=-\x/60](-90:3.3)--(90:0.65);
%\draw[line width=2.5pt,rotate=-\x](-90:1.1)--(-90:0.4);
%\draw[thick,rotate=-\x](-90:0.4)--(90:3.5);
%
%\end{tikzpicture}
%}
%\end{frame}

\begin{frame}
	\frametitle{}
	\begin{animateinline}[autoplay,loop,controls]{10} % 10 fps
\multiframe{20}{rPos=0.05+0.05}{ % 20帧,每帧移动0.05
\begin{tikzpicture}[scale=0.8]
% 绘制坐标轴
\draw[gray!30, thin] (-0.2,0) grid (4.2,2.2);
\draw[->] (-0.2,0) -- (4.5,0) node[right] {$x$};
\draw[->] (0,-0.2) -- (0,2.5) node[above] {$y$};
% 移动的方块
\filldraw[blue!70, opacity=0.8] (\rPos,1) rectangle (\rPos+1,2);
\node at (\rPos+0.5,1.5) {方块};
% 标记位置
\node[below] at (\rPos+0.5,1) {位置: \rPos};
\end{tikzpicture}
}
\end{animateinline}
\end{frame}

\begin{frame}
	\frametitle{总体技术路线}
	自研的虚拟机技术,重新封装硬件,通过一系列自研数据结构、算法与特定编程方法,改变硬件呈现的特性,\textcolor{red}{在不改动\textrm{VASP}代码基础上},构建面向高通量数据处理的高效运行环境
	\begin{figure}[ht]
		\centering
		\begin{minipage}[b]{0.4\textwidth}
%	\begin{animateinline}[loop]{1} % 20 fps
%		\multiframe{10}{rPOS = -2.0 + 0.2}{ % 72帧,每帧移动 0.05
			\begin{tikzpicture}[
			box-rect/.style={rectangle,draw=none,node distance=1cm,text width=4.5em,text centered,rounded corners,minimum height=0.1em,thin},
    straightline/.style={line width = 2pt, -},
    arrow/.style={draw,-latex', line width = 2pt},
		]
		\node<1-> [box-rect, text width=8.0em, minimum height=0.1em, thick](Single-Band) {\fontsize{5.5pt}{0.2pt}\selectfont{单能带\textrm{(Single-Band)计算}}};
%		\node [box-rect, fill=NavyBlue!50, text width=9.0em, draw=cyan, minimum height=1.5em, thick, dash dot dot](Critical-Virtual-Machine) at (\rPOS, -0.4) {\fontsize{5.5pt}{0.2pt}\selectfont{\textcolor{red}{无临界虚拟机}}};
		\node<2>[box-rect, fill=NavyBlue!50, text width=9.0em, draw=cyan, minimum height=1.5em, thick, dash dot dot](Critical-Virtual-Machine) at (0.0, -0.4) {\fontsize{5.5pt}{0.2pt}\selectfont{\textcolor{red}{无临界虚拟机}}};
		\node<1-> [box-rect, below = 0.5 of Single-Band, fill=purple!50, text width=0.5em, draw=cyan, minimum height=0.2em, thick](VASP-process-3) {\fontsize{5.5pt}{0.2pt}\selectfont{\textrm{V\\A\\S\\P\\}计算进程}};
		\node<1-> [box-rect, left = 0.5 of VASP-process-3, fill=purple!50, text width=0.5em, draw=cyan, minimum height=0.2em, thick](VASP-process-2) {\fontsize{5.5pt}{0.2pt}\selectfont{\textrm{V\\A\\S\\P\\}计算进程}};
		\node<1-> [box-rect, left = 0.5 of VASP-process-2, fill=purple!50, text width=0.5em, draw=cyan, minimum height=0.2em, thick](VASP-process-1) {\fontsize{5.5pt}{0.2pt}\selectfont{\textrm{V\\A\\S\\P\\}计算进程}};
		\node<1-> [box-rect, right = 0.5 of VASP-process-3, fill=purple!50, text width=0.5em, draw=cyan, minimum height=0.2em, thick](VASP-process-4) {\fontsize{5.5pt}{0.2pt}\selectfont{\textrm{V\\A\\S\\P\\}计算进程}};
		\node<1-> [box-rect, right = 0.5 of VASP-process-4, fill=purple!50, text width=0.5em, draw=cyan, minimum height=0.2em, thick](VASP-process-5) {\fontsize{5.5pt}{0.2pt}\selectfont{\textrm{V\\A\\S\\P\\}计算进程}};
		\node<1-> [box-rect, below = 0.2 of VASP-process-1, fill=violet!50, text width=0.5em, draw=cyan, minimum height=0.1em, thick](VASP-MPI-1) {\fontsize{5.5pt}{0.2pt}\selectfont{\textcolor{white}{\textrm{M\\P\\I\\}}}};
		\node<1-> [box-rect, below = 0.2 of VASP-process-2, fill=violet!50, text width=0.5em, draw=cyan, minimum height=0.1em, thick](VASP-MPI-2) {\fontsize{5.5pt}{0.2pt}\selectfont{\textcolor{white}{\textrm{M\\P\\I\\}}}};
		\node<1-> [box-rect, below = 0.2 of VASP-process-3, fill=violet!50, text width=0.5em, draw=cyan, minimum height=0.1em, thick](VASP-MPI-3) {\fontsize{5.5pt}{0.2pt}\selectfont{\textcolor{white}{\textrm{M\\P\\I\\}}}};
		\node<1-> [box-rect, below = 0.2 of VASP-process-4, fill=violet!50, text width=0.5em, draw=cyan, minimum height=0.1em, thick](VASP-MPI-4) {\fontsize{5.5pt}{0.2pt}\selectfont{\textcolor{white}{\textrm{M\\P\\I\\}}}};
		\node<1-> [box-rect, below = 0.2 of VASP-process-5, fill=violet!50, text width=0.5em, draw=cyan, minimum height=0.1em, thick](VASP-MPI-5) {\fontsize{5.5pt}{0.2pt}\selectfont{\textcolor{white}{\textrm{M\\P\\I\\}}}};
		\node<1-> [box-rect, below = 0.3 of VASP-MPI-3, fill=green!50, text width=8.0em, draw=cyan, minimum height=2.0em, thick](Critical-MPI) {\fontsize{5.5pt}{0.2pt}\selectfont{\textrm{MPI}临界区}};

		\path
		(Critical-MPI.north)++(-4.0em, 0.0em) coordinate (Critical-MPI-1)
		(Critical-MPI.north)++(-2.0em, 0.0em) coordinate (Critical-MPI-2)
		(Critical-MPI.north)++(0.0em, 0.0em) coordinate (Critical-MPI-3)
		(Critical-MPI.north)++(2.0em, 0.0em) coordinate (Critical-MPI-4)
		(Critical-MPI.north)++(4.0em, 0.0em) coordinate (Critical-MPI-5);

		\draw<1-> [straightline, black,very thick] (VASP-process-1.south) -- (VASP-MPI-1.north);
		\draw<1-> [straightline, black,very thick] (VASP-process-2.south) -- (VASP-MPI-2.north);
		\draw<1-> [straightline, black,very thick] (VASP-process-3.south) -- (VASP-MPI-3.north);
		\draw<1-> [straightline, black,very thick] (VASP-process-4.south) -- (VASP-MPI-4.north);
		\draw<1-> [straightline, black,very thick] (VASP-process-5.south) -- (VASP-MPI-5.north);
		\path<1-> [arrow, NavyBlue, thin] (VASP-MPI-1.south) -- (Critical-MPI-1);
		\path<1-> [arrow, NavyBlue, thin] (VASP-MPI-2.south) -- (Critical-MPI-2);
		\path<1-> [arrow, NavyBlue, thin] (VASP-MPI-3.south) -- (Critical-MPI-3);
		\path<1-> [arrow, NavyBlue, thin] (VASP-MPI-4.south) -- (Critical-MPI-4);
		\path<1-> [arrow, NavyBlue, thin] (VASP-MPI-5.south) -- (Critical-MPI-5);
			\end{tikzpicture}
%		}
%	\end{animateinline}
		\end{minipage}
		\label{fig:Single-Band-orig}
	\end{figure}
\end{frame}
