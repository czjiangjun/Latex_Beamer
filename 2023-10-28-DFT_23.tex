%%%%%%%%%%%%%%%%%%%%%%%%%%%%%%%%%%%%%%%%%%  不使用 authblk 包制作标题  %%%%%%%%%%%%%%%%%%%%%%%%%%%%%%%%%%%%%%%%%%%%%%
%-------------------------------PPT Title-------------------------------------
\title{23-选讲专题:~机器学习方法简介}
%-----------------------------------------------------------------------------
%----------------------------Author & Date------------------------------------

%\author[\textrm{Jun\_Jiang}]{姜\;\;骏\inst{}} %[]{} (optional, use only with lots of authors)
%% - Give the names in the same order as the appear in the paper.
%% - Use the \inst{?} command only if the authors have different
%%   affiliation.
%\institute[BCC]{\inst{}%
\institute[Gain~Strong]{\inst{}%
%\vskip -20pt 北京市计算中心}
\vskip -20pt {\large 格致斯创~科技}}
\date[\today] % (optional, should be abbreviation of conference name)
{%	{\fontsize{6.2pt}{4.2pt}\selectfont{\textcolor{blue}{E-mail:~}\url{jiangjun@bcc.ac.cn}}}
\vskip 45 pt {\fontsize{8.2pt}{6.2pt}\selectfont{%清华大学\;\;物理系% 报告地点
	\vskip 5 pt \textrm{2023.04.22}}}
}

%% - Either use conference name or its abbreviation
%% - Not really information to the audience, more for people (including
%%   yourself) who are reading the slides onlin%%   yourself) who are reading the slides onlin%%   yourself) who are reading the slides onlineee
%%%%%%%%%%%%%%%%%%%%%%%%%%%%%%%%%%%%%%%%%%%%%%%%%%%%%%%%%%%%%%%%%%%%%%%%%%%%%%%%%%%%%%%%%%%%%%%%%%%%%%%%%%%%%%%%%%%%%

\subject{}
% This is only inserted into the PDF information catalog. Can be left
% out.
%\maketitle
\frame
{
%	\frametitle{\fontsize{9.5pt}{5.2pt}\selectfont{\textcolor{orange}{“高通量并发式材料计算算法与软件”年度检查}}}
\titlepage
}
%-----------------------------------------------------------------------------

%------------------------------------------------------------------------------列出全文 outline ---------------------------------------------------------------------------------
%\section*{}
%\frame[allowframebreaks]
%{
%  \frametitle{Outline}
%%  \frametitle{\textcolor{mycolor}{\secname}}
%  \tableofcontents%[current,currentsection,currentsubsection]
%}
%%在每个section之前列出全部Outline
%%类似的在每个subsection之前列出全部Outline是\AtBeginSubsection[]
%\AtBeginSection[]
%{
%  \frame<handout:0>%[allowframebreaks]
%  {
%    \frametitle{Outline}
%%全部Outline中,本部分加亮
%    \tableofcontents[current,currentsection]
%  }
%}

%-----------------------------------------------PPT main Body------------------------------------------------------------------------------------
\small
%\section{\rm{VASP~}软件中\rm{PAW~}计算的实现}
%\frame
%
%	\frametitle{\textrm{VASP}计算的特色}
%	相比于与普通的第一原理计算软件,\textrm{VASP}很好地平衡了计算效率和精度的问题,总的来说,\textrm{VASP}主要通过这几个特色保证了计算的高效能
%	\begin{itemize}
%	     \item 迭代与优化算法的多样性\\
%		     本质上电荷密度迭代 \textrm{\&\&} 体系总能量优化是相同的优化问题,采用了类似的算法\upcite{CMS6-15_1996,PRB54-11169_1996}:\\
%			\textcolor{blue}{\textrm{Pseudo-Newton、Conjugate-Gradient、Broyden~mix、damping-factor、RMM-DIIS}}
%	     \item 尽可能采用局域基(原子轨道基)函数:~\\
%		     \textcolor{blue}{\textrm{LREAL}}=\textcolor{red}{\textrm{.TRUE.}}\\
%			优化的投影函数也尽可能在实空间表示
%	     \item \textrm{PAW}原子数据集:\textcolor{blue}{优异的赝势}\upcite{PRB59-1758_1999}
%	\end{itemize}
%}
