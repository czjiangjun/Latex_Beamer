\documentclass[cjk,slidestop,compress,mathserif,blue]{beamer}
%\documentclass[cjk,slidestop,handout,compress,mathserif,blue]{beamer}	%打印PPT用,handout(讲义)可去掉过渡效果,如\pause引起的多页显示,为打印时节省纸张
%handout去掉过渡效果,如\pause引起的多页显示
%dvipdfm选项是关键,否则编译统统通不过
%beamer的颜色选项定义的是导航条和标题的颜色(即关键词structure的颜色)

%%%%%%%%%%%%%%%%仅限于XeTeX可使用的宏包%%%%%%%%%%%%%%%%%%%%%%%%%%%%
\usepackage{fontspec,xunicode,xltxtra,beamerthemesplit}
%\usepackage{beamerthemesplit}
\usepackage{xeCJK}
\setCJKmainfont[BoldFont=黑体, ItalicFont=楷体, BoldItalicFont=仿宋]{黑体}
%\setsansfont[Mapping=tex-text]{Adobe 黑体 Std}
%如果装了Adobe Acrobat,可在font.conf中配置Adobe字体的路径以使用其中文字体
%也可直接使用系统中的中文字体如SimSun,SimHei,微软雅黑 等
%原来beamer用的字体是sans family;注意Mapping的大小写,不能写错

%%%%%%%%   确定标题和导航条结构的框架     %%%%%%%%%%%%
\usepackage{beamerthemeshadow}                       %
%\usepackage{beamerthemeclassic}%导航条色与背景色一致%
%%%%%%%%%%%%%%%%%%%%%%%%%%%%%%%%%%%%%%%%%%%%%%%%%%%%%%
\setbeamerfont{roman title}{size={}}
%\usepackage{CJK} % CJK 中文支持                                  %
\usepackage{amsmath,amsthm,amsfonts,amssymb,bm}
\usepackage{mathrsfs}
\usepackage{xcolor}                                        %使用默认允许使用颜色
\usepackage{hyperref} 
\usepackage{graphicx}
\usepackage{subfigure}           %图片跨页
\usepackage{animate}		 %插入动画

\usepackage[dvipdfmx]{movie15_dvipdfmx} %插入视频
%\usepackage{handoutWithNotes}		%(讲义)在打印PPT的时候会留出给每一页做注释的部分
%\pgfpagesuselayout{1 on 1 with notes landscape}[a4paper,border shrink=5mm]

%\usepackage[numbers,sort&compress]{natbib} %紧密排列             %
\usepackage[sectionbib]{chapterbib}        %每章节单独参考文献   %
\usepackage{hypernat}                                                                         %
%\usepackage[dvipdfm,bookmarksopen=true,pdfstartview=FitH,CJKbookmarks]{hyperref}		%
\hypersetup{bookmarksnumbered,colorlinks,linkcolor=brown,citecolor=blue,urlcolor=red}         %
%参考文献含有超链接引用时需要下列宏包,注意与natbib有冲突        %
%\usepackage[dvipdfm]{hyperref}                                  %
%\usepackage{hypernat}                                           %
\newcommand{\upcite}[1]{\hspace{0ex}\textsuperscript{\cite{#1}}} %

%\useoutertheme{smoothbars}
\useinnertheme[shadow=true]{rounded}
\usetheme{Berkeley}                                          %主题式样
%\usetheme{Luebeck}

\usecolortheme{lily}                                        %颜色主题式样

\usefonttheme{professionalfonts}                           %字体主题样式宏包

%\beamertemplatetransparentcoveredhigh                      %使所有被隐藏的文本高度透明
\beamertemplatetransparentcovereddynamicmedium             %使所有被隐藏的文本完全透明,动态,动态的范围很小
\mode<presentation>
%\beamersetaveragebackground{gray}                          %设置背景颜色(单一色) 
\beamertemplateshadingbackground{green!10}{red!5}         %设置背景颜色(渐变色)

%i放置单位logo
%\logo{\includegraphics[width=1.6cm,height=0.35cm]{Figures/BCC_logo-1.png}}	%简单设置logo

%\pgfdeclareimage[width=3.5cm]{logoname}{Figures/BCC_logo-1.png}		%logo置于左侧微调
%\logo{\pgfuseimage{logoname}{\vspace{0.2cm}\hspace*{-2.0cm}}}

%在指定位置精确放置logo
\usepackage{tikz}
\usepackage{beamerfoils}
\usepackage{pgf}
%\logo{\pgfputat{\pgfxy(11.68,0.15)}{\includegraphics[height=1.0cm,viewport=0 0 140 120,clip]{Figures/BCC_logo-1.png}}\pgfputat{\pgfxy(10.508,-0.22)}{\includegraphics[height=0.369cm,viewport=140 0 540 120,clip]{Figures/BCC_logo-1.png}}}
%\MyLogo{
%	\pgfputat{\pgfxy(-50,-50)}{\pgfbox[right,base]{\includegraphics[height=1cm]{Figures/BCC_logo-1.png}}}

%logo作为背景放置
%\setbeamertemplate{background}{
%	\pgfputat{\pgfxy(6.5,-0.5)}{\pgfbox[left,top]{\pgfimage[height=1.1cm]{Figures/BCC_logo-1.png}}}}

%\logo{}									%不显示logo

\begin{document}
%\begin{CJK*}{GBK}{song}
%\begin{CJK*}{GBK}{kai}
%beamer下不能用\songyi、\zihao等命令!
%\graphicspath{Figures/}

%-------------------------------PPT Title-------------------------------------
\title{专家与京剧}
%-----------------------------------------------------------------------------

%----------------------------Author & Date------------------------------------
\author{姜骏}
\date{\textrm{2016.08.22}}
%\date{2013.09.10}
\frame{\titlepage}
%-----------------------------------------------------------------------------

%------------------------------------------------------------------------------列出全文 outline ---------------------------------------------------------------------------------
\section*{}
\frame[allowframebreaks]
{
  \frametitle{Outline}
%  \frametitle{\textcolor{mycolor}{\secname}}
  \tableofcontents%[current,currentsection,currentsubsection]
}
%在每个section之前列出全部Outline
%类似的在每个subsection之前列出全部Outline是\AtBeginSubsection[]
\AtBeginSection[]
{
  \frame<handout:0> %讲义(handout)不显示 <handout:0> 讲义显示 <handout:1> / %beamer不显示 <beamer:0> beamer显示 <beamer:1>	
  {
    \frametitle{Outline}
%全部Outline中,本部分加亮
    \tableofcontents[current,currentsection]
  }
}

%------------------------------------------------------------------------------PPT main Body------------------------------------------------------------------------------------
\small
\section{“三大贤”及其代表作}
\frame
{
	\frametitle{朱家溍先生}
\begin{figure}[h!]
\centering
\vspace{-0.2in}
\includegraphics[height=0.64\textwidth,width=0.46\textwidth,viewport=0 0 500 720,clip]{Figures/Zhu_Jiajin.jpg}
\hskip 5pt
\includegraphics[height=0.64\textwidth,width=0.46\textwidth,viewport=0 0 400 610,clip]{Figures/Zhu_Tuishilu.jpg}
\caption{朱家溍先生(1914.8-2003.9.29)}
\label{Zhu_Jiajin}
\end{figure}
}

\frame
{
	\frametitle{刘曾复先生}
\begin{figure}[h!]
\centering
\vspace{-0.2in}
\includegraphics[height=0.64\textwidth,width=0.46\textwidth,viewport=0 0 360 520,clip]{Figures/Liu_Zengfu.jpg}
\hskip 5pt
\includegraphics[height=0.62\textwidth,width=0.45\textwidth,viewport=100 85 660 875,clip]{Figures/Liu_Xinxu.jpg}
\caption{刘曾复先生(1914.11.9-2012.6.27)}
\label{Liu_Zengfu}
\end{figure}
}

\frame
{
	\frametitle{吴小如先生}
\begin{figure}[h!]
\centering
\vspace{-0.2in}
\includegraphics[height=0.64\textwidth,width=0.46\textwidth,viewport=0 0 360 520,clip]{Figures/Wu_Xiaoru.jpg}
\hskip 5pt
\includegraphics[height=0.64\textwidth,width=0.46\textwidth,viewport=39 5 200 235,clip]{Figures/Wu_Wenlu.jpg}
\caption{吴小如先生(1922.9.7-2014.5.11)}
\label{Wu_Xiaoru}
\end{figure}
}

\frame
{
	\frametitle{}
\begin{figure}[h!]
\centering
\vspace{-10.5pt}
\includegraphics[height=0.60\textwidth,width=0.9\textwidth,viewport=0 0 510 350,clip]{Figures/Zhu-Liu.jpg}
\caption{\textrm{朱家溍先生与刘曾复先生合影}}
\label{Collect_Wang}
\end{figure}
}

\frame
{
	\frametitle{}
\begin{figure}[h!]
\centering
%\vspace{-10.5pt}
\includegraphics[height=0.60\textwidth,width=1.0\textwidth,viewport=0 0 500 300,clip]{Figures/Collect_Zhu-Liu-Wu-Wang.jpg}
\caption{\fontsize{7.3pt}{3.9pt}\selectfont{\textrm{左起:~刘曾先生、朱家溍先生、吴小如先生、王金璐先生(1919-2016.6.1)等合影}}}
\label{Collect_Wang}
\end{figure}
}

\section{吴家与京昆}
\frame
{
	\frametitle{吴氏弟兄}
\begin{figure}[h!]
\centering
\vspace{-10.5pt}
\includegraphics[height=0.52\textwidth,width=0.9\textwidth,viewport=0 0 225 140,clip]{Figures/Collect_Wus-1.jpg}
\caption{合影-吴征铠教授(1913-2007)、吴白匋(吴征铸)教授(1906-1992)、吴征鉴研究员(1909-1982)、吴征镒研究员(1916-2013)(从左到右)}
%\caption{合影-\fbox{吴征铠}教授、\fbox{吴白匋}(吴征铸)教授、\fbox{吴征鉴}研究员、\fbox{吴征镒}研究员(从左到右)}
%\caption{合影-\frame{吴征铠}教授、\frame{吴白匋}(吴征铸)教授、\frame{吴征鉴}研究员、\frame{吴征镒}研究员(从左到右)}
\label{Collect_Wu_s-1}
\end{figure}
}

\frame
{
	\frametitle{吴门三院士}
\begin{figure}[h!]
\centering
\vspace{-10.5pt}
\includegraphics[height=0.57\textwidth,width=0.9\textwidth,viewport=0 0 755 500,clip]{Figures/Collect_Wus-2.jpg}
\caption{合影-吴征铠教授、吴征鉴研究员、吴征镒研究员}
%\caption{合影-\fbox{吴征铠}教授、\fbox{吴征鉴}研究员、\fbox{吴征镒}研究员(从左到右)}
%\caption{合影-\frame{吴征铠}教授、\frame{吴征鉴}研究员、\frame{吴征镒}研究员(从左到右)}
\label{Collect_Wu_s-2}
\end{figure}
}

\frame
{
	\frametitle{}
\begin{figure}[h!]
\centering
\vspace{-10.5pt}
\includegraphics[height=0.52\textwidth,width=0.9\textwidth,viewport=0 0 435 250,clip]{Figures/1994_6_5.jpg}
\caption{1994年6月5日唐敖庆教授(1915-2008)、吴征铠教授、卢嘉锡教授(1915-2001)、徐光宪教授(1920-2015)和高小霞教授(1919-1998)(从左到右)在第七次院士大会上}
%\caption{1994年6月5日\fbox{唐敖庆}教授、\fbox{吴征铠}教授、\fbox{卢嘉锡}教授、\fbox{徐光宪}教授和\fbox{高小霞}教授(从左到右)在第七次院士大会上}
%\caption{1994年6月5日\frame{唐敖庆}教授、\frame{吴征铠}教授、\frame{卢嘉锡}教授、\frame{徐光宪}教授和\frame{高小霞}教授(从左到右)在第七次院士大会上}
\label{Tang_Wu_Lu_Xu}
\end{figure}
}

\frame
{
	\frametitle{吴征镒院士致刘曾复先生函}
\begin{figure}[h!]
\centering
\vspace{-0.2in}
\includegraphics[height=0.40\textwidth,width=0.60\textwidth,viewport=100 100 1080 660,clip]{Figures/Wu_to_Liu_Letter-1.jpg}\\
\includegraphics[height=0.33\textwidth,width=0.62\textwidth,viewport=210 200 1100 650,clip]{Figures/Wu_to_Liu_Letter-2.jpg}
\caption{吴征镒院士致刘曾复先生函}
\label{Wu_to_Liu-1}
\end{figure}
}

\frame
{
	\frametitle{吴征镒院士致刘曾复先生函}
\begin{figure}[h!]
\centering
\vspace{-0.2in}
\includegraphics[height=0.62\textwidth,width=0.9\textwidth,viewport=90 210 1050 850,clip]{Figures/Wu_to_Liu_Letter-3.jpg}
\caption{吴征镒院士致刘曾复先生函}
\label{Wu_to_Liu-2}
\end{figure}
}

%\section{刘曾复先生物理系的同学}
%\frame
%{
%	\frametitle{}
%\begin{figure}[h!]
%\centering
%\vspace{-10.5pt}
%\includegraphics[height=0.52\textwidth,width=0.9\textwidth,viewport=0 0 920 500,clip]{Figures/Collect_Phys_1936.jpeg}
%\caption{\textrm{1936}年清华大学物理系第八级毕业生留念:\\{\fontsize{9.3pt}{3.9pt}\selectfont\textrm{前排左起:~王大珩、戴中扆(黄葳)、许孝慰、何泽慧、郁钟正(于光远)\\后排左起:~钱三强、杨镇邦、陈亚伦、杨龙生、谢毓章}}}
%\label{Tsinghua_Phys_1936}
%\end{figure}
%}

%%%%%%%%%%%%%%%%%%%%%%%%%%%%%%%%%  插入音频/视频,使用url 要求视频在指定目录下 %%%%%%%%%%%%%%%%%%%%%%%%%%%%%%%%
%%%%%%%%%%%%%%%%%%%%%%%%%%%%%%%%%%%%%%%%%%%%%%%%%%%%%%%%%%%%%%%%%%%%%%%%%%%%%%%%%%%%%%%%%%%%%%%%%%%%%%%%%%%%%%%
\frame													      %
{													      %
	\frametitle{京剧名宿遗音}									      %
\begin{figure}[ht]											      %
%	\includemovie[poster, controls, mouse, url] {0.8\textwidth}{0.6\textwidth}{traffic.avi}		      %
%%	\includemovie[poster, controls, mouse, url] {0.8\textwidth}{0.6\textwidth}{Yuan_Kuocheng.mp4}	      %
	\includemovie[poster, controls, mouse, url] {0.8\textwidth}{0.2\textwidth}{Figures/Liu-Xiongzhouguan.mp3}     %
	\includemovie[poster, controls, mouse, url] {0.8\textwidth}{0.2\textwidth}{Figures/Zhu_Liu-Luomahu.mp3}	      %
%	\includemovie[poster, controls, mouse, url] {0.8\textwidth}{0.2\textwidth}{Figures/Liu-Pantaohui.mp3}	      %
	\includemovie[poster, controls, mouse, url] {0.8\textwidth}{0.2\textwidth}{Figures/Wu-Pantaohui.mp3}	      %
\caption{京剧名宿遗留音}											      %
\end{figure}												      %
}													      %
%%%%%%%%%%%%%%%%%%%%%%%%%%%%%%%%%%%%%%%%%%%%%%%%%%%%%%%%%%%%%%%%%%%%%%%%%%%%%%%%%%%%%%%%%%%%%%%%%%%%%%%%%%%%%%%

%------------------------------------------------------------------------Reference----------------------------------------------------------------------------------------------
%\begin{thebibliography}{99}
%-----------------------------------------------------------------------------------------------------------------------------------------------------------------------%
%\frame
%{
%\frametitle{主要参考文献}
%{\small
%\bibitem{Singh_Book}\textrm{D. J. Singh. \textit{Plane Wave, PseudoPotential and the LAPW method} (Kluwer Academic, Boston,USA, 1994)}					%
%  \nocite{*}																				%
%}
%}
%\end{thebibliography}
\begin{thebibliography}{99}
\frame
{
\frametitle{主要参考文献}
{\small
	\bibitem{Zhu_Tuishilu}朱家溍 著, {\textit{故宫退食录}}\;\textrm{({\textit{上、下}})}\:北京出版社, 北京, 1999\\
朱家溍 著, {\textit{故宫退食录}}\;\textrm{({\textit{上、下}})}\:紫禁城出版社, 北京, 2009
	\bibitem{Liu_Xinxu}刘曾复 编著、屠楚材 记谱, {\textit{京剧新序}}\:燕山出版社, 北京, 1999\\
{\fontsize{7.0pt}{3.9pt}\selectfont 刘曾复 编著、屠楚材 记谱,娄悦、何毅 整理, {\textit{京剧新序}}\;\textrm{(修订版)}\:学苑出版社, 北京, 2009}\\
刘曾复 传述, {\textit{刘曾复说戏剧本集}}\:华东师范大学出版社, 上海, 2015
	\bibitem{Wu_Wenlu}吴小如 著, {\textit{吴小如戏曲文录}}\:北京大学出版社, 北京, 1995 \\
	吴小如 著, {\textit{吴小如戏曲随笔集补编}}\:天津古籍出版社, 天津, 2006
	\bibitem{XQYS1-32_1983}\textrm{刘曾复、王世续、王金彦, 京剧老生把子见闻录\:\textit{戏曲艺术}, \textbf{第一期} (1983), 32}
	\bibitem{ZGXJ1-57_1993}\textrm{刘曾复, 京剧书文指伪录\:\textit{中国戏剧}, \textbf{第01期} (1993), 57}
}
\nocite*{}
}
\end{thebibliography}
%{\small
%\phantomsection\addcontentsline{toc}{section}{Bibliography}	 %直接调用\addcontentsline命令可能导致超链指向不准确,一般需要在之前调用一次\phantomsection命令加以修正	%
%\bibliography{Myref}																			%
%\bibliographystyle{mybib}																		%
%  \nocite{*}																				%
%}
%-----------------------------------------------------------------------------------------------------------------------------------------------------------------------%


%-----------------------------------------------------------Beamer下不建议使用bib,因为涉及分页--------------------------------------------------------------------------%
%{\small
%\phantomsection\addcontentsline{toc}{section}{Bibliography}	 %直接调用\addcontentsline命令可能导致超链指向不准确,一般需要在之前调用一次\phantomsection命令加以修正	%
%\bibliography{Myref}																			%
%\bibliographystyle{mybib}																		%
%  \nocite{*}																				%
%}

%------------------------------------------------------------------------------------------------------------------------------------------------------------------------------%

%-------------------------------------------------------------------------Thanks------------------------------------------------------------------------------------------------
%\section{致谢}
%\frame
%{
%\frametitle{致$\quad$谢}
%\begin{itemize}
%    \setlength{\itemsep}{20pt}
%  \item 感谢本团队高兴誉、吴泉生、宋红州等各位老师参与的讨论
%  \item 感谢莫所长、宋主任以及软件中心各位老师和同事
%  \item 感谢王崇愚先生的帮助
%\end{itemize}
%}
\frame
{
\vskip 60 pt
%\hskip 10pt \textcolor{blue}{\Huge 感谢答辩委员会各位老师\,\textrm{!}}\\
\vskip 35 pt
\hskip 60pt \textcolor{blue}{\Huge 谢谢大家\:!}
%\vskip 15 pt
%\hskip 40pt \textcolor{blue}{\Huge \textrm{for your attention\:!}}
}

%-------------------------------------------------------------------------------------------------------------------------------------------------------------------------------

\clearpage
%\end{CJK*}
\end{document}
