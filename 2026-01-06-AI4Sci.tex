%%%%%%%%%%%%%%%%%%%%%%%%%%%%%%%%%%%%%%%%%%  不使用 authblk 包制作标题  %%%%%%%%%%%%%%%%%%%%%%%%%%%%%%%%%%%%%%%%%%%%%%
%-------------------------------PPT Title-------------------------------------
\title{\textrm{AI for Science:~}数据驱动视野下的材料设计}
%-----------------------------------------------------------------------------

%----------------------------Author & Date------------------------------------
%\author[]{\vskip +10pt 姜\;\;骏\inst{}} %[]{} (optional, use only with lots of authors)
%% - Give the names in the same order as the appear in the paper.
%% - Use the \inst{?} command only if the authors have different
%%   affiliation.
\institute[BCC]{\inst{}%
%\institute[Gain~Strong]{\inst{}%
\vskip -15pt 北京市计算中心有限公司~材料计算团队}
%\vskip -20pt {\large 格致斯创~科技}}
\date[\today] % (optional, should be abbreviation of conference name)
{	%{\fontsize{6.2pt}{4.2pt}\selectfont{\textcolor{blue}{E-mail:~}\url{jiangjun@bcc.ac.cn}}}
\vskip 45 pt {\fontsize{8.2pt}{6.2pt}\selectfont{%清华大学\;\;物理系% 报告地点
	\vskip 5 pt \textrm{2026.01.06}}}
}

%% - Either use conference name or its abbreviation
%% - Not really information to the audience, more for people (including
%%   yourself) who are reading the slides onlin%%   yourself) who are reading the slides onlin%%   yourself) who are reading the slides onlineee
%%%%%%%%%%%%%%%%%%%%%%%%%%%%%%%%%%%%%%%%%%%%%%%%%%%%%%%%%%%%%%%%%%%%%%%%%%%%%%%%%%%%%%%%%%%%%%%%%%%%%%%%%%%%%%%%%%%%%

\subject{}
% This is only inserted into the PDF information catalog. Can be left
% out.
%\maketitle
\frame
{
%	\frametitle{\fontsize{9.5pt}{5.2pt}\selectfont{\textcolor{orange}{“高通量并发式材料计算算法与软件”年度检查}}}
\titlepage
}
%-----------------------------------------------------------------------------

%------------------------------------------------------------------------------列出全文 outline ---------------------------------------------------------------------------------
\section*{}
\frame[allowframebreaks]
{
	\frametitle{\textrm{Outline}}
%  \frametitle{\textcolor{mycolor}{\secname}}
  \tableofcontents%[current,currentsection,currentsubsection]
}
%在每个section之前列出全部Outline
%类似的在每个subsection之前列出全部Outline是\AtBeginSubsection[]
%\AtBeginSection[]
%{
%  \frame<handout:0>%[allowframebreaks]
%  {
%    \frametitle{Outline}
%%全部Outline中,本部分加亮
%    \tableofcontents[current,currentsection]
%  }
%}

%-----------------------------------------------PPT main Body------------------------------------------------------------------------------------
\small
\section*{团队简介}
\frame
{
	\frametitle{团队概况}
	计算材料团队成立于\textrm{2016}年
	\vskip 2pt 
	自\textrm{2017}年起,团队承担国家重点研发计划\textcolor{magenta}{``材料基因工程关键技术与支撑平台''}项目相关研究任务,包括
\begin{itemize}
		\item \textcolor{blue}{高通量并发式材料计算算法和软件}
		\item \textcolor{blue}{高通量材料计算的工作流设计与交互图形化}
		\item \textcolor{blue}{材料基因工程数据汇交与管理服务技术平台}
	\end{itemize}
	在此基础上,经过数年探索,团队确立了\textcolor{purple}{以人工智能和数据为主要驱动力,面向基础科学中的应用研究问题}的指导思想,主要围绕
	{\fontsize{7.2pt}{6.2pt}\selectfont{
	\begin{itemize}
		\item 微观尺度计算材料平台建设、第一原理计算方法与算法研究
		\item 结合机器学习方法和人工智能,对金属氧化物、半导体和金属材料的电子结构和热力学性质进行理论计算,形成材料数据库/
		\item \textrm{AI for Scienece}:~化学-化工知识图谱建设
		\item 金属催化\textrm{\ch{CO2}}还原的机理研究(\textcolor{blue}{新近开展研究})
	\end{itemize}}}
	等方向展开研究,并形成了一批研究成果和产品
}

\frame
{
	\frametitle{团队成员}
\begin{figure}[h!]
\vspace*{-0.10in}
\centering
\includegraphics[height=2.40in,width=4.00in,viewport=0 0 430 270,clip]{Figures/Team_Member.png}
%\caption{\tiny \textrm{Pseudopotential for metallic sodium, based on the empty core model and screened by the Thomas-Fermi dielectric function.}}%(与文献\cite{EPJB33-47_2003}图1对比)
\label{Team_Membwe}
\end{figure}
}

\section{材料基因工程简介}
\frame
{
	\frametitle{数据驱动的科学研究}
科学的新驱动力:~\textcolor{red}{密集数据}+\textcolor{red}{人工智能}
\begin{figure}[h!]
%\vspace*{-0.05in}
\centering
\includegraphics[height=2.30in,width=3.70in]{Figures/Four_Model_1.png}
%\caption{\tiny \textrm{Pseudopotential for metallic sodium, based on the empty core model and screened by the Thomas-Fermi dielectric function.}}%(与文献\cite{EPJB33-47_2003}图1对比)
\label{Four_Model_1}
\end{figure}
前所未有的计算能力和大规模的数据收集能力:\\推动现代科学进入``第四范式''
}

\frame
{
	\frametitle{数据科学推动研究的范式变更}
\begin{figure}[h!]
\vspace*{-0.28in}
\centering
\includegraphics[height=2.00in,width=4.15in]{Figures/Four_Model_3.png}
%\caption{\tiny \textrm{Pseudopotential for metallic sodium, based on the empty core model and screened by the Thomas-Fermi dielectric function.}}%(与文献\cite{EPJB33-47_2003}图1对比)
\label{Four_Model}
\end{figure}
\begin{minipage}[b]{0.48\textwidth}
 {\fontsize{7.5pt}{6.0pt}\selectfont\begin{itemize}%[+-| alert@+>]
	 \setlength{\itemsep}{10pt}
 \item 逐步趋于理性
 \item 逐步趋于复杂
 \end{itemize}}
\end{minipage}
\hfill
\begin{minipage}[b]{0.48\textwidth}
 {\fontsize{7.5pt}{6.0pt}\selectfont\begin{itemize}%[+-| alert@+>]
	 \setlength{\itemsep}{10pt}
 \item 逐步趋于抽象
 \item 逐步趋于深刻
 \end{itemize}}
\end{minipage}
}

\frame
{
	\frametitle{材料模拟的基本思想}
	\textrm{2016}年起,我国开始推动实施``材料基因工程计划''
\begin{minipage}[c]{0.36\textwidth}
\begin{itemize}%[+-| alert@+>]
\vspace*{-1.55in}
 {\fontsize{7.5pt}{6.0pt}\selectfont
	 \setlength{\itemsep}{10pt}
 \item 变革研发模式:\\计算-实验-理论-数据科学相融合,高效、低耗、按需设计
 \item 面向复杂材料的模拟:\\建设数据驱动的材料创新平台}
 \end{itemize}
\end{minipage}
\hfill
\begin{minipage}[b]{0.62\textwidth}
\begin{figure}[h!]
%\vspace*{-0.25in}
\centering
%\includegraphics[height=1.80in,width=2.75in]{Figures/Multi-Scale-6.png}
\includegraphics[height=1.65in,width=2.45in]{Figures/MGE.png}
%\caption{\tiny \textrm{Pseudopotential for metallic sodium, based on the empty core model and screened by the Thomas-Fermi dielectric function.}}%(与文献\cite{EPJB33-47_2003}图1对比)
\label{Multi-Scale}
\end{figure}
\end{minipage}
计算材料模拟:~从材料组分、结构出发,完成材料物性的模拟预测
\begin{figure}[h!]
\vspace*{-0.10in}
\centering
\includegraphics[height=0.80in,width=4.05in]{Figures/MGE-2.png}
%\caption{\tiny \textrm{Pseudopotential for metallic sodium, based on the empty core model and screened by the Thomas-Fermi dielectric function.}}%(与文献\cite{EPJB33-47_2003}图1对比)
\label{MGE}
\end{figure}
}

\begin{frame}
	\frametitle{材料基因工程推动新材料发展}
\begin{figure}[h!]
\vspace*{-0.25in}
\centering
\includegraphics[height=2.90in,width=4.80in,viewport=0 0 1250 710,clip]{Figures/MGE_idea.png}
%\caption{\tiny \textrm{Pseudopotential for metallic sodium, based on the empty core model and screened by the Thomas-Fermi dielectric function.}}%(与文献\cite{EPJB33-47_2003}图1对比)
\label{MGE_idea}
\end{figure}
\end{frame}

%\begin{frame}
%	\frametitle{理论、方法与软件}
%\begin{figure}[h!]
%\vspace*{-0.25in}
%\centering
%\includegraphics[height=2.80in,width=4.95in,viewport=5 3 1250 780,clip]{Figures/Method_Procedure.png}
%%\caption{\tiny \textrm{Pseudopotential for metallic sodium, based on the empty core model and screened by the Thomas-Fermi dielectric function.}}%(与文献\cite{EPJB33-47_2003}图1对比)
%\label{Method-Procedure}
%\end{figure}
%\end{frame}

\section{什么是人工智能}
\begin{frame}{人工智能的定义}
  \begin{itemize}
	  \item 人工智能\textrm{(AI)}:~让机器模拟人类智能的科学
	  \item \textrm{Turing}测试:~判断机器是否智能的标准~\textrm{(1950)}
	  \item 强~\textrm{AI} \textrm{vs.} 弱\textrm{AI}
		  \vskip 2pt
    \begin{itemize}
	    \item 强\textrm{AI}:~具有自主意识的机器(尚未实现)
	    \item 弱\textrm{AI}:~专用领域智能(当前主流)
    \end{itemize}
  \end{itemize}
\begin{figure}[h!]
\vspace*{-0.10in}
\centering
\includegraphics[height=1.55in, width=3.2in, viewport=0 0 2250 1230,clip]{Figures/The_Turing-Test.png}
%\caption{\tiny \textrm{Pseudopotential for metallic sodium, based on the empty core model and screened by the Thomas-Fermi dielectric function.}}%(与文献\cite{EPJB33-47_2003}图1对比)
\label{Turing_TEST}
\end{figure}
\end{frame}

%\section{发展历程}
\begin{frame}{历史里程碑}
%  \begin{columns}
 %   \column{0.5\textwidth}
%    \begin{itemize}
%	    \item \textrm{1956:~Dartmouth}会议
%	    \item \textrm{1997:~Deep-Blue}击败\textrm{Garry Kasparov}
 %     \item \textrm{2016:~Alpha~Go}战胜李世石
  %    \item \textrm{2022:~ChatGPT}发布
  %    \item \textrm{2025:~DeepSeek}发布
  %  \end{itemize}
%
%    \column{0.5\textwidth}
\begin{figure}[h!]
\vspace*{-0.10in}
\centering
   \includegraphics[width=\textwidth]{Figures/AI-History_timeline.jpg}
\label{AI-History}
\end{figure}
%  \end{columns}
\end{frame}

\begin{frame}{典型应用}
  \begin{itemize}
	  \item \textrm{AI}是通过机器学习模仿人类智能的技术
    \item 深度学习推动突破性进展
    \item 在多个领域产生革命性影响
%    \item 需要重视伦理规范和社会影响
  \end{itemize}

  \begin{columns}
    \column{0.33\textwidth}
    \centering
    \includegraphics[width=0.8\textwidth]{Figures/AI_Deeplearning-NLP.png}\\
    自然语言处理

    \column{0.33\textwidth}
    \centering
    \includegraphics[width=0.8\textwidth]{Figures/AI_Deeplearning-CV.png}\\
    计算机视觉

    \column{0.33\textwidth}
    \centering
    \includegraphics[width=0.8\textwidth]{Figures/AI_Robot-AI-machine-learning-hero.jpg}\\
    智能机器人
  \end{columns}

  \begin{columns}
    \column{0.33\textwidth}
    \centering
    \includegraphics[width=0.8\textwidth]{Figures/AI_for_Science-1.png}

    \column{0.33\textwidth}
    \centering
    \includegraphics[width=0.8\textwidth]{Figures/AI_for_Science-3.jpeg}\\
    \textcolor{red}{\textrm{AI for Science}}

    \column{0.33\textwidth}
    \centering
    \includegraphics[width=0.8\textwidth]{Figures/AI_for_Science-4.jpeg}
  \end{columns}
\end{frame}

\begin{frame}
    \frametitle{通用大模型}
    \textcolor{red}{通用大模型:}~{\fontsize{8.2pt}{6.2pt}\selectfont{如\textrm{GPT}、\textrm{LLaMA}、\textrm{DeepSeek}}}
\begin{figure}[h!]
\vspace*{-0.05in}
\centering
\includegraphics[height=2.0in, width=4.0in, viewport=0 0 1080 510,clip]{Figures/LLM_model-logo_Chinese.jpg}
%\caption{\tiny \textrm{Pseudopotential for metallic sodium, based on the empty core model and screened by the Thomas-Fermi dielectric function.}}%(与文献\cite{EPJB33-47_2003}图1对比)
\label{LLM_model-logo_Chinese}
\end{figure}
\textcolor{blue}{具备强大的通用知识和语言理解能力,预训练参数规模巨大}
\end{frame}

%\section{核心技术}
%\subsection{机器学习}
\begin{frame}{人工智能的底层:~机器学习}
  \begin{block}{机器学习的核心思想}
    从数据中自动学习规律
  \end{block}

  \begin{example}
    \begin{equation*}
      y = w \cdot x + b
    \end{equation*}
  \end{example}
  \begin{columns}
    \column{0.3\textwidth}
机器学习的分类
  \begin{itemize}
    \item 监督学习\\(分类/回归)
    \item 无监督学习\\(聚类)
    \item 强化学习\\(奖励机制)
  \end{itemize}
    \column{0.7\textwidth}
\begin{figure}[h!]
\vspace*{-0.45in}
\centering
   \includegraphics[height=1.6in, width=2.9in, viewport=0 0 210 120,clip]{Figures/The_main-types-of-machine-learning.png}
\label{Machine-Learning-types}
\end{figure}
  \end{columns}
\end{frame}

\frame
{
	\frametitle{深度神经网络}
%{\fontsize{8.0pt}{4.2pt}\selectfont{
深度神经网络可以包含上百层神经元,通常有上万个参数,再加上超参数,实际的参数空间几乎是无限大的%}}
\begin{figure}[h!]
%\vspace*{0.05in}
\centering
\includegraphics[height=1.90in,width=4.05in]{Figures/ANN_Algorithm.png}
\caption{\tiny \textrm{Deep Learning Neural Network.}}%(与文献\cite{EPJB33-47_2003}图1对比)
\label{Fig:Deep-Learning-NN}
\end{figure}
}
%\begin{frame}[fragile]{神经网络}
%  \begin{columns}
%    \column{0.4\textwidth}
%    \begin{itemize}
%      \item 模仿人脑神经元结构
%      \item 多层感知机\textrm{(MLP)}
%      \item 激活函数:~\textrm{ReLU,~Sigmoid}
%    \end{itemize}
%
%    \column{0.6\textwidth}
%    \begin{minted}{python}
%# 简单神经网络示例
%model = Sequential()
%model.add(Dense(64, activation='relu'))
%model.add(Dense(10, activation='softmax'))
%    \end{minted}
%  \end{columns}
%\end{frame}
%
%\begin{frame}
%	\frametitle{传统神经网络方法的局限}
%    在自然语言处理\textrm{(NLP)}和计算机视觉\textrm{(CV)}等诸多人工智能领域,循环神经网络\textrm{(RNN)}与卷积神经网络\textrm{(CNN)}曾占据主导地位\\
%    {\fontsize{7.2pt}{6.2pt}\selectfont{\textcolor{red}{随着数据规模的增长和任务复杂度的提升,传统神经网络在长距离依赖和并行计算方面的局限性逐渐凸显}}}
%    \begin{itemize}
%	    \item \textrm{RNN}长距离依赖难题:~实际训练中,由于梯度消失或梯度爆炸问题,\textrm{RNN}很难学习到远距离的信息,\\
%		    {\fontsize{7.2pt}{6.2pt}\selectfont{\textcolor{red}{例如:~在文本翻译任务中,开头的单词信息很难传递到句子末尾}}}\\
%		    \textrm{RNN}的循环结构无法充分利用硬件的并行计算能力,大大增加了训练时间
%	    \item \textrm{CNN}通过堆叠多层卷积来扩大感受范围,获取长距离依赖信息的效率较低\\
%		    {\fontsize{7.2pt}{6.2pt}\selectfont{\textcolor{red}{比如:~在处理长文本时,难以直接捕捉到相隔较远的文本片段之间的语义关联}}}\\
%		    虽然\textrm{CNN}在一定程度上可以并行计算,但由于卷积操作的局部特性,在捕捉全局依赖关系时,计算资源的浪费较为严重
%    \end{itemize}
%\end{frame}
%
%
%\begin{frame}
%	\frametitle{\textrm{Transformer}架构总览}
%	\textrm{Transformer}于\textrm{2017}年%在论文《Attention Is All You Need》中被
%    提出,放弃了\textrm{RNN}和\textrm{CNN}的序列结构与卷积操作,完全基于注意力机制\\
%    {\fontsize{7.2pt}{6.2pt}\selectfont{在长文本处理、并行计算上表现卓越,迅速成为\textrm{NLP}的主流模型}}
%\begin{figure}[h!]
%%\vspace*{-0.15in}
%\centering
%\includegraphics[height=2.0in, width=1.8in, viewport=300 0 1200 1500,clip]{Figures/Transformer_full_architecture.png}
%\caption{\tiny \textrm{Transformer:~full architecture.}}%(与文献\cite{EPJB33-47_2003}图1对比)
%\label{Transformer_full_architecture}
%\end{figure}
%\end{frame}
%
%\subsection{应用领域}
%\subsection{伦理与挑战}
%\begin{frame}{值得思考的问题}
%  \begin{alertblock}{伦理挑战}
%    \begin{itemize}
%      \item 算法偏见与公平性
%      \item 隐私保护问题
%      \item 就业市场影响
%      \item 自主武器系统
%    \end{itemize}
%  \end{alertblock}
%
%  \begin{exampleblock}{未来展望}
%    人类与AI协同发展
%  \end{exampleblock}
%\end{frame}

\section{\rm{AI for Science:}~材料学}
\frame
{
	\frametitle{\textrm{AI for Science:}~材料学}
\begin{figure}[h!]
\vspace*{-0.15in}
\centering
\includegraphics[height=2.95in,width=3.05in]{Figures/AI-for-Science.png}
\label{AI_for_Sciences}
\end{figure}
}

\subsection{\textrm{AI4Sci:}数据基础:~材料计算平台与数据库}
\begin{frame}
	\frametitle{适应异质界面催化模拟自动流程软件}
\begin{minipage}[c]{0.42\linewidth}
\begin{itemize}
\vspace*{-2.75in}
%	\item “标准化”对称性分析功能:~降低\textrm{DFT}的计算量
	\item \textcolor{blue}{前处理}:\\
		计算模型分析与预处理
%	\item \textcolor{magenta}{$\vec k\cdot\vec p$方法}:~提升电子计算的规模%,为\textrm{DFT-MD}计算提供基础
	\item \textcolor{blue}{计算流程设计与管理}:\\
		\begin{enumerate}
			\item 支持计算过程的模块化
			\item 支持高通量、跨尺度材料模拟
			\item 提供计算结果数据管理接口
		\end{enumerate}
	\item \textcolor{blue}{后处理}:\\
		结果数据的分析、挖掘与可视化展示
%	\item \textcolor{magenta}{机器学习}:~优化电子计算结果,获得\textrm{MD}尺度力场,\textrm{DFT-MD}耦合%,获得\textrm{MD}尺度下准确的多体相互作用的力场函数。
%	\item 设计合理完善的程序流程:~利用\textrm{MongoDB}支持的\textrm{FireWorks}计算流程管理%,由微观尺度\textrm{DFT}计算获得介观或宏观尺度的计算物性或者使不同尺度的计算结果更好地实现耦合自洽
\end{itemize}
\end{minipage}
\hskip 2pt
\begin{minipage}[b]{0.47\linewidth}
\begin{figure}[h!]
\centering
%\hskip -35pt
\includegraphics[height=2.18in]{Figures/MP_comp_BCC.png}
\caption{\fontsize{6.5pt}{4.5pt}\selectfont{适用于异质界面的高通量材料计算自动流程软件架构}}%
\label{MP_comp_BCC}
\end{figure}
\end{minipage}
\end{frame}

%\begin{frame}
%	\frametitle{材料智能计算平台}
%	“\textcolor{magenta}{材料多尺度模拟仿真与多目标机器学习大数据平台}”
%	\begin{itemize}
%		\item 材料多尺度模拟流程,电子结构计算优化,化学反应动力学过程与多目标数据收集、特征工程、模型建立和验证等材料机器学习算法相融合
%		\item 材料计算数据库技术应用:~晶体预测结构,半导体带隙,相稳定性,存能与功能材料的物理化学性质等
%	\end{itemize}
%\begin{figure}[h!]
%\centering
%\vspace*{-7pt}
%%\animategraphics[autoplay, loop, width=3.95in, height=1.45in]{15}{Figures/DNN-}{0}{15}
%\includegraphics[height=1.60in,width=2.55in,viewport=0 0 1200 870,clip]{Figures/BCC-Process_1.png}
%\includegraphics[height=0.85in,width=1.40in,viewport=0 0 801 486,clip]{Figures/Patent_license.png}
%%\includegraphics[height=2.00in,width=3.15in,viewport=0 0 1200 870,clip]{Figures/BCC-Process_1.png}
%%\caption{\fontsize{6.5pt}{4.5pt}\selectfont{面向多尺度材料智能计算平台}}%
%\label{BCC-Process_1}
%\end{figure}
%\end{frame}
%

\begin{frame}
	\frametitle{应用:~机器学习构建催化描述符}
\begin{figure}[h!]
\centering
%\hskip -35pt
\includegraphics[height=1.85in]{Figures/MP_comp_BCC-4.png}
%\caption{\fontsize{6.5pt}{4.5pt}\selectfont{面向多尺度材料智能计算平台}}%
\label{MP_comp_BCC_4}
\end{figure}
{\fontsize{7.5pt}{5.5pt}\selectfont{
	面向\textrm{2D~MXenes}有序二元合金\textrm{(OBAs)}催化活性:
	\begin{itemize}
		\item 根据理化知识筛选特征向量
		\item 基于机器学习得到好的特征向量
		\item 对多目标优化,检验特征向量间相关度
		\item 基于特征向量筛选潜在优势催化活性材料
	\end{itemize}}}
\end{frame}

\begin{frame}
	\frametitle{应用:~类石墨烯材料产氢性能优化预测}
\begin{figure}[h!]
\centering
%\hskip -35pt
\includegraphics[height=1.85in]{Figures/MP_comp_BCC-3.png}
%\caption{\fontsize{6.5pt}{4.5pt}\selectfont{面向多尺度材料智能计算平台}}%
\label{MP_comp_BCC_3}
\end{figure}
{\fontsize{7.5pt}{5.5pt}\selectfont{
	应用高通量\textrm{DFT}计算,集成机器学习框架,预测\textrm{2D~MXenes}有序二元合金\textrm{(OBAs)}催化活性趋势并指导\textrm{HER}催化剂设计:}}
{\fontsize{5.5pt}{4.5pt}\selectfont{
	\begin{itemize}
		\item 由\textcolor{red}{数千个}\textrm{2D~MXenes}中筛选出的\textcolor{red}{110种}热稳定性、\textrm{HER}活性优于贵金属%\ch{Pt}
		\textrm{Pt}的潜在\textrm{2D~MXenes~OBAs}
	\item 特别是%\ch{Ti}
		\textrm{Ti}元素主要存在于\textrm{2D~MXenes~OBAs}理想催化剂中与实验合成的\textrm{MXenes}一致,\textcolor{red}{提高效率80\%}\\
	\end{itemize}
		获“\textcolor{blue}{2019中国大数据与智能计算技术创新奖}” \\
\textrm{J.~Mater.~Chem.~A,~2020} ~~~~~~~\url{https://doi.org/10.1039/D0TA06583H}}}
\end{frame}

\begin{frame}
	\frametitle{相关成果和奖励}
\begin{figure}[h!]
\centering
\vskip -5pt
\includegraphics[height=2.5in,width=1.9in]{Figures/2021-BigData_Expo.jpg}
\includegraphics[height=2.5in,width=1.9in]{Figures/2024-Innovation.jpg}
\label{Fig:Award}
%\caption{\fontsize{5.2pt}{6.2pt}\selectfont{$\vec k\cdot\vec p$方法保证计算精度,并计算效率提升}}%
\end{figure}
\end{frame}

\frame
{
	\frametitle{材料数据积累的相关成果和数据登记}
\begin{figure}[h!]
\vspace*{-0.05in}
\centering
\includegraphics[height=2.75in,width=1.85in,viewport=0 0 579 810,clip]{Figures/Registration_Certificate.png}
\includegraphics[height=2.75in,width=2.10in,viewport=0 0 609 799,clip]{Figures/Certificate-of-Patent.png}
%\caption{\tiny \textrm{Pseudopotential for metallic sodium, based on the empty core model and screened by the Thomas-Fermi dielectric function.}}%(与文献\cite{EPJB33-47_2003}图1对比)
\label{Certification}
\end{figure}
}

\subsection{\rm{AI4Sci}支持的微观动力学模拟:~机器学习势函数}
\frame
{
	\frametitle{机器学习势函数}
\begin{figure}[h!]
\vspace*{-0.18in}
\centering
\includegraphics[height=2.65in,width=3.05in,viewport=0 0 832 748,clip]{Figures/NEP_GPUMD.png}
\caption{\tiny\textrm{The NEP model and the GPUMD program can be used to perform atomistic simulation using the trained potentials}}%(与文献\cite{EPJB33-47_2003}图1对比)
\label{AI_for_Science:NEP-GPUMD}
\end{figure}
}

\begin{frame}
	\frametitle{机器学习势函数的构建}
%	构建\textrm{nep}势函数($\mathrm{Zr}$-$\mathrm{H}$势、$\mathrm{Zr}$-$\mathrm{O}$势)的主要步骤
	\begin{itemize}
		\item 准备各类所需的计算模型,完成模型的体系基态能量和原子受力的\textcolor{magenta}{\textrm{DFT}计算}
		\item 针对各模型,定义合适的\textcolor{blue}{描述符}:~与键长、键角和二面角相关
			{\fontsize{6.2pt}{4.2pt}\selectfont{\begin{displaymath}
				\begin{aligned}
					G_i^2=&\sum_{j\neq i}\mathrm{e}^{-\eta(r_{ij}-r_s)^2}f_c(r_{ij})\\
					G_i^3=&2^{1-\zeta}\sum_{j,k\neq i}(1+\lambda\cos\theta_{ijk})^{\zeta}\mathrm{e}^{-\eta(r_{ij}^2+r_{ik}^2+r_{jk}^2)}f_c(r_{ij})f_c(r_{ik})f_c(r_{jk})\\
					G_i^9=&2^{1-\zeta}\sum_{j,k\neq i}(1+\lambda\cos\theta_{ijk})^{\zeta}\mathrm{e}^{-\eta(r_{ij}^2+r_{ik}^2)}f_c(r_{ij})f_c(r_{ik})
				\end{aligned}
			\end{displaymath}}}
		\item 选定模型中的一部分结构及其能量和原子受力,应用机器学习方法(神经网络),用来\textcolor{magenta}{训练机器学习势}
		\item \textcolor{magenta}{优化}训练集的\textcolor{blue}{描述符},进一步改善机器学习势
		\item \textcolor{magenta}{评估并测试}产生的机器学习势
	\end{itemize}
\end{frame}
	
\begin{frame}
	\frametitle{神经网络势的训练}
	能量的表示
	\begin{displaymath}
		E_i=f_1^3\bigg\{b_1^3+\sum_{k=1}^{\mathrm{M}_{\mathrm{layer},2}}\omega_{n1}^{23}f_n^2\bigg[b_n^2+\sum_{m=1}^{\mathrm{M}_{\mathrm{layer},1}}\omega_{mn}^{12}f_m^1\bigg(b_m^1+\sum_{l=1}^{\mathrm{M}_{\mathrm{sym}}}\omega_{lm}^{01}G_{ij}\bigg)\bigg]\bigg\}
	\end{displaymath}
神经网络的优化函数
\begin{displaymath}
	\Gamma=\sum_{n=1}^{N_{\mathrm{struct}}}(E_{\mathrm{NN}}^n-E_{\mathrm{ref}}^n)^2+\beta^2\sum_{n=1}^{N_{\mathrm{struct}}}\sum_{m=1}^{3N_{\mathrm{atom}}^n}(F_{n,\mathrm{NN}}^m-F_{n,\mathrm{ref}}^m)^2
\end{displaymath}
\end{frame}

\begin{frame}
	\frametitle{技术服务:~中子慢化材料辐照微观动力学研究}
	\textcolor{red}{探索反应堆中子慢化材料在辐照工况下的微观动力学机理}:\\
	受北京科技大学委托,承担基于第一原理-机器学习势函数模拟研究中子慢化材料氢化锆固体中氢原子在高温、密闭真空环境中的扩散行为,比照实验结果
\begin{figure}[!ht]
\centering
\vspace*{-0.05in}
\includegraphics[height=1.50in,width=1.95in,viewport=200 0 1920 1088,clip]{Figures/Contract-BCC-2025-05.png}
\includegraphics[height=1.50in,width=1.95in,viewport=0 0 930 748,clip]{Figures/Phase_diagram-Zr_H-systems.png}
\caption{\tiny \textrm{Phase diagram Zr-H systems.}}
\label{Fig:Phase_diagram-Zr_H-systems}
\end{figure}
\end{frame}

\subsection{\rm{AI4Sci}的材料数据关联:~化学-化工知识图谱建设}
\begin{frame}
	\frametitle{化学-化工知识图谱}
	受中科合成油技术股份有限公司委托,开发化学-化工知识图谱,探索\textrm{AI for Science}的数据组织与关联的应用潜力
\begin{figure}[h!]
\centering
%\vskip -8pt
\includegraphics[height=2.20in,width=4.00in,viewport=0 0 240 130,clip]{Figures/KG_Chem-Enflurane.png}
\caption{\tiny 知识图谱的词条内容:~化合物\textrm{安氟醚}的有关知识}%(与文献\cite{EPJB33-47_2003}图1对比)
\label{Fig:KG_Chem-Enflurane}
\end{figure}
\end{frame}

\begin{frame}[allowframebreaks]
	\frametitle{化学-化工知识图谱的目标}
	\begin{itemize}
%	 \setlength{\itemsep}{30pt}
%{\fontsize{7.5pt}{5.5pt}\selectfont{
		\item 以化合物为核心,借助语义网\textrm{(Semantic Web)},组织、表示和存储化学-化工和领域特定类型的知识
	\item 面向碳基础材料,构建拥有学习和推理能力
		\item 具备初级的创造知识的能力,发挥人工智能的可能作用%}}
\end{itemize}
\begin{figure}[h!]
\centering
\includegraphics[height=1.50in,width=1.75in,viewport=0 0 950 790,clip]{Figures/Mapping-the-relationship-between-molecule-and-synthon.png}
\hspace{5pt}
\includegraphics[height=1.50in,width=1.55in,viewport=0 0 750 790,clip]{Figures/TWA-KG-Marie.png}
%\caption{\small\textrm{Mapping the relationship molecule (chemical) and synthon (abstract) concepts and illustrating them with instrances. cite from~\cite{ACR56-128_2023}}}%(与文献\cite{EPJB33-47_2003}图1对比)
\label{Fig:Mapping-relationship-molecule-synthon-2}
\end{figure}
\textcolor{purple}{目标:}~面向人工智能的全方位转型:~智能实验室-智能科学家
\end{frame}

%\subsection{化学-化工知识图谱的总体目标和框架}
\frame
{
	\frametitle{化学-化工知识图谱项目的目标}
%化学-化工知识图谱项目的目标:~
	\textcolor{magenta}{通过制定化学化工数据采集方法,采用``本体-要素-概念''三位一体的技术实现业务建模,支持智慧语义认知算法和多种分析挖掘算法,完成化学-化工知识库的建设} 
\begin{itemize}
	\item \textcolor{blue}{化学-化工知识体系分类研究}:~获取其共性特征以及歧义特征\\
		{\fontsize{7.2pt}{5.2pt}\selectfont{采用标签化、智能化建立知识库的方法,主要研究知识体系、产研报告、供应链信息、原料资料、安全质量、节能环保等细分语义分类,便于在知识图谱中进行建模训练}}

\item \textcolor{blue}{行业知识数据梳理入库}\\
	{\fontsize{7.2pt}{5.2pt}\selectfont{通过百度百科、\textrm{Wiki}百科等渠道,利用数据采集系统,按研究分类自定义采集模板,将更完善的行业知识入库,有效区分结构化数据与非结构化数据的处理方案}}

\item \textcolor{blue}{构建组织表达模型}\\
	{\fontsize{7.2pt}{5.2pt}\selectfont{采用数据提取、数据整合、知识融合、知识推理、质量评估、知识存储、图谱应用等建模流程节点方案,建立一套完善、可靠的化工化学全业务知识图谱系统方案}}
\end{itemize}
\textcolor{red}{构建化学-化工行业知识图谱,为化学化工行业的研究、应用和决策提供有力支持}
}
%\subsection{化学-化工知识图谱的总体框架}
\begin{frame}
	\frametitle{化学-化工知识组织的技术实现}
	以知识图谱为例,说明(煤)化学-化工类知识采集、分类与组织
\begin{figure}[h!]
%\vspace*{-0.05in}
\centering
\includegraphics[height=2.08in,width=4.00in,viewport=0 0 245 113,clip]{Figures/KG_Chem-Frame.png}
\caption{\tiny 项目总体技术框架.}%(与文献\cite{EPJB33-47_2003}图1对比)
\label{KG_Chem-Frame}
\end{figure}
\end{frame}

\begin{frame}
	\frametitle{化学-化工知识的数据采集}
数据采集:~收集和整理化学-化工领域的文献、数据集、专利信息等
\begin{figure}[h!]
\centering
\includegraphics[height=1.80in,width=3.00in,viewport=0 0 240 150,clip]{Figures/KG_Chem-Tech_Frame.png}
\caption{\tiny 数据采集技术框架.}%(与文献\cite{EPJB33-47_2003}图1对比)
\label{KG_Chem-Tech-Frame}
\end{figure}
\vspace*{-0.1in}
\begin{itemize}
	\item {\fontsize{7.2pt}{5.2pt}\selectfont{建立规范的数据采集流程,确保数据的准确性和完整性}}
\item {\fontsize{7.2pt}{5.2pt}\selectfont{对业务进行建模,明确了化学化工领域的概念和关系,为后续知识图谱构建奠定基础}}
\end{itemize}
\end{frame}

\begin{frame}
	\frametitle{化学-化工知识库构建}
	采用本体构建技术,将领域知识转化为计算机可理解的形式:~
\begin{figure}[h!]
\centering
\includegraphics[height=1.30in,width=3.80in,viewport=0 0 200 57,clip]{Figures/KG_Chem-data-flow.png}
\caption{\tiny 知识表达模型和构建.}%(与文献\cite{EPJB33-47_2003}图1对比)
\label{KG_Chem-data-flow}
\end{figure}
	\begin{itemize}
		\item {\fontsize{7.2pt}{5.2pt}\selectfont{创建本体的概念、属性和关系,并进行本体的验证和优化}}
		\item {\fontsize{7.2pt}{5.2pt}\selectfont{对知识库进行了组织和分类,使用户可以方便地浏览和检索相关信息}}
	\end{itemize}
\end{frame}

%\subsection{相关技术与算法}
\begin{frame}
	\frametitle{数据:~一切知识的基础}
%	针对化学化工相关信息多源异构难以获取的问题,
%	基于网络爬虫技术实现多源数据信息的分布式爬虫,获取相关信息,训练并扩大,形成相对结构化的化学-化工信息语料库 
\begin{figure}[h!]
\centering
\vskip -8pt
\includegraphics[height=2.10in,width=4.00in,viewport=0 0 210 95,clip]{Figures/KG_Chem-Info.png}
\caption{\tiny 有组织的数据是一切\textrm{AI}知识的基础}%(与文献\cite{EPJB33-47_2003}图1对比)
\label{Fig:KG_Chem-Info}
\end{figure}
\end{frame}

\begin{frame}
	\frametitle{数据的提取}
化学-化工专家文档以层级结构为特征,通过神经网络可抽取文档上下文层级语义关系:~
根据专家简介词、句、段落文档层级结构特征,分层捕获文档之间的语义关系,实现化学化工领域细粒度分类
\begin{figure}[h!]
\centering
%\vskip -8pt
\includegraphics[height=1.60in,width=4.00in,viewport=0 0 210 80,clip]{Figures/KG_Chem-extract.png}
\caption{\tiny 数据的提取是一切\textrm{AI}知识分类的逻辑起点}%(与文献\cite{EPJB33-47_2003}图1对比)
\label{Fig:KG_Chem-Extract}
\end{figure}
\end{frame}

\begin{frame}
	\frametitle{知识图谱\textrm{RDF}图数据检索}
\begin{figure}[h!]
\centering
%\vskip -8pt
\includegraphics[height=2.60in,width=4.00in,viewport=0 0 240 180,clip]{Figures/KG_Chem-Inorganic.png}
\caption{\tiny 知识图谱的数据\textrm{RDF}图:~无机化合物}%(与文献\cite{EPJB33-47_2003}图1对比)
\label{Fig:KG_Chem-Inorganic}
\end{figure}
\end{frame}

\begin{frame}
	\frametitle{知识图谱\textrm{RDF}图数据:~示例}
\begin{figure}[h!]
\centering
%\vskip -8pt
\includegraphics[height=2.40in,width=4.00in,viewport=0 0 150 100,clip]{Figures/KG_Chem-Chemist.png}
\caption{\tiny 知识图谱的数据\textrm{RDF}图:~化工专家-姓名-性别}%(与文献\cite{EPJB33-47_2003}图1对比)
\label{Fig:KG_Chem-Chemist}
\end{figure}
\end{frame}

\begin{frame}
	\frametitle{知识图谱\textrm{RDF}图数据:~示例}
\begin{figure}[h!]
\centering
%\vskip -8pt
\includegraphics[height=2.10in,width=4.00in,viewport=0 0 220 115,clip]{Figures/KG_Chem-Chemist_2.png}
\caption{\tiny 知识图谱的数据\textrm{RDF}图:~化工专家-姓名-身份-学历}%(与文献\cite{EPJB33-47_2003}图1对比)
\label{Fig:KG_Chem-Chemist2}
\end{figure}
\end{frame}

\begin{frame}
	\frametitle{大模型与知识图谱的关系}
\begin{figure}[h!]
\centering
\vskip -8pt
\includegraphics[height=1.2in,width=3.90in,viewport=3 0 185 60,clip]{Figures/KG_Chem-LLM.png}
\caption{\tiny\textrm{大模型与知识图谱的互补关系}}%(与文献\cite{EPJB33-47_2003}图1对比)
\label{Fig:KG_Chem-LLM}
\end{figure}
\begin{itemize}
	\item {\fontsize{7.2pt}{6.2pt}\selectfont{知识图谱对大模型的增强:~为大模型提供真实可靠的知识,减轻大模型产生幻觉的问题,提供解释和推理知识的手段,探究大模型内部复杂的工作步骤和推理过程}}%,还可以作为外部检索工具,帮助大模型解决公平、隐私和安全等问题
	\item {\fontsize{7.2pt}{6.2pt}\selectfont{大模型对知识图谱的增强:大模型在零样本和少样本的训练中,能够应对知识图谱构建、补全、推理和问答等各种挑战}}%。例如,大模型可以利用零样本或少样本学习的信息提取能力,从文本或其他数据源中完成实体抽取和关系抽取任务,节约数据标注的时间和成本;还可以作为额外知识库提取可信知识,完成知识图谱的补全。
\end{itemize}
\end{frame}

%\frame
%{
%	\frametitle{化学-化工知识图谱的组成}
%\begin{figure}[h!]
%\centering
%\vskip -8pt
%\includegraphics[height=2.45in,width=4.05in,viewport=0 0 1170 700,clip]{Figures/Connection-of-OntoSpecies-to-segments-of-KG.png}
%\caption{\tiny\textrm{Connection of OntoSpecies to other segments of TWA KG. cite from~\cite{ACR56-128_2023}}}%(与文献\cite{EPJB33-47_2003}图1对比)
%\label{Fig:OntoSpecies-to-segments-TWA}
%\end{figure}
%}
%
%\frame
%{
%	\frametitle{化学-化工知识图谱}
%	化学-化工知识图谱:~以化学物种(元素、化合物)为核心的\textcolor{cyan}{多个知识的\textrm{Ontology}组成}
%	\begin{itemize}
%		\item \textrm{OntoSpecites}:~主要纪录化学物种的知识,包括分子式、电荷、分子量和自旋多重度等
%		\item \textrm{OntoKin}:~表示反应机理的知识,纪录反应物、产物和反应过程的信息
%		\item \textrm{OntoCompChem}:~表示计算化学的信息的知识,计算信息的描述包括
%			\begin{itemize}
%				\item 计算对象:~单点计算、结构优化和频率计算
%				\item 计算使用的软件,{\fontsize{7.2pt}{5.2pt}\selectfont{如\textrm{Gaussian~16}}}
%				\item 计算中使用的方法,包括泛函和基组{\fontsize{7.2pt}{5.2pt}\selectfont{~如\textrm{B3LYP,~6-31G(d)}}}
%				\item 电荷与自旋极化
%			\end{itemize}
%		\item \textrm{OntoCompExp}:~表示化学实验信息的知识,包括各类化学实验条件
%	\end{itemize}
%}
%
%\frame
%{
%	\frametitle{\textrm{OntoSpecies}:~化学物种的描述}
%\begin{figure}[h!]
%\centering
%\vskip -8pt
%\includegraphics[height=2.40in,width=3.25in,viewport=0 0 990 750,clip]{Figures/Key_OntoSpecies-and-external_concepts.png}
%\caption{\tiny\textrm{Key OntoSpecies (black) and external (blue) concepts, along with a number of properties (green) used to describe chemical species in TWA KG. cite from~\cite{ACR56-128_2023}}}%(与文献\cite{EPJB33-47_2003}图1对比)
%\label{Fig:Key-OntoSpecies-and-external-concepts}
%\end{figure}
%}
%
%\frame
%{
%	\frametitle{\textrm{Agent}:~化学-化工知识图谱的组织工具}
%	\textrm{Agent}:~能够感知环境、进行决策和执行动作的智能处理软件
%	\begin{itemize}
%		\item \textrm{Agent}工作方式类似于人类代理:~能接收输入数据(如传感器信息、文本、图像等),通过分析和处理数据,理解环境和任务要求,并做出相应的决策和行动
%		\item 应用场景广泛,如自动驾驶车辆、智能机器人、语音助手等
%		\item \textrm{Agent}核心功能:~感知、推理和决策
%			\begin{itemize}
%				\item 感知:~通过传感器等方式获取环境信息的能力,例如通过摄像头获取图像或通过麦克风获取声音
%				\item 推理:~基于获取的信息进行逻辑推理和分析的能力,以了解环境和任务需求
%				\item 决策:~根据推理结果做出相应的决策,并执行相应的动作
%			\end{itemize}
%		\item 通过与环境的交互和反馈,\textrm{Agent}可以逐步改进性能和表现,实现好的任务执行能力\\
%		\item \textrm{Agent}设计和训练,需要结合机器学习和人工智能技术,如强化学习、深度学习等
%	\end{itemize}
%}
%
%\frame
%{
%	\frametitle{知识图谱组织示例}
%\begin{figure}[h!]
%\centering
%\vskip -8pt
%\includegraphics[height=2.70in,width=3.75in,viewport=0 0 1010 750,clip]{Figures/Automated-linking-between-OntoSepcies-Kin-CompChem.png}
%\caption{\tiny\textrm{Automated linking between OntoSpecies, OntoKin and OntoCompChem. cite from~\cite{ACR56-128_2023}}}%(与文献\cite{EPJB33-47_2003}图1对比)
%\label{Fig:Automated-linking-between-OntoSpecies-Kin-CompChem}
%\end{figure}
%}
%
%\frame
%{
%	\frametitle{知识图谱组织示例}
%\begin{figure}[h!]
%\centering
%\vskip -8pt
%\includegraphics[height=2.20in,width=4.05in,viewport=0 0 1330 700,clip]{Figures/Key_concepts-in-OntoMOPs-and-designed-MOPs.png}
%\caption{\tiny\textrm{Key concepts in OntoMOPs (left) and examples of newly rationally designed MOPs (right). cite from~\cite{ACR56-128_2023}}}%(与文献\cite{EPJB33-47_2003}图1对比)
%\label{Fig:OntoMOPs-MOPs}
%\end{figure}
%}

%\begin{frame}
%	\frametitle{应用:~类石墨烯材料的稳定性优化预测}
%\begin{figure}[h!]
%\centering
%%\hskip -35pt
%\includegraphics[height=1.55in]{Figures/MP_comp_BCC-5.png}
%%\caption{\fontsize{6.5pt}{4.5pt}\selectfont{面向多尺度材料智能计算平台}}%
%\label{MP_comp_BCC_5}
%\end{figure}
%{\fontsize{7.5pt}{5.5pt}\selectfont{
%	\begin{itemize}
%		\item 应用高通量建模软件构建潜在构型5600多种,利用\textrm{Materials~Projects}材料计算数据库提取竞争相数据
%		\item 通过热分解过程,组合化学反应式2000多组,筛选出热力学稳定的材料299种
%		\item 通过支持向量、高斯过程、随机深林、神经网络以及\textrm{adaboost}多种机器学习回归模型,利用13种常见特征参数对稳定性做了预测
%		\item 预测准确率达到94\%,节省计算成本高达70\%
%	\end{itemize}}}
%\end{frame}
%
%\begin{frame}
%	\frametitle{应用:~监督学习预测半导体材料带隙}
%\begin{figure}[h!]
%\centering
%\hspace*{-8pt}
%\includegraphics[height=1.45in]{Figures/MP_comp_BCC-6.png}
%%\caption{\fontsize{6.5pt}{4.5pt}\selectfont{面向多尺度材料智能计算平台}}%
%\label{MP_comp_BCC_6}
%\end{figure}
%{\fontsize{7.5pt}{5.5pt}\selectfont{
%	\begin{itemize}
%		\item 常规通行的材料模拟中带隙计算相当耗时
%		\item 利用%类石墨烯新能源
%	材料高通量智能计算与多目标机器学习集成研发平台,高通量自动化快速构建类石墨烯材料结构23870种,并构建数据库结合\textrm{KRR}、\textrm{SVR}、\textrm{GPR}、\textrm{Bagging}机器学习回归模型进行训练预测
%		\item \textrm{GPR}方法预测准确性达到了97\%,可以节省计算成本90\%多
%	\end{itemize}}}
%\end{frame}

%\begin{frame}[allowframebreaks]
%	\frametitle{主要合作与推广应用}
%		中科合成油(合作)
%	\begin{itemize}
%	 \setlength{\itemsep}{30pt}
%	\item 化学-化工知识图谱的建设
%\begin{figure}[h!]
%\centering
%\includegraphics[height=1.50in,width=1.75in,viewport=0 0 950 790,clip]{Figures/Mapping-the-relationship-between-molecule-and-synthon.png}
%\hspace{5pt}
%\includegraphics[height=1.50in,width=1.55in,viewport=0 0 750 790,clip]{Figures/TWA-KG-Marie.png}
%%\caption{\small\textrm{Mapping the relationship molecule (chemical) and synthon (abstract) concepts and illustrating them with instrances. cite from~\cite{ACR56-128_2023}}}%(与文献\cite{EPJB33-47_2003}图1对比)
%\label{Fig:Mapping-relationship-molecule-synthon}
%\end{figure}
%	\begin{itemize}
%{\fontsize{7.5pt}{5.5pt}\selectfont{
%		\item 以化合物为核心,借助语义网\textrm{(Semantic Web)},组织、表示和存储化学-化工和领域特定类型的知识
%		\item 构建拥有学习和推理能力,具备初级的创造知识的能力}}
%	\end{itemize}
%	\item 问-答式煤化工智能模型建设
%\begin{figure}[h!]
%\centering
%\includegraphics[height=1.40in,width=3.50in,viewport=0 0 1986 800,clip]{Figures/MeetingRecord_SCTC-BCC.png}
%\label{Fig:Meeting_Record}
%\end{figure}
%\begin{itemize}
%	\item 面向人工智能的全方位转型:\\
%		面向碳基础材料、发挥人工智能的作用
%	\item 大模型加持专业知识
%\end{itemize}
%	\end{itemize}
%\textcolor{purple}{目标:}~智能实验室-智能科学家
%\end{frame}
%
%\begin{frame}
%	\frametitle{数据驱动的材料研发:~应用前景}
%	\begin{enumerate}
%	 \setlength{\itemsep}{20pt}
%	 \item 航空发动机材料:~\textcolor{blue}{镍基单晶高温合金材料}\\
%	合金组分优化与强化功能提升
%
%\item 煤化工催化材料:~\textcolor{blue}{新型铁触媒材料}\\
%	反应活化性能提升与化学平衡的移动
%
%		\item 稀土功能材料:~\textcolor{blue}{钕铁硼永磁材料},\textcolor{blue}{稀土发光材料}\\
%	3\textit{d}-4\textit{f} 电子相互作用机制的认知
%
%	\end{enumerate}
%
%	\textcolor{magenta}{材料组分趋于复杂、材料机理认知趋于微观、材料与数据趋于膨胀}
%%\begin{figure}[h!]
%%\vspace*{-0.20in}
%%\centering
%%\includegraphics[height=2.90in,width=3.70in]{Figures/Main-qimg.jpeg}
%%\label{BMDS}
%%\end{figure}
%\end{frame}
%------------------------------------------------------------------------Reference----------------------------------------------------------------------------------------------
%		\frame[allowframebreaks]
%{
%\frametitle{主要参考文献}
%\begin{thebibliography}{99}
%{\tiny
%	\bibitem{PR136-B864_1964}\textrm{P. Hohenberg and W. Kohn, \textit{Phys. Rev.} \textbf{136} (1964), B864}
%	\bibitem{PR140-A1133_1965}\textrm{W. Kohn and L.J. Sham, \textit{Phys. Rev.} \textbf{140} (1965), A1133}
%	\bibitem{PRB50-17953_1994}\textrm{P. E. Bl\"ochl. \textit{Phys. Rev.} B, \textbf{50} (1994), 17953}
%	\bibitem{PRB59-1758_1999}\textrm{G. Kresse and D. Joubert \textit{Phys. Rev.} B, \textbf{59} (1999), 1758}
%	\bibitem{Elect_Stru}\textrm{Richard. M. Martin. \textit{Electronic Structure: Basic Theory and Practical Methods} (Cambridge University Press, Cambridge, England, 2004)}
%        \bibitem{Singh}\textrm{D. J. Singh. \textit{Plane Wave, PseudoPotential and the LAPW method} (Kluwer Academic, Boston,USA, 1994)}					%
%}
%\end{thebibliography}
%%\nocite*{}
%}
%\subsection{通用语义大模型的搭建}
%\begin{frame}
%	\frametitle{知识问答大模型的搭建}
%\begin{figure}[h!]
%\centering
%\vskip -8pt
%\includegraphics[height=1.90in,width=3.75in,viewport=0 0 1409 750,clip]{Figures/MaxKB_login.png}
%\caption{\tiny\textrm{The login of MaxKB}}%(与文献\cite{EPJB33-47_2003}图1对比)
%\label{Fig:MaxKB_login}
%\end{figure}
%{\fontsize{7.5pt}{6.0pt}\selectfont{化学-化工知识助手基于通用的\textrm{MaxKB}大模型知识问答系统搭建 (当前通用模型大小约\textrm{13TB}),主要通过对中科合成油提供的数据(主要是文献,约\textrm{500}篇)的学习,训练面向煤化工研究的专业知识问-答模型}}
%\end{frame}
%
%\begin{frame}
%	\frametitle{知识问答大模型的搭建}
%\begin{figure}[h!]
%\centering
%\vskip -8pt
%\includegraphics[height=2.2in,width=3.90in,viewport=3 0 1848 1041,clip]{Figures/MaxKB_Creat-APP.png}
%\caption{\tiny\textrm{The Configuration of MaxKB}}%(与文献\cite{EPJB33-47_2003}图1对比)
%\label{Fig:MaxKB_Creat-APP}
%\end{figure}
%\end{frame}
%
%\begin{frame}
%	\frametitle{知识问答大模型的搭建}
%\begin{figure}[h!]
%\centering
%\vskip -8pt
%\includegraphics[height=2.20in,width=3.90in,viewport=0 0 1850 1041,clip]{Figures/MaxKB_Chose-Model.png}
%\caption{\tiny\textrm{The Configuration of MaxKB}}%(与文献\cite{EPJB33-47_2003}图1对比)
%\label{Fig:MaxKB_Chose-Model}
%\end{figure}
%\end{frame}
%
%\begin{frame}
%	\frametitle{知识问答大模型的搭建}
%\begin{figure}[h!]
%\centering
%\vskip -8pt
%\includegraphics[height=0.80in,width=3.90in,viewport=0 650 1850 1054,clip]{Figures/MaxKB_NewDatabase.png}
%\includegraphics[height=1.80in,width=3.90in,viewport=0 0 1850 880,clip]{Figures/MaxKB_Database.png}
%\caption{\tiny\textrm{Uploading a New File to the Database}}%(与文献\cite{EPJB33-47_2003}图1对比)
%\label{Fig:MaxKB_Database}
%\end{figure}
%\end{frame}
%
%\subsection{化学-化工知识问答}

\begin{frame}
	\frametitle{化学-化工知识问答:~示例}	
\begin{figure}[h!]
\centering
%\vskip -8pt
\includegraphics[height=2.30in,width=4.00in,viewport=0 0 1528 875,clip]{Figures/MaxKB_Chem.png}
\caption{\tiny\textrm{\url{http://123.59.0.69:8080/ui/chat/f6385eb7c6c8a618}}}%(与文献\cite{EPJB33-47_2003}图1对比)
\label{Fig:MaxKB_Chem}
\end{figure}
\end{frame}

\begin{frame}
	\frametitle{化学-化工知识问答:~示例}	
\begin{figure}[h!]
\centering
\vskip -8pt
\includegraphics[height=1.40in,width=3.30in,viewport=0 0 978 447,clip]{Figures/Allma_MaxKB-1.png}
\includegraphics[height=1.30in,width=3.30in,viewport=0 342 924 759,clip]{Figures/Allma_MaxKB-2.png}
%\includegraphics[height=2.60in,width=3.70in,viewport=0 0 924 759,clip]{Figures/Allma_MaxKB-2.png}
\caption{\tiny\textrm{Chemical-Chemistry Chat-Model}}%(与文献\cite{EPJB33-47_2003}图1对比)
\label{Fig:MaxKB_Chat-Model-2}
\end{figure}
\end{frame}
%

\frame
{
	\frametitle{其他成果与论著}
\begin{itemize}
	\item {\fontsize{8.2pt}{6.2pt}\selectfont{\textcolor{red}{国家自然科学基金}~\textcolor{blue}{``低维材料等离和激子极化激元的第一性原理研究''}}}
%		{\fontsize{8.2pt}{4.2pt}\selectfont{(项目批准号:~\textrm{12474217})}}
%		\vskip 1pt
%		{\fontsize{8.2pt}{6.2pt}\selectfont{(项目持续年度:~ \textrm{2025.01-2028.12})}}
	\item {\fontsize{8.2pt}{6.2pt}\selectfont{\textcolor{red}{新材料研发及应用国家重大专项}~\textcolor{blue}{``基于人工智能技术的高性能多尺度分子动力学模拟平台''项目}}}
\end{itemize}
\begin{itemize}
%	\setlength{\itemsep}{1pt}
	\item {\fontsize{8.0pt}{4.2pt}\selectfont{\underline{赵琉涛}, \underline{姜骏}, \underline{王彩群}, 潘勇, 潘震西, \textcolor{blue}{计算材料科学理论与实践}, 人民邮电出版社, (北京), 2021}}
\end{itemize}
\begin{figure}[h!]
\centering
\vskip -5pt
\includegraphics[height=1.7in,width=1.3in]{Figures/Cover-Computing_Materials.jpg}
\label{Fig:Cover}
%\caption{\fontsize{5.2pt}{6.2pt}\selectfont{$\vec k\cdot\vec p$方法保证计算精度,并计算效率提升}}%
\end{figure}
}

\begin{frame}
	\frametitle{其他成果与论著}
	\begin{itemize}
		\item {\fontsize{5.2pt}{1.2pt}\selectfont{\textrm{Zhenxi Pan, Yong Pan, \underline{Jun Jiang} and \underline{Liutao Zhao}$^{\dagger}$, \textcolor{blue}{High-Throughput Electronic Band Structure Calculations for Hexaborides}, \textit{CompCom~2019 AIOSC}, \textbf{998}, (2019), 386-395}}}
		\item {\fontsize{5.2pt}{1.2pt}\selectfont{\textrm{Yurui Wang, Zhihui Du$^{\dagger}$, \underline{Jun Jiang}, Baokun Lu and Chongyu Wang, \textcolor{blue}{Modeling the Parallel Efficiency of Density Functional Theory based Jobs on Sunway TaihuLight}, \textit{2019 IEEE-CSE and IEEE-EUC}, (2019), 199-204}}}
		\item {\fontsize{5.2pt}{1.2pt}\selectfont{\textrm{Zhihui Du$^{\dagger}$, Xinning Hui, Yurui Wang, \underline{Jun Jiang}, Jason Liu, Baokun Lu and Chongyu Wang, \textcolor{blue}{Inter-Job Scheduling of High-Througput Material Screening Applications}, \textit{2020 IPDPS}, (2020), 841-852}}}
		\item {\fontsize{5.2pt}{1.2pt}\selectfont{\textrm{Jianxin Huang, Jinkai Wang, Hao Wang$^{\dagger}$, Jiajun Lu, Xiao-Gang Lu, \underline{Jun Jiang} and Ying Chen, \textcolor{blue}{Influences of multicenter bonding and interstitial elements on psudo-twinned $\gamma$-\ch{TiAl} crystal}, \textit{Phys. Scr.}, \textbf{97}, (2022), 085403}}}
		\item {\fontsize{5.2pt}{1.2pt}\selectfont{\textrm{\underline{Caiqun Wang}, \underline{Penglin Gao}, \underline{Hongfei Li}, Mei Yang, \underline{Jun Jiang}, \underline{Liutao Zhao} and Ping Qian$^{\dagger}$, \textcolor{blue}{Single-atom catalysts:~Effects of end-group regulation on catalytic activity}, \textit{Mater. Tod. Comm..}, \textbf{40}, (2024), 109482}}}
	\end{itemize}
\end{frame}

\section{团队面临的问题}
\frame
{
	\frametitle{面临的问题}
			基于材料基因工程的理念,计算材料团队探索\textrm{AI4Sci}支持的材料微观尺度模拟的新模式,取得了一些进展,但也面临着一些问题
	\begin{itemize}
	\setlength{\itemsep}{5pt}
\item 基于\textrm{AI4Sci}的工作依然处于起步阶段,还没有形成成熟完整的技术路线,对\textrm{AI}在新材料研究中的作用和本质,还有待更深入的理解和认识
		\item 基础研究需要协同与交叉,但当前团队成员与高校、科研院所的交流不够,有待加强
		\item 团队的组织和研发模式需要探索\\
			{\fontsize{8.2pt}{6.2pt}\selectfont{研发重点面向基础科研的应用领域,一方面相关研究工作对计算资源的需求较大、产品和成果的市场化程度低,但同时相关领域研究的自主可控需求程度高,需要探索研究应用-市场需求的平衡}}
	\end{itemize}
	\huge\textcolor{red}{衷心感谢中心领导对团队的支持}
}
