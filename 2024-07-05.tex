%%%%%%%%%%%%%%%%%%%%%%%%%%%%%%%%%%%%%%%%%%  不使用 authblk 包制作标题  %%%%%%%%%%%%%%%%%%%%%%%%%%%%%%%%%%%%%%%%%%%%%%
%-------------------------------PPT Title-------------------------------------
\title{\ch{Ni}表面\ch{CO2}的吸附}
%-----------------------------------------------------------------------------

%----------------------------Author & Date------------------------------------
%\author[\textrm{Jun\_Jiang}]{姜\;\;骏\inst{}} %[]{} (optional, use only with lots of authors)
%% - Give the names in the same order as the appear in the paper.
%% - Use the \inst{?} command only if the authors have different
%%   affiliation.
\institute[BCC]{\inst{}%
%\institute[Gain~Strong]{\inst{}%
\vskip 0pt 北京市计算中心有限公司~材料计算团队}
%\vskip -20pt {\large 格致斯创~科技}}
\date[\today] % (optional, should be abbreviation of conference name)
{	%{\fontsize{6.2pt}{4.2pt}\selectfont{\textcolor{blue}{E-mail:~}\url{jiangjun@bcc.ac.cn}}}
\vskip 45 pt {\fontsize{8.2pt}{6.2pt}\selectfont{%北京科技大学% 报告地点
	\vskip 5 pt \textrm{2024.07.05}}}
}

%% - Either use conference name or its abbreviation
%% - Not really information to the audience, more for people (including
%%   yourself) who are reading the slides onlin%%   yourself) who are reading the slides onlin%%   yourself) who are reading the slides onlineee
%%%%%%%%%%%%%%%%%%%%%%%%%%%%%%%%%%%%%%%%%%%%%%%%%%%%%%%%%%%%%%%%%%%%%%%%%%%%%%%%%%%%%%%%%%%%%%%%%%%%%%%%%%%%%%%%%%%%%
\subject{}
% This is only inserted into the PDF information catalog. Can be left
% out.
%\maketitle
\frame
{
%	\frametitle{\fontsize{9.5pt}{5.2pt}\selectfont{\textcolor{orange}{“高通量并发式材料计算算法与软件”年度检查}}}
\titlepage
}
%-----------------------------------------------------------------------------

%------------------------------------------------------------------------------列出全文 outline ---------------------------------------------------------------------------------
%\section*{}
%\frame[allowframebreaks]
%{
%  \frametitle{}
%%  \frametitle{\textcolor{mycolor}{\secname}}
%  \tableofcontents%[current,currentsection,currentsubsection]
%}
%在每个section之前列出全部Outline
%类似的在每个subsection之前列出全部Outline是\AtBeginSubsection[]
%\AtBeginSection[]
%{
%  \frame<handout:0>%[allowframebreaks]
%  {
%    \frametitle{Outline}
%%全部Outline中,本部分加亮
%    \tableofcontents[current,currentsection]
%  }
%}

%-----------------------------------------------PPT main Body------------------------------------------------------------------------------------
\small
\frame
{
	\frametitle{基底模型:~\textrm{\ch{Ni}}-\textrm{\ch{SiO2}}}
	\begin{itemize}
		\item 金属的结构:\textrm{HCP-\ch{Ni}}
		\item \textrm{\ch{SiO2}}与底层\textrm{\ch{Ni}}原子固定,上层\textrm{\ch{Ni}}原子可弛豫
	\end{itemize}
\begin{figure}[h!]
\vskip -0.05in
\centering
\includegraphics[width=3.1in,height=1.8in,viewport=0 0 1528 879, clip]{/home/jun-jiang/Pictures/Screenshot from 2024-07-04 16-10-04.png}
\caption{\tiny \textrm{\ch{Ni}-\ch{SiO2}}基底模型.}%(与文献\cite{EPJB33-47_2003}图1对比)
\label{Substrct_0}
\end{figure}
\vskip -0.08in
基底体系的基态能量 $\mathrm{E}_{sub}=-1014.052159~\mathrm{eV}$
}

\frame
{
	\frametitle{吸附模型:~基底吸附\textrm{\ch{CO2}}}
\begin{figure}[h!]
\vskip -0.1in
\centering
\includegraphics[width=3.5in,height=2.0in,viewport=0 0 1528 879, clip]{/home/jun-jiang/Pictures/Screenshot from 2024-07-04 16-29-18.png}
\caption{\tiny \textrm{基底吸附\textrm{\ch{CO2}}模型.}}%(与文献\cite{EPJB33-47_2003}图1对比)
\label{Substrct_1}
\end{figure}
基底吸附\textrm{CO2}的基态能量 $\mathrm{E}_{sub}^{\mathrm{CO_2}}=-1035.594648~\mathrm{eV}$
}

\frame
{
	\frametitle{反应中间体模型}
\begin{figure}[h!]
\vskip -0.1in
\centering
\includegraphics[width=3.5in,height=2.0in,viewport=0 0 1528 879, clip]{/home/jun-jiang/Pictures/Screenshot from 2024-07-04 16-32-53.png}
\caption{\tiny \textrm{反应中间体模型.}}%(与文献\cite{EPJB33-47_2003}图1对比)
\label{Substrct_2}
\end{figure}
%\setchemfig{atom sep=2em,bond style={line width=1pt, red, dash pattern=on 2pt off 2pt}}
%\chemname{\chemfig{H-C(-[2]H)-C(=[1]O)-O>H}}{\textrm{Acetaldehyde}}
反应中间体的基态能量 $\mathrm{E}_{sub}^{\mathrm{\chemfig{C(=[1]O)-OH}}}=-1037.7071~\mathrm{eV}$
}

\frame
{
	\frametitle{吸附模型:~基底吸附\textrm{\ch{CO}}}
\begin{figure}[h!]
\vskip -0.1in
\centering
\includegraphics[width=3.5in,height=2.0in,viewport=0 0 1528 879, clip]{/home/jun-jiang/Pictures/Screenshot from 2024-07-04 16-30-46.png}
\caption{\tiny \textrm{基底吸附\textrm{\ch{CO}}模型.}}%(与文献\cite{EPJB33-47_2003}图1对比)
\label{Substrct_3}
\end{figure}
基底吸附\textrm{CO}的基态能量 $\mathrm{E}_{sub}^{\mathrm{CO}}=-1028.222503~\mathrm{eV}$
}

\frame
{
	\frametitle{吸附与反应的热力学评估}
	体系能量的表示
	\begin{displaymath}
		\begin{aligned}
			\mathrm{E}_0=&\mathrm{E}_{sub}+\mathrm{E}_{\mathrm{CO_2}}+\mathrm{E}_\mathrm{H_2}\\
			\mathrm{E}_1=&\mathrm{E}_{sub}^{\mathrm{CO_2}}+\mathrm{E}_\mathrm{H_2}\\
			\mathrm{E}_0=&\mathrm{E}_{sub}^{\mathrm{COOH}}+\frac12\mathrm{E}_\mathrm{H_2}\\
			\mathrm{E}_0=&\mathrm{E}_{sub}^{\mathrm{CO}}+\mathrm{E}_\mathrm{H_2O}
		\end{aligned}
	\end{displaymath}
\begin{minipage}{0.43\textwidth}
	$\mathrm{E}_{\mathrm{CO_2}}=-23.2054~\mathrm{eV}$\\
	$\mathrm{E}_{\mathrm{CO}}=-15.1820~\mathrm{eV}$\\
	$\mathrm{E}_{\mathrm{H_2}}=-6.8~\mathrm{eV}$\\
	$\mathrm{E}_{\mathrm{H_2O}}=-14.227~\mathrm{eV}$
\end{minipage}
\hspace*{1pt}
\begin{minipage}{0.55\textwidth}
\begin{figure}[h!]
\vskip -0.6in
\hspace*{1.5in}
\centering
\includegraphics[width=2.5in,height=1.4in,viewport=0 0 292 167, clip]{/home/jun-jiang/Pictures/Energy.png}
\caption{\tiny \textrm{热力学评估.}}%(与文献\cite{EPJB33-47_2003}图1对比)
\label{Substrct_4}
\end{figure}
\end{minipage}
}
