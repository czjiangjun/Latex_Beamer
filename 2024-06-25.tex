%%%%%%%%%%%%%%%%%%%%%%%%%%%%%%%%%%%%%%%%%%  不使用 authblk 包制作标题  %%%%%%%%%%%%%%%%%%%%%%%%%%%%%%%%%%%%%%%%%%%%%%
%-------------------------------PPT Title-------------------------------------
\title{数据驱动视野下的材料设计}
%-----------------------------------------------------------------------------

%----------------------------Author & Date------------------------------------
%\author[\textrm{Jun\_Jiang}]{姜\;\;骏\inst{}} %[]{} (optional, use only with lots of authors)
%% - Give the names in the same order as the appear in the paper.
%% - Use the \inst{?} command only if the authors have different
%%   affiliation.
\institute[BCC]{\inst{}%
%\institute[Gain~Strong]{\inst{}%
\vskip 0pt 北京市计算中心有限公司}
%\vskip -20pt {\large 格致斯创~科技}}
\date[\today] % (optional, should be abbreviation of conference name)
{	%{\fontsize{6.2pt}{4.2pt}\selectfont{\textcolor{blue}{E-mail:~}\url{jiangjun@bcc.ac.cn}}}
\vskip 45 pt {\fontsize{8.2pt}{6.2pt}\selectfont{%北京科技大学% 报告地点
	\vskip 5 pt \textrm{2024.06}}}
}

%% - Either use conference name or its abbreviation
%% - Not really information to the audience, more for people (including
%%   yourself) who are reading the slides onlin%%   yourself) who are reading the slides onlin%%   yourself) who are reading the slides onlineee
%%%%%%%%%%%%%%%%%%%%%%%%%%%%%%%%%%%%%%%%%%%%%%%%%%%%%%%%%%%%%%%%%%%%%%%%%%%%%%%%%%%%%%%%%%%%%%%%%%%%%%%%%%%%%%%%%%%%%

\subject{}
% This is only inserted into the PDF information catalog. Can be left
% out.
%\maketitle
\frame
{
%	\frametitle{\fontsize{9.5pt}{5.2pt}\selectfont{\textcolor{orange}{“高通量并发式材料计算算法与软件”年度检查}}}
\titlepage
}
%-----------------------------------------------------------------------------

%------------------------------------------------------------------------------列出全文 outline ---------------------------------------------------------------------------------
%\section*{}
%\frame[allowframebreaks]
%{
%  \frametitle{Outline}
%%  \frametitle{\textcolor{mycolor}{\secname}}
%  \tableofcontents%[current,currentsection,currentsubsection]
%}
%%在每个section之前列出全部Outline
%%类似的在每个subsection之前列出全部Outline是\AtBeginSubsection[]
%\AtBeginSection[]
%{
%  \frame<handout:0>%[allowframebreaks]
%  {
%    \frametitle{Outline}
%%全部Outline中,本部分加亮
%    \tableofcontents[current,currentsection]
%  }
%}

%-----------------------------------------------PPT main Body------------------------------------------------------------------------------------
\small
%\section{引言}
\frame
{
	\frametitle{科学研究的范式变更}
\begin{figure}[h!]
\vspace*{-0.28in}
\centering
\includegraphics[height=2.00in,width=4.15in]{Figures/Four_Model_3.png}
%\caption{\tiny \textrm{Pseudopotential for metallic sodium, based on the empty core model and screened by the Thomas-Fermi dielectric function.}}%(与文献\cite{EPJB33-47_2003}图1对比)
\label{Four_Model}
\end{figure}
\begin{minipage}[b]{0.48\textwidth}
 {\fontsize{9.5pt}{6.0pt}\selectfont\begin{itemize}%[+-| alert@+>]
	 \setlength{\itemsep}{6pt}
 \item 逐步趋于理性
 \item 逐步趋于复杂
 \end{itemize}}
\end{minipage}
\hfill
\begin{minipage}[b]{0.48\textwidth}
 {\fontsize{9.5pt}{6.0pt}\selectfont\begin{itemize}%[+-| alert@+>]
	 \setlength{\itemsep}{6pt}
 \item 逐步趋于抽象
 \item 逐步趋于深刻
 \end{itemize}}
\end{minipage}
}

\frame
{
	\frametitle{材料基因工程的理念}
\begin{minipage}[c]{0.31\textwidth}
\begin{itemize}%[+-| alert@+>]
\vspace*{-1.85in}
 {\fontsize{8.5pt}{6.0pt}\selectfont
	 \setlength{\itemsep}{10pt}
 \item 变革研发模式,计算-实验-理论-数据科学相融合: 高效、低耗按需设计
 \item 数据驱动的材料创新平台主要面向复杂材料的模拟}
 \end{itemize}
\end{minipage}
\hfill
\begin{minipage}[b]{0.67\textwidth}
\begin{figure}[h!]
%\vspace*{-0.25in}
\centering
\includegraphics[height=1.60in,width=2.55in]{Figures/Mat_Geno_Ene-1.png}
%\caption{\tiny \textrm{Pseudopotential for metallic sodium, based on the empty core model and screened by the Thomas-Fermi dielectric function.}}%(与文献\cite{EPJB33-47_2003}图1对比)
\label{Mater_Genome}
\end{figure}
\end{minipage}
\vskip 5pt
\textcolor{magenta}{本单位参与国家重点研发计划项目}
\begin{itemize}
	\setlength{\itemsep}{3pt}
	\item {\fontsize{8.2pt}{5.0pt}\selectfont{“\textcolor{blue}{高通量并发式材料计算算法和软件}”}}{\fontsize{6.2pt}{4.2pt}\selectfont{(编号:~\textrm{2017YFB0701500})}}
		%{\fontsize{8.2pt}{6.2pt}\selectfont{(本单位任务责任人)}}
	\item {\fontsize{8.2pt}{5.0pt}\selectfont{“\textcolor{blue}{材料高通量计算/实验平台数据自动汇交技术}”}}{\fontsize{6.2pt}{4.2pt}\selectfont{(编号:~\textrm{2017YFB0704302})}}
		%{\fontsize{8.2pt}{6.2pt}\selectfont{(参与)}}
	\item {\fontsize{8.2pt}{5.0pt}\selectfont{“\textcolor{blue}{国家材料基因工程数据管理与数据服务技术平台}”}}{\fontsize{6.2pt}{4.2pt}\selectfont{(编号:~\textrm{2018YFB0704300})}}
		%{\fontsize{8.2pt}{6.2pt}\selectfont{(参与)}}
\end{itemize}
}

\frame
{
	\frametitle{材料设计的基本思想}
\begin{figure}[h!]
\vspace*{-0.20in}
\centering
\includegraphics[height=0.70in,width=3.75in]{Figures/MGE-2.png}
\includegraphics[height=1.90in,width=3.75in,viewport=-120 0 775 480,clip]{Figures/Multi_Scale-2.jpeg}
%\caption{\tiny \textrm{Pseudopotential for metallic sodium, based on the empty core model and screened by the Thomas-Fermi dielectric function.}}%(与文献\cite{EPJB33-47_2003}图1对比)
\label{MGE}
\end{figure}
{\fontsize{7.2pt}{5.0pt}\selectfont{还原论\textrm{(reductionism)}:\\\textcolor{red}{复杂系统表现的性质(现象),可归结为最基本的组成单元和决定单元行为的基本规律}}}
}

\frame
{
	\frametitle{跨尺度计算:~微观尺度的尝试}
	\textcolor{blue}{将复杂还原为简单的目的是从简单重建复杂}
\begin{itemize}
	\item 量子力学:~原子间相互作用用原子核与电子运动(定量)描述
\begin{figure}[h!]
\vspace*{-0.08in}
\centering
\includegraphics[height=1.05in,width=2.10in,viewport=0 0 350 160,clip]{Figures/H-bonding.jpeg}
%\caption{\tiny \textrm{Schematic illustration of modeling a simple alloy from atoms.}}%(与文献\cite{EPJB33-47_2003}图1对比)
\label{H-bondinfg}
\end{figure}
	\item 分子动力学:~原子间相互作用用力场函数(唯象)描述
\begin{figure}[h!]
\vspace*{-0.10in}
\centering
\includegraphics[height=1.05in,width=2.10in,viewport=0 0 1050 650,clip]{Figures/Chemical_Bonding_2.jpg}
%\caption{\tiny \textrm{Schematic illustration of modeling a simple alloy from atoms.}}%(与文献\cite{EPJB33-47_2003}图1对比)
\label{Chemical_Bonding}
\end{figure}
\end{itemize}

}

\frame
{
	\frametitle{跨尺度计算:~微观尺度的尝试}
%	\textcolor[rgb]{0.00, 1.00, 0.00}{还原论\textrm{(reductionism)}}:~将复杂还原为简单,然后再从简单重建复杂\\
\begin{itemize}
	\item 在每一还原层次,系统特征的空间尺度迅速变小,特征的能量/时间尺度急剧升高
\begin{figure}[h!]
\vspace*{-0.08in}
\centering
\includegraphics[height=1.12in,width=3.85in,viewport=0 0 365 115,clip]{Figures/Alloy_modeling.png}
\caption{\tiny \textrm{Schematic illustration of modeling a simple alloy from atoms.}}%(与文献\cite{EPJB33-47_2003}图1对比)
\label{Alloy_modeling}
\end{figure}
	\item 数值计算的困难程度随着体系尺度的增大而指数增加,从理论上准确预测大量粒子组成体系的性质难度极大
\end{itemize}
}

\frame
{
	\frametitle{跨尺度计算:~微观尺度的尝试}
\begin{figure}[h!]
\vspace*{-0.15in}
\centering
\includegraphics[height=2.25in,width=3.65in,viewport=0 0 1215 822,clip]{Figures/Schematic-illustration-machine_learning_algorithm-to-find-the-relationship-of-the-atomic_configuration-and-energy.png}
%\caption{\tiny \textrm{Schematic illustration of modeling a simple alloy from atoms.}}%(与文献\cite{EPJB33-47_2003}图1对比)
\label{Schematic-illustration-machine_learning_algorithm-to-find-the-relationship-of-the-atomic_configuration-and-energy}
\end{figure}
机器学习方法:~挖掘复杂体系原子结构与(量子力学)能量内在关联
}

\frame
{
	\frametitle{跨尺度计算的主要困难:~\textcolor[rgb]{1.00, 0.20 0.20}{多者异也}}
		\begin{itemize}
		\item 面向尺度和复杂性:~\textrm{Philip~W.~Anderson}提出``多者异也''~\textrm{(\textcolor[rgb]{0.00, 0.00, 1.00}{More is different})}的思想:~
\vskip 3pt
{\fontsize{7.2pt}{5.0pt}\selectfont{复杂体系在每一不同的聚集层次,都会呈现出许多预想不到的全新复杂物理性质,这些性质已经远超出组成基元的物理学规律}}
\begin{figure}[h!]
\vspace*{-0.05in}
\centering
\includegraphics[height=1.35in,width=1.65in,viewport=0 0 425 420,clip]{Figures/Ti-Al_alloy_model.jpg}
\includegraphics[height=1.35in,width=1.65in,viewport=55 10 335 305,clip]{Figures/high-entropy_alloys.jpg}
%\caption{\tiny \textrm{Pseudopotential for metallic sodium, based on the empty core model and screened by the Thomas-Fermi dielectric function.}}%(与文献\cite{EPJB33-47_2003}图1对比)
\label{Flow-of-multi_scale-modelling}
\end{figure}
	\end{itemize}
	{\fontsize{7.2pt}{5.0pt}\selectfont{演生论\textrm{(emergence)}:\\
	\textcolor{red}{在每一个复杂性的发展层次中,都会呈现出全新的物理概念、物理定律和物理原理}}}
}

\frame
{
	\frametitle{数据驱动:~跨尺度计算的曙光}
\begin{figure}[h!]
\vspace*{-0.15in}
\centering
\includegraphics[height=2.65in,width=4.05in,viewport=0 0 1179 721,clip]{Figures/Schematic_flow-of-multi_scale-modelling.png}
%\caption{\tiny \textrm{Pseudopotential for metallic sodium, based on the empty core model and screened by the Thomas-Fermi dielectric function.}}%(与文献\cite{EPJB33-47_2003}图1对比)
\label{Data_for-Machine-Leaning}
\end{figure}
}

\frame
{
	\frametitle{材料计算数据}
\begin{figure}[h!]
\vspace*{-0.15in}
\centering
\includegraphics[height=2.55in,width=4.05in,viewport=0 0 1529 873,clip]{Figures/Database_Materials.png}
%\caption{\tiny \textrm{Pseudopotential for metallic sodium, based on the empty core model and screened by the Thomas-Fermi dielectric function.}}%(与文献\cite{EPJB33-47_2003}图1对比)
\label{DataBase_Materials}
\end{figure}
}

\frame
{
	\frametitle{材料计算数据的硬件基础:~计算资源}
全国高校计算联盟

}
%-----------------------------------------------------------------------------------------------------------------------------------------------------------------------%
\frame
{
	\frametitle{跨尺度计算的需求:~合金材料}
\begin{figure}[h!]
\vspace*{-0.20in}
\centering
\includegraphics[height=2.80in,width=3.35in,viewport=0 0 170 150,clip]{Figures/Multi_Scale-6.jpg}
%\caption{\tiny \textrm{Pseudopotential for metallic sodium, based on the empty core model and screened by the Thomas-Fermi dielectric function.}}%(与文献\cite{EPJB33-47_2003}图1对比)
\label{Alloy-Multi_Scale}
\end{figure}
}

\frame
{
	\frametitle{跨尺度计算的需求:~电池材料}
\begin{figure}[h!]
\vspace*{-0.13in}
\centering
\includegraphics[height=2.50in,width=4.05in,viewport=0 0 224 125,clip]{Figures/Multiple_scales-Battery_cell.jpg}
%\caption{\tiny \textrm{Pseudopotential for metallic sodium, based on the empty core model and screened by the Thomas-Fermi dielectric function.}}%(与文献\cite{EPJB33-47_2003}图1对比)
\label{Battery_Cell-Multi_Scale}
\end{figure}
}

\frame
{
%\documentclass[tikz,svgnames,border={0 2}]{standalone}
%\renewcommand\vec[1]{\boldsymbol{#1}}
%\begin{document}
\begin{tikzpicture}[
    box/.style={rectangle,draw,fill=DarkGray!20,node distance=1cm,text width=15em,text centered,rounded corners,minimum height=2em,thick},
    box1/.style={rectangle,draw,fill=green!50,node distance=1cm,text width=12em,text centered,rounded corners,minimum height=2em,thick},
    box2/.style={rectangle,draw,fill=magenta!75,node distance=1cm,text width=12em,text centered,rounded corners,minimum height=2em,thick},
    box3/.style={rectangle,draw,fill=blue!30,node distance=1cm,text width=12em,text centered,rounded corners,minimum height=2em,thick},
    box4/.style={rectangle,draw,fill=violet!45,node distance=2cm,text width=13em,text centered,rounded corners,minimum height=2em,thick},
%    arrow/.style={draw,-latex', thick},
    arrow/.style={draw,-latex', red, line width=2pt},
  ]
%  \node [box] (potential) {$v_{\text{ext},s}(\vec r)=v_\text{H}(\vec r) + v_\text{xc}(\vec r) + v_\text{ext}(\vec r)$};
  \node [box1,text width=8em] (Softwares) {实际程序\\实现和应用};
%  \node [box,below=0.5 of potential] (hamiltonian) {$\hat{H}_{KS}=-\frac{\hbar^2}{2m}\vec{\nabla}^2 + v_{\text{ext},s}(\vec r)$};
  \node [box2,below=0.5 of Softwares] (Method) {基组和势函数处理方法\\$\vec k$空间积分方案};
%  \node [box,below=0.5 of hamiltonian] (se) {$\hat{H}_{KS} \phi_i(\vec r)= E_i \phi_i(\vec r)$};
  \node [box3,below=0.7 of Method, text width=13em] (Theory) {基本理论\\\textrm{DFT}和\textrm{Band Structure}};
%  \node [box,below=0.5 of se] (density) {$\rho(\vec r)=\sum_{i=1}^n f_i\,|\phi_i(\vec r_i)|^2$};
  \node [box4,below=0.8 of Theory, text width=15em] (Basic-Equ) {(学科)基本方程\\量子力学:~\textrm{Schr\"odinger}方程\\$\hat{\mathbf H}\Psi=E\Psi$};
%  \node [box,below=0.5 of density] (criterion) {Convergence criterion satisfied?};

%  \node [box,above=1.5 of potential, fill=orange!30, text width=20em] (initial) {Supply initial density guess $\rho_\text{ini}(\vec r)$ to Kohn Sham equations};
  \node [box4,left=0.5 of Basic-Equ, fill=violet!30, text width=5em, minimum height=5em] (PDE) {经典数值\\分析算法};
%  \node [box,below=1.5 of criterion, fill=blue!30, text width=20em] (energy) {Use $\rho_\text{fin}(\vec r)$ to minimize total energy functional $E_{V_\text{ext}}[\rho]=T_{e,s}[\phi_i\{\rho\}] + V_{ee,H}[\rho] + E_{xc}[\rho] + V_{eI}[\rho]$};
  \node [box4,left=1.0 of Method, fill=magenta!50, text width=5em] (Algorithm) {具体\\算法\\实现};

  \node [box4,right=0.5 of Basic-Equ, fill=violet!30, text width=5em] (OS) {操作\\系统\\与\\编程\\技术};
  \node [box4,right=1.0 of Method, fill=magenta!50, text width=5em] (coding) {程序语言};
%  \node [box4,right=1.0 of Method, fill=magenta!50, text width=5em, draw=red, thick] at (-5,0) (Figure) {\includegraphics[width=0.5in, height=0.4in]{Figures/Multi-Scale-6.png}};
%  \path [arrow] (initial) -- (potential);
  \path [arrow] (Method) -- (Softwares);
  \path [arrow] (Theory) -- (Method);
  \path [arrow] (Basic-Equ) -- (Theory);
%  \path [arrow] (density) -- (criterion);

  \path [arrow] (PDE) -- (Algorithm);
  \path [arrow] (OS) -- (coding);
 % \path [arrow] (OS) -- (Figure);
  \path [arrow] (coding) |- (Softwares.east);
  \path [arrow] (Algorithm) |- (Softwares.west);
  \path [arrow] (Method.east) -- (coding.west);
  \path [arrow] (Method.west) -- (Algorithm.east);

%%%%%% ------------ 定义 虚线方框转角 -----------------  %%%%%%
%  \path
%  (potential.north west) ++(-1em,1em) coordinate (potential fit)
%  (criterion.south east) ++(1em,-1em) coordinate (criterion fit);

%%%%%% ------------ 绘制 虚线方框 -----------------  %%%%%%
%  \node [rectangle,draw,dashed,inner sep=1em,fit=(potential fit) (criterion fit)] (enclosure) {};
%  \node [above=-0.8em of enclosure,anchor=south,draw,outer sep=0pt,fill=white] (enclosure label) {\Large\textbf{Kohn-Sham method}};

%  \path [arrow] (criterion) -- (energy) node [midway,left=0.1,draw,outer sep=0pt,fill=white] (TextNode) {Yes};
%  \path [draw,thick] (criterion.south) ++(0em,-1em) -- (criterion fit) node [midway,below=0.1,sloped,draw,outer sep=0pt,fill=white] (TextNode) {No};
%  \draw [arrow] (criterion fit) |- (potential.east);
\end{tikzpicture}
}

\frame
{
	\frametitle{材料科学的梦想:~按需设计材料}
\begin{figure}[h!]
\vspace*{-0.18in}
\centering
\includegraphics[height=2.55in,width=4.05in]{Figures/Schematic_Material-Design.png}
\caption{\tiny \textrm{AI for materials:~I~Have~A~Dream.}}%(与文献\cite{EPJB33-47_2003}图1对比)
%\caption{\tiny \textrm{Pseudopotential for metallic sodium, based on the empty core model and screened by the Thomas-Fermi dielectric function.}}%(与文献\cite{EPJB33-47_2003}图1对比)
\label{Schematic_Material-Design}
\end{figure} 
}

