%%%%%%%%%%%%%%%%%%%%%%%%%%%%%%%%%%%%%%%%%%  不使用 authblk 包制作标题  %%%%%%%%%%%%%%%%%%%%%%%%%%%%%%%%%%%%%%%%%%%%%%
%-------------------------------PPT Title-------------------------------------
\title{\rm{Python~OpenAI~API}简介}
%-----------------------------------------------------------------------------

%----------------------------Author & Date------------------------------------
\author[]{\vskip +10pt 姜\;\;骏\inst{}} %[]{} (optional, use only with lots of authors)
%% - Give the names in the same order as the appear in the paper.
%% - Use the \inst{?} command only if the authors have different
%%   affiliation.
\institute[BCC]{\inst{}%
%\institute[Gain~Strong]{\inst{}%
\vskip -15pt 北京市计算中心}
%\vskip -20pt {\large 格致斯创~科技}}
\date[\today] % (optional, should be abbreviation of conference name)
{	{\fontsize{6.2pt}{4.2pt}\selectfont{\textcolor{blue}{E-mail:~}\url{jiangjun@bcc.ac.cn}}}
\vskip 45 pt {\fontsize{8.2pt}{6.2pt}\selectfont{%清华大学\;\;物理系% 报告地点
	\vskip 5 pt \textrm{2025.02}}}
}

%% - Either use conference name or its abbreviation
%% - Not really information to the audience, more for people (including
%%   yourself) who are reading the slides onlin%%   yourself) who are reading the slides onlin%%   yourself) who are reading the slides onlineee
%%%%%%%%%%%%%%%%%%%%%%%%%%%%%%%%%%%%%%%%%%%%%%%%%%%%%%%%%%%%%%%%%%%%%%%%%%%%%%%%%%%%%%%%%%%%%%%%%%%%%%%%%%%%%%%%%%%%%

\subject{}
% This is only inserted into the PDF information catalog. Can be left
% out.
%\maketitle
\frame
{
%	\frametitle{\fontsize{9.5pt}{5.2pt}\selectfont{\textcolor{orange}{“高通量并发式材料计算算法与软件”年度检查}}}
\titlepage
}
%-----------------------------------------------------------------------------

%------------------------------------------------------------------------------列出全文 outline ---------------------------------------------------------------------------------
\section*{}
\frame[allowframebreaks]
{
	\frametitle{\textrm{Outline}}
%  \frametitle{\textcolor{mycolor}{\secname}}
  \tableofcontents%[current,currentsection,currentsubsection]
}
%在每个section之前列出全部Outline
%类似的在每个subsection之前列出全部Outline是\AtBeginSubsection[]
%\AtBeginSection[]
%{
%  \frame<handout:0>%[allowframebreaks]
%  {
%    \frametitle{Outline}
%%全部Outline中,本部分加亮
%    \tableofcontents[current,currentsection]
%  }
%}

%-----------------------------------------------PPT main Body------------------------------------------------------------------------------------
\small
\section{\rm{OpenAI~API}入门}
\begin{frame}{\textrm{OpenAI~API}概述}
    \begin{itemize}
	    \item<1-> \textcolor{magenta}{\textrm{OpenAI API}}\\
		    \textrm{OpenAI}提供的一组接口,开发者可通过编程方式调用其强大的人工智能模型\\
		    {\fontsize{7.5pt}{6.2pt}\selectfont{\textcolor{magenta}{借助该\textrm{API},开发者将先进的自然语言处理能力集成到自己的应用程序、网站或服务中,无需自行从头训练模型}}}
        \item<2-> 主要功能和应用场景
        \begin{itemize}
		\setlength{\itemsep}{5pt}
		\item<2-> \textcolor{blue}{文本生成:}~根据提示生成连贯且相关的文本内容\\
			{\fontsize{7.5pt}{6.2pt}\selectfont{如故事、诗歌、文章等}}
		\item<2-> \textcolor{blue}{问答系统:}~理解用户问题并给出准确回答\\
			{\fontsize{7.5pt}{6.2pt}\selectfont{适用于智能客服等场景}}
		\item<2-> \textcolor{blue}{机器翻译:}~支持不同语言间的文本翻译
		\item<2-> \textcolor{blue}{文本摘要:}~凝炼长篇文本的关键信息,生成简明的摘要
		\item<2-> \textcolor{blue}{情感分析:}~判断文本的情感倾向(积极、消极、中性)
        \end{itemize}
%        \item<3-> 不同模型的特点(如\textrm{GPT-3.5}, \textrm{GPT-4}等)
    \end{itemize}
\end{frame}

%\section{Python 环境准备}
\begin{frame}[fragile]
	\frametitle{\textrm{Python}环境准备}
    % 说明安装 Python 和 OpenAI Python 库的要求
%    \begin{itemize}
%	    \item 安装\textrm{Python}(建议\textrm{Python~3.7}及以上版本)
%        \item 
	安装\textrm{OpenAI~Python}库
%    \end{itemize}
    % 展示安装 OpenAI Python 库的命令
	\begin{block}{使用默认源安装}
    \begin{lstlisting}[style=pythonstyle]
pip install openai
    \end{lstlisting}
    \end{block}
       \begin{block}{使用国内源安装}
%        在国内,
	       由于网络原因,使用默认源安装可能较慢,可使用国内镜像源来加速安装\\
	       \vskip 3pt
	{\fontsize{7.5pt}{6.2pt}\selectfont{以下是使用阿里云源的示例:
        \begin{lstlisting}[style=pythonstyle]
pip install openai -i https://mirrors.aliyun.com/pypi/simple/
\end{lstlisting}}}
%       常用的国内源还有:
%       \begin{itemize}
%           \item 清华大学r:~\url{https://pypi.tuna.tsinghua.edu.cn/simple}
%           \item 中国科学技术大学:~\url{https://pypi.mirrors.ustc.edu.cn/simple}
%       \end{itemize}
    \end{block}
\end{frame}
%\section{API 密钥设置}
%\begin{frame}[fragile]{\textrm{API}的密钥设置}
%    % 介绍获取和设置 OpenAI API 密钥的步骤
%    \begin{itemize}
%	    \item 在\textrm{OpenAI}平台获取 \textrm{API}密钥
%	    \item 在 \textrm{Python}代码中设置\textrm{API}密钥
%    \end{itemize}
%    % 展示在 Python 代码中设置 API 密钥的示例
%    \begin{lstlisting}[style=pythonstyle]
%import openai
%
%openai.api_key = "YOUR_API_KEY"
%    \end{lstlisting}
%\end{frame}
%
\subsection{基本请求示例}
\begin{frame}[fragile]{基本请求示例}
    % 给出一个使用 OpenAI API 生成文本的简单示例
	以下是一个简单的使用\textrm{OpenAI API}生成文本的示例:
    \begin{lstlisting}[style=pythonstyle]
import openai

openai.api_key = "YOUR_API_KEY"

response = openai.Completion.create(
    model="gpt-4.0-turbo",   # model参数用于指定具体的模型名称
#    engine="text-davinci-003", # engine参数通常用于指定使用的GPT模型
    prompt="Once upon a time",
    max_tokens=50
)

print(response.choices[0].text)
    \end{lstlisting}
\end{frame}

%\section{Python 代码实操}
\begin{frame}[fragile]{通过\textrm{openai}接入\textrm{DeepSeep-API}}
        \begin{lstlisting}[style=pythonstyle]
from openai import OpenAI

# 创建OpenAI 客户端
client = OpenAI(api_key="YOUR_API_KEY", base_url="https://api.deepseek.com")

response = client.chat.completions.create(
    model="deepseek-chat",
    messages=[
        {"role": "system", "content": "You are a helpful assistant"},
#        {"role": "user", "content": "Hello"},
        {"role": "user", "content": "介绍一下长城"},
    ],
    stream=False
)

print(response.choices[0].message.content)
        \end{lstlisting}
\end{frame}

\subsection{控制参数调整}
\begin{frame}[fragile, allowframebreaks]{调整参数:~\textrm{temperature}}
    % 解释 temperature 参数的作用和效果,并给出示例代码
    \begin{itemize}
	    \item \texttt{\textcolor{blue}{temperature}}:~控制生成文本的随机性\\
	{\fontsize{7.5pt}{6.2pt}\selectfont{\texttt{temperature}取值范围从 0 到 1:~值越大,生成的文本越随机、越有创意;值越小,生成的文本越确定、越保守}}
        \begin{lstlisting}[style=pythonstyle]
import openai

openai.api_key = "YOUR_API_KEY"

# 低 temperature,生成文本较保守
response_low_temp = openai.Completion.create(
    engine="text-davinci-003",
    prompt="Write a short story about a cat",
    max_tokens=100,
    temperature=0.2
)

# 高 temperature,生成文本较随机
response_high_temp = openai.Completion.create(
    engine="text-davinci-003",
    prompt="Write a short story about a cat",
    max_tokens=100,
    temperature=0.8
)

print("Low temperature result:", response_low_temp.choices[0].text)
print("High temperature result:", response_high_temp.choices[0].text)
        \end{lstlisting}
    \end{itemize}
    \end{frame}

\begin{frame}[fragile, allowframebreaks]{调整参数:~\textrm{top\_p}}
    % 解释 top_p 参数的作用和效果,并给出示例代码
    \begin{itemize}
	    \item \texttt{\textcolor{blue}{top\_p}}:~是另一种控制生成文本随机性的方法,也称为核采样\\
		    {\fontsize{7.5pt}{6.2pt}\selectfont{\texttt{top\_p}会从累积概率超过指定值,如 \textrm{(0.9)},的词中进行采样}}
        \begin{lstlisting}[style=pythonstyle]
import openai

openai.api_key = "YOUR_API_KEY"

# 低 top_p,生成文本较集中
response_low_top_p = openai.Completion.create(
    engine="text-davinci-003",
    prompt="Describe a beautiful sunset",
    max_tokens=100,
    top_p=0.2
)

# 高 top_p,生成文本较分散
response_high_top_p = openai.Completion.create(
    engine="text-davinci-003",
    prompt="Describe a beautiful sunset",
    max_tokens=100,
    top_p=0.8
)

print("Low top_p result:", response_low_top_p.choices[0].text)
print("High top_p result:", response_high_top_p.choices[0].text)
        \end{lstlisting}
    \end{itemize}
\end{frame}

\subsection{错误处理}
\begin{frame}[fragile,allowframebreaks]{错误处理}
%    \begin{itemize}
 %       \item 错误处理:
        \begin{lstlisting}[style=pythonstyle]
try:
    # 发送请求代码
    # 若请求成功,打印生成的文本
        print(response.choices[0].text)
except openai.error.AuthenticationError as e:
    # 处理认证错误,通常是 API 密钥无效
        print(f"认证失败,请检查 API 密钥。错误信息: {e}")
except openai.error.InvalidRequestError as e:
    # 处理无效请求错误,如请求参数错误
        print(f"无效请求。错误信息: {e}")
except openai.error.RateLimitError as e:
    # 处理速率限制错误,即请求频率过高
        print(f"请求速率超过限制。错误信息: {e}")
except openai.error.ServiceUnavailableError as e:
    # 处理服务不可用错误,可能是 OpenAI 服务器问题
        print(f"服务不可用,请稍后再试。错误信息: {e}")
except openai.error.APIConnectionError as e:
    # 处理 API 连接错误,如网络问题
        print(f"API 连接错误,请检查网络。错误信息: {e}")
except openai.error.Timeout as e:
    # 处理请求超时错误
        print(f"请求超时,请稍后再试。错误信息: {e}")
except Exception as e:
    # 处理其他未知错误
        print(f"发生未知错误: {e}")
except requests.RequestException as e:
    print(f"请求发生错误: {e}")
        \end{lstlisting}
  %      \item 注意事项:
  %      \begin{itemize}
%		\item \textrm{API}密钥安全:~保管好\textrm{API}密钥,避免泄露
%		\item \textrm{API}版本和模型差异:~根据不同版本和模型调整请求参数和响应处理方式
%		\item 费用问题:~调用\textrm{API}可能产生费用,关注官方计费规则
%        \end{itemize}
%    \end{itemize}
\end{frame}

\section{\rm{Hugging Face}}
\begin{frame}{\textrm{Hugging Face}简介}
    % 介绍 Hugging Face 平台的基本信息
    \begin{itemize}
	    \item \textrm{Hugging Face}:~专注于自然语言处理\textrm{(NLP)}的开源社区和平台
	    \item 提供丰富的预训练模型、数据集和工具,方便开发者进行\textrm{NLP}任务的开发
    \end{itemize}
    \vskip 5pt
\textrm{Hugging Face}主要功能:
    % 详细说明 Hugging Face 的主要功能
    \begin{itemize}
	    \setlength{\itemsep}{8pt}
	    \item \textcolor{blue}{模型库:}~拥有大量的预训练模型,涵盖多种语言和任务\\
		    {\fontsize{7.5pt}{6.2pt}\selectfont{如文本分类、情感分析、机器翻译等}}
	    \item \textcolor{blue}{数据集:}~提供各种公开的数据集,方便模型训练和评估
	    \item \textcolor{blue}{Transformers 库:}~一个强大的 \textrm{Python}库,用于加载和使用预训练模型,简化模型的使用流程
    \end{itemize}
\end{frame}

\begin{frame}[fragile]{\textrm{Hugging Face}使用示例}
    % 介绍使用 Hugging Face 的步骤和示例代码
    \begin{itemize}
	    \item 安装 \textrm{Transformers}库
        \begin{lstlisting}[style=pythonstyle]
pip install transformers
        \end{lstlisting}
        \item 使用预训练模型进行文本生成示例
        \begin{lstlisting}[style=pythonstyle]
from transformers import pipeline

generator = pipeline('text-generation', model='gpt2')
result = generator("Once upon a time", max_length=50, num_return_sequences=1)
print(result[0]['generated_text'])
        \end{lstlisting}
    \end{itemize}
\end{frame}

\begin{frame}{\textrm{Hugging Face}与\textrm{OpenAI API}对比}
    % 对比 Hugging Face 和 OpenAI API 的开放性、模型多样性和易用性
    \begin{itemize}
	    \setlength{\itemsep}{10pt}
	    \item \textcolor{red}{开放性:} \\
		    {\fontsize{7.5pt}{6.2pt}\selectfont{\textrm{Hugging Face}是开源社区,模型和代码可自由使用和修改}}\\
		    {\fontsize{7.5pt}{6.2pt}\selectfont{\textrm{OpenAI API}部分模型需付费使用,且有使用限制}}
	    \item \textcolor{red}{模型多样性:}\\
		    {\fontsize{7.5pt}{6.2pt}\selectfont{\textrm{Hugging Face}模型库丰富,涵盖多种架构和语言}}\\
		    {\fontsize{7.5pt}{6.2pt}\selectfont{ \textrm{OpenAI}主要以其自身研发的模型为主}}
	    \item \textcolor{red}{易用性:} \\
		    {\fontsize{7.5pt}{6.2pt}\selectfont{\textrm{Hugging Face}的\textrm{Transformers}库简化了模型使用}}\\
		    {\fontsize{7.5pt}{6.2pt}\selectfont{\textrm{OpenAI API}提供了统一的接口,使用方便}}
    \end{itemize}
\end{frame}

%\section{错误处理和最佳实践}
%\begin{frame}{错误处理和最佳实践}
    % 强调处理 API 请求错误和遵循使用政策的重要性
%    \begin{itemize}
%        \item 处理 API 请求中的错误(如网络错误、权限错误等)
%        \item 遵循 OpenAI 和 Hugging Face 的使用政策和最佳实践
%    \end{itemize}
%\end{frame}

%\section{\rm{DeepSeek API}简介}
%\begin{frame}{\textrm{DeepSeek API}简介}
%    \begin{itemize}
%	    \item %\texrm{DeepSeek}是字节跳动研发的前沿大语言模型,
%		    \textrm{DeepSeek API}为开发者提供与模型交互的接口,可将模型能力集成到各类应用中
%        \item 应用场景广泛,如智能客服、内容创作辅助、智能问答系统等
%    \end{itemize}
%%    \includegraphics[width=0.5\textwidth]{客服聊天窗口.jpg} % 替换为实际图片
% %   \includegraphics[width=0.5\textwidth]{文案编辑界面.jpg} % 替换为实际图片
%\end{frame}
%
%%\section{准备工作}
%\begin{frame}[fragile]{准备工作}
%    \begin{itemize}
%	    \item 获取 \textrm{API}密钥:~在\textrm{DeepSeek}官方平台注册账号并获取\textrm{API}密钥
%%        \includegraphics[width=0.8\textwidth]{平台注册与密钥获取页面.jpg} % 替换为实际截图
%	    \item 安装\textrm{requests}库:~用于发送\textrm{HTTP}请求到\textrm{DeepSeek~API}
%        \begin{lstlisting}[style=pythonstyle]
%pip install requests
%        \end{lstlisting}
%    \end{itemize}
%\end{frame}
%
%%\section{Python 代码实操}
%\begin{frame}[fragile]{框架搭建}
%    \begin{itemize}
%        \item 设置 API 端点和 API 密钥:
%        \begin{lstlisting}[style=pythonstyle]
%API_ENDPOINT = "https://api.deepseek.com/v1/chat/completions"
%API_KEY = "your_api_key"
%        \end{lstlisting}
%        \item 设置请求头:
%        \begin{lstlisting}[style=pythonstyle]
%headers = {
%    "Content-Type": "application/json",
%    "Authorization": f"Bearer {API_KEY}"
%}
%        \end{lstlisting}
%    \end{itemize}
%\end{frame}
%
%\begin{frame}[fragile]{\textrm{Python}请求体设置与发送请求}
%    \begin{itemize}
%        \item 设置请求体:
%        \begin{lstlisting}[style=pythonstyle]
%data = {
%    "model": "deepseek-chat",
%    "messages": [
%        {
%            "role": "user",
%            "content": "介绍一下长城"
%        }
%    ]
%}
%        \end{lstlisting}
%\item 发送\textrm{POST}请求:
%        \begin{lstlisting}[style=pythonstyle]
%response = requests.post(API_ENDPOINT, headers=headers, data=json.dumps(data))
%        \end{lstlisting}
%    \end{itemize}
%\end{frame}
%
%\begin{frame}[fragile]{处理响应}
%    \begin{itemize}
%        \item 检查响应状态码:
%        \begin{lstlisting}[style=pythonstyle]
%if response.status_code == 200:
%    result = response.json()
%    print(result["choices"][0]["message"]["content"])
%elif response.status_code == 402:
%        error_info = response.json()
%        print(f"账户余额不足,请充值。错误信息: {error_info['error']['message']}")
%else:
%    print(f"请求失败,状态码: {response.status_code},错误信息: {response.text}")
%        \end{lstlisting}
%    \end{itemize}
%\end{frame}

%\section{总结}
%\begin{frame}{总结}
%    % 总结 Python 与 OpenAI API、Hugging Face 的使用要点并鼓励探索
%    \begin{itemize}
%        \item 回顾 \textrm{Python} 与 \textrm{OpenAI~API}、\textrm{Hugging Face} 的使用要点
%        \item 鼓励进一步探索和实践
%    \end{itemize}
%\end{frame}
