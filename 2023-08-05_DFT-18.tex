%%%%%%%%%%%%%%%%%%%%%%%%%%%%%%%%%%%%%%%%%%  不使用 authblk 包制作标题  %%%%%%%%%%%%%%%%%%%%%%%%%%%%%%%%%%%%%%%%%%%%%%
%-------------------------------PPT Title-------------------------------------
\title{18-晶格振动与\rm{AIMD}的分类}
%-----------------------------------------------------------------------------
%----------------------------Author & Date------------------------------------

%\author[\textrm{Jun\_Jiang}]{姜\;\;骏\inst{}} %[]{} (optional, use only with lots of authors)
%% - Give the names in the same order as the appear in the paper.
%% - Use the \inst{?} command only if the authors have different
%%   affiliation.
%\institute[BCC]{\inst{}%
\institute[Gain~Strong]{\inst{}%
%\vskip -20pt 北京市计算中心}
\vskip -20pt {\large 格致斯创~科技}}
\date[\today] % (optional, should be abbreviation of conference name)
{%	{\fontsize{6.2pt}{4.2pt}\selectfont{\textcolor{blue}{E-mail:~}\url{jiangjun@bcc.ac.cn}}}
\vskip 45 pt {\fontsize{8.2pt}{6.2pt}\selectfont{%清华大学\;\;物理系% 报告地点
	\vskip 5 pt \textrm{2023.08.19}}}
}

%% - Either use conference name or its abbreviation
%% - Not really information to the audience, more for people (including
%%   yourself) who are reading the slides onlin%%   yourself) who are reading the slides onlin%%   yourself) who are reading the slides onlineee
%%%%%%%%%%%%%%%%%%%%%%%%%%%%%%%%%%%%%%%%%%%%%%%%%%%%%%%%%%%%%%%%%%%%%%%%%%%%%%%%%%%%%%%%%%%%%%%%%%%%%%%%%%%%%%%%%%%%%

\subject{}
% This is only inserted into the PDF information catalog. Can be left
% out.
%\maketitle
\frame
{
%	\frametitle{\fontsize{9.5pt}{5.2pt}\selectfont{\textcolor{orange}{“高通量并发式材料计算算法与软件”年度检查}}}
\titlepage
}
%-----------------------------------------------------------------------------

%------------------------------------------------------------------------------列出全文 outline ---------------------------------------------------------------------------------
%\section*{}
%\frame[allowframebreaks]
%{
%  \frametitle{Outline}
%%  \frametitle{\textcolor{mycolor}{\secname}}
%  \tableofcontents%[current,currentsection,currentsubsection]
%}
%%在每个section之前列出全部Outline
%%类似的在每个subsection之前列出全部Outline是\AtBeginSubsection[]
%\AtBeginSection[]
%{
%  \frame<handout:0>%[allowframebreaks]
%  {
%    \frametitle{Outline}
%%全部Outline中,本部分加亮
%    \tableofcontents[current,currentsection]
%  }
%}

%-----------------------------------------------PPT main Body------------------------------------------------------------------------------------
\small
%\section{\rm{VASP~}软件中\rm{PAW~}计算的实现}
%\frame
%
%	\frametitle{\textrm{VASP}计算的特色}
%	相比于与普通的第一原理计算软件,\textrm{VASP}很好地平衡了计算效率和精度的问题,总的来说,\textrm{VASP}主要通过这几个特色保证了计算的高效能
%	\begin{itemize}
%	     \item 迭代与优化算法的多样性\\
%		     本质上电荷密度迭代 \textrm{\&\&} 体系总能量优化是相同的优化问题,采用了类似的算法\upcite{CMS6-15_1996,PRB54-11169_1996}:\\
%			\textcolor{blue}{\textrm{Pseudo-Newton、Conjugate-Gradient、Broyden~mix、damping-factor、RMM-DIIS}}
%	     \item 尽可能采用局域基(原子轨道基)函数:~\\
%		     \textcolor{blue}{\textrm{LREAL}}=\textcolor{red}{\textrm{.TRUE.}}\\
%			优化的投影函数也尽可能在实空间表示
%	     \item \textrm{PAW}原子数据集:\textcolor{blue}{优异的赝势}\upcite{PRB59-1758_1999}
%	\end{itemize}
%}
\section{晶格振动与声子}
\subsection{晶格振动与简谐振动}
\frame
{
	\frametitle{晶格振动}
		晶体中的格点表示原子的平衡位置,晶格振动是原子在格点附近的振动
\begin{figure}[h!]
\centering
%\hspace*{-10pt}
\vspace*{-0.1in}
\includegraphics[height=1.9in,width=4.0in,viewport=0 0 400 185,clip]{Figures/Schematic-diagrams-of-the-crystal-structures-and-longitudinal-optical-lattice-vibration-modes-(LO_modes)-of-diamonds.jpg}
\caption{\tiny \textrm{Schematic diagrams of the crystal structures and longitudinal optical lattice vibration modes (LO modes) of diamonds. (a) Crystal structure of a diamond. (b) 3D view of a LO mode. (c) Top view of a LO mode.}}%
\label{lattice-virbration}
\end{figure} 
}

\frame
{
	\frametitle{简谐近似}
	\begin{itemize}
		\item 红外、\textrm{Raman}光谱、中子衍射谱,热容、热导,电阻、超导和电-声耦合等都与晶格振动有关
		\item 绝热近似%(\textrm{Born-Oppenheimer}近似)
			下,原子核是在电子能量函数$E(\mathbf{R})$构成的势能面上运动
	\end{itemize}
	含有$N$个原子,平衡位置是$\mathbf{R}_i^0$,偏移位置矢量$\mathbf{\mu}_i(t)$,体系的势能函数在平衡位置作\textrm{Taylor~}级数展开
	\begin{displaymath}
%		\begin{aligned}
		V=V_0+\sum_{i=1}^{3N}\left( \frac{\partial V}{\partial \mu_i} \right)_0\mu_i+\underline{\textcolor{red}{\frac12\sum_{i,j=1}^{3N}\left( \frac{\partial^2V}{\partial\mu_i\partial\mu_j} \right)_0\mu_i\mu_j}}+\mbox{高阶项}
%		\end{aligned}
	\end{displaymath}
	平衡位置$\left( \frac{\partial V}{\partial\mu_i} \right)_0=0$\\
	\textcolor{blue}{简谐近似}保留到$\mu_i$的二次项\\
	引入高阶项,则势函数可以包括\textcolor{purple}{非简谐近似}的贡献
}

\frame
{
	\frametitle{简谐振动与简正坐标}
	$N$原子体系的动能函数
	\begin{displaymath}
		T=\frac12\sum_{i=1}^{3N}m_i\dot{\mu}_i^2
	\end{displaymath}
	引入简正坐标,\textcolor{blue}{与原子位移坐标$\mu_i$正交变换}
	\begin{displaymath}
		\sqrt{m_i}\mu_i=\sum_{j=1}^{3N}a_{ij}Q_j
	\end{displaymath}
	\textcolor{red}{目的}:~系统的势能函数与动能函数有简单形式(只有平方项)
	\begin{displaymath}
%		\begin{aligned}
			T=\frac12\sum_{i=1}^{3N}\dot{Q}_i^2\quad
			V=\frac12\sum_{i=1}^{3N}\omega_i^2Q_i^2
%		\end{aligned}
	\end{displaymath}
	由此可得谐振方程
	\begin{displaymath}
		\ddot{Q}_i+\omega_i^2Q_i=0\quad i=1,2,3,\cdots,3N
	\end{displaymath}

}

\frame
{
	\frametitle{简谐振动与振动模式}
	任意简正坐标解
	\begin{displaymath}
		Q_i=A\sin(\omega_it+\delta)
	\end{displaymath}
	由此得到原子位移坐标
	\begin{displaymath}
		\mu_i=\frac{a_{ij}}{\sqrt{m_i}}A\sin(\omega_it+\delta)
	\end{displaymath}
\begin{figure}[h!]
\centering
%\hspace*{-10pt}
\vspace*{-0.1in}
\includegraphics[height=1.in,width=2.in,viewport=0 20 420 250,clip]{Figures/RF_vir.jpg}
\caption{\tiny \textrm{Schematic example of vibration model of dimethyl.}}%
\label{virbration_model}
\end{figure} 
\textcolor{red}{简谐振动不表示某个原子的振动,表示整个体系所有原子参与的振动。这种体系中所有原子一起参加的集体运动常称为振动模}
}

\frame
{
	\frametitle{简谐振动与振动模式}
\begin{figure}[h!]
\centering
%\hspace*{-10pt}
\vspace*{-0.05in}
\includegraphics[height=1.1in,width=1.95in,viewport=0 0 520 310,clip]{Figures/Viberation_of_B1u.png}
\includegraphics[height=1.1in,width=1.95in,viewport=0 0 515 310,clip]{Figures/Viberation_of_A1g.png}
\vskip 4pt
\includegraphics[height=1.1in,width=1.95in,viewport=0 0 520 310,clip]{Figures/Viberation_of_B3g.png}
\includegraphics[height=1.1in,width=1.95in,viewport=0 0 515 310,clip]{Figures/Viberation_of_B2u.png}
\caption{\tiny \textrm{Schematic example of the symmetry of vibration model.}}%
\label{virbration_model-symmetry}
\end{figure} 
}

\frame
{
	\frametitle{一维单原子链}
\begin{figure}[h!]
\centering
%\hspace*{-10pt}
\vspace*{-0.25in}
\includegraphics[height=1.0in,width=2.8in,viewport=0 0 1400 500,clip]{Figures/virbration.png}
\caption{\tiny \textrm{Schematic example of vibration of 1D-atomic chain.}}%
\label{virbration}
\end{figure} 
单原子链可以视为最简单的晶格,平衡时相邻原子距离为$\mathbf{a}$,原子限制在沿链方向运动,偏离格点位置用$\cdots,\mathbf{X}_{n-1},\mathbf{X}_{n},\mathbf{X}_{n+1},\cdots$,原子的振动可以表示为
\begin{displaymath}
	\mu_{nq}=A\mathrm{e}^{\mathrm{i}(\omega t-qx)}
\end{displaymath}
其中振幅$A$是常数,$\omega$是圆频率,$q=\tfrac{2\pi}{\lambda}$是波数,$\lambda$是波长
\vskip 5pt
\textcolor{blue}{根据量子理论,每种简谐振动的能量是量子化的,可以用声子表示}
\begin{displaymath}
	\varepsilon_{nq}=\left( n+\frac12 \right)\hbar\omega_q
\end{displaymath}
}

\frame
{
	\frametitle{谐振子模型}
\begin{figure}[h!]
\centering
%\hspace*{-10pt}
\vspace*{-0.15in}
\includegraphics[height=2.7in,width=2.9in,viewport=0 0 650 800,clip]{Figures/Quantum-viberation.png}
\caption{\tiny \textrm{Schematic example of quantization of  harmonic oscillator model.}}%
\label{Harmonic-oscillator-model}
\end{figure} 
}

\frame
{
	\frametitle{双原子链与光学支和声学支}
\begin{figure}[h!]
\centering
%\hspace*{-10pt}
\vspace*{-0.20in}
\includegraphics[height=0.7in,width=2.6in,viewport=0 0 1400 400,clip]{Figures/virbration-2.png}
\caption{\tiny \textrm{Schematic example of vibration of 1D-diatomic chain.}}%
\label{virbration-2D}
\end{figure} 
一维双原子链是最简单的复式晶格,平衡时相邻原子间距为$\mathbf{a}$,每个原胞含有两个不同原子\textrm{P}和\textrm{Q},质量分别是$m$和$M$,原子现在在沿链方向运动,偏离位移用$\cdots,\mu_{2n},\mu_{2n+1},\cdots$\\原子的运动方程
\begin{displaymath}
	\begin{aligned}
		&\mbox{\textrm{P}原子:~}m\ddot{\mu}_{2n}=-\beta(2\mu_{2n}-\mu_{2n+1}-\mu_{2n-1})\\
		&\mbox{\textrm{Q}原子:~}M\ddot{\mu}_{2n+1}=-\beta(2\mu_{2n+1}-\mu_{2n+2}-\mu_{2n})
	\end{aligned}
\end{displaymath}
可得关于振动频率$\omega$的两组解
\begin{displaymath}
	\omega^2\left.
	\begin{aligned}
		&\nearrow\omega_+^2\\
		&\searrow\omega_-^2
	\end{aligned}\right\}
	=\beta\frac{m+M}{mM}\left\{ 1\pm\left[ 1-\frac{4mM}{(m+M)^2}\sin^2aq \right]^{1/2} \right\}
\end{displaymath}
}

\frame
{
	\frametitle{光学支和声学支的长波极限}
\begin{figure}[h!]
\begin{minipage}[t]{0.3\linewidth}
\centering
\vspace*{-0.3in}
%\hspace*{-10pt}
\includegraphics[height=1.in,width=1.in,viewport=0 0 700 800,clip]{Figures/Optic-Acous.png}
\label{optic_acous}
\end{minipage}
\hfill
\begin{minipage}[t]{0.67\linewidth}
%\vspace*{-0.3in}
	\begin{itemize}
		\item \textcolor{blue}{光学支}:~属于频率$\omega_+$的晶格简谐振动
		\item \textcolor{blue}{声学支}:~属于频率$\omega_-$的晶格简谐振动
	\end{itemize}
\end{minipage}
\caption{\tiny \textrm{The acoustic branch and optical branch.}}%
\end{figure} 
声学支的长波极限($q\rightarrow0$):
\begin{displaymath}
	\omega_-\approx a\sqrt{\frac{2\beta}{m+M}}q\quad\mbox{\textcolor{blue}{一维链看成连续介质的弹性波}}
\end{displaymath}
光学支的长波极限($q\rightarrow0$):
\begin{displaymath}
	\omega_+\approx a\sqrt{\frac{2\beta}{\left( \frac{mM}{m+M} \right)}}\quad\mbox{\textcolor{blue}{两种原子具有相反的相位,质心保持不动}}
\end{displaymath}
}

\frame
{
	\frametitle{声子振动模式的横向传播与纵向传播}
\begin{figure}[h!]
\centering
%\hspace*{-10pt}
\vspace*{-0.30in}
\includegraphics[height=2.5in,width=4.0in,viewport=0 0 850 610,clip]{Figures/examples-of-transverse-optical-to-and-longitudinal-optical-lo-phonons-in-1d.png}
\caption{\tiny \textrm{Schematic examples of transverse optical to and longitudinal optical phonons in 1d diatomic lattice.}}%
\label{To-Lo-1D}
\end{figure} 
}

\frame
{
	\frametitle{声学支和光学支的长波极限}
\begin{figure}[h!]
\centering
%\hspace*{-10pt}
\vspace*{-0.10in}
\includegraphics[height=2.3in,width=4.0in,viewport=0 0 1210 650,clip]{Figures/Phonon-Acoutic-Optic.png}
\caption{\tiny \textrm{Representation of the difference between acoustic and optic modes in the limit of wave vector $\vec q\rightarrow0$ for the model diatomic chain: acoustic (in-phase) and optic (out-of-phase) modes.}}%
\label{Aucous-Optic}
\end{figure} 
}

\frame
{
	\frametitle{经典三维振动模式}
	位于$\mathbf{R}_I(t)$的原子核运动的经典力学描述
			\begin{displaymath}
				M_I\frac{\partial^2\mathbf{R}_I}{\partial t^2}=\vec F_I(\mathbf{R})=-\frac{\partial}{\partial\mathbf{R}_I}E(\mathbf{R})
			\end{displaymath}
			晶格平衡位置$\{\mathbf{R}_I^0\}=\mathbf R^0$由原子核受力平衡确定
			\begin{displaymath}
				\vec F_I(\mathbf R^0)=0
			\end{displaymath}
			\textcolor{blue}{对平衡位置偏移的受力方程为}
			\begin{displaymath}
				C_{I,\alpha;J,\beta}=\frac{\partial^2E(\mathbf{R})}{\partial\mathbf{R}_{I,\alpha}\partial\mathbf{R}_{J,\beta}}
			\end{displaymath}
			其中$\alpha,\beta\cdots$是\textrm{cartesian}坐标

			\textcolor{blue}{谐振子近似下},频率为$\omega$的谐振模式下,晶格对平移位置的偏移为
			\begin{displaymath}
				\mathbf{u}_I(t)=\mathbf{R}_I(t)-\mathbf{R}_I^0\equiv\mathbf{u}_I\mathrm{e}^{\mathrm{i}\omega t}
			\end{displaymath}
}

\frame
{
	\frametitle{三维晶格振动模式}
			对位于$I$的原子核(质量为$M_I$),有
			\begin{displaymath}
				-\omega^2M_Iu_{I\alpha}=-\sum_{J\beta}C_{I,\alpha;J\beta}u_{J\beta}
			\end{displaymath}
			因此振动频率$\omega$,由经典谐振方程确定
%			\begin{displaymath}
%				\det\left|\frac1{\sqrt{M_IM_J}}C_{I,\alpha;J\beta}-\omega^2\right|=0
%			\end{displaymath}
%	对于周期性的晶格振动,根据\textrm{Bl\"och~}定理,振动引起的位置偏移
%			\begin{displaymath}
%				\mathbf{u}_s(\vec T_n)\equiv\mathbf{R}_s(\vec T_n)-\mathbf{R}_s^0(\vec T_n)=\mathrm{e}^{\mathrm{i}\vec k\cdot\vec T_n}\mathbf{u}_s(\vec k)
%			\end{displaymath}
%			由此得谐振方程
			\begin{displaymath}
				\det\left|\frac1{\sqrt{M_sM_{s^{\prime}}}}C_{s,\alpha;s^{\prime}\alpha^{\prime}}-\omega_{i\vec k}^2\right|=0
			\end{displaymath}
			这里原子标记$s=1,\cdots,s$,对应的谐振模式$i=1,\cdots,3s$

			每个$\vec k$的约化力常数矩阵可表示为
			\begin{displaymath}
				\begin{aligned}
				C_{s,\alpha;s^{\prime}\alpha^{\prime}}(\vec k)=&\sum_{\vec T_n}\mathrm{e}^{\mathrm{i}\vec k\cdot\vec T_n}\frac{\partial^2 E(\mathbf{R})}{\partial\mathbf{R}_{s,\alpha}(0)\partial\mathbf{R}_{s^{\prime},\alpha^{\prime}}(\vec T_n)}\\
				=&\frac{\partial^2E(\mathbf{R})}{\partial\mathbf{u}_{s,\alpha}(\vec k)\partial\mathbf{u}_{s^{\prime},\alpha^{\prime}}(\vec k)} 
				\end{aligned}
			\end{displaymath}
}

\section{绝热近似:~\rm{Hellmann-Feynman}定理与电-声耦合}
\frame
{
	\frametitle{绝热近似}
	绝热近似下,电子-原子核的运动能量上完全分离,原子核运动的影响,在电子的本征态(本征值$E_i\{\mathbf{R}\}$,波函数$\Psi_i(\{\mathbf{r}\};\{\mathbf{R}\})$)中表现为含有原子核位置参数$\{\mathbf{R}\}$

	如果考虑核与电子体系,\textrm{Hamiltonian~}算符可以写成
	\begin{displaymath}
		\hat H=\hat T_N+\hat T_e+\hat U
	\end{displaymath}
	$U$是全部相互作用,可由\textcolor{blue}{电子坐标$\{\mathbf{r}\}$}和\textcolor{blue}{原子核位置$\{\mathbf{R}\}$}表示

	电子运动的本征态由下式确定
	\begin{displaymath}
		\hat H(\mathbf{R})\psi_s(\{\mathbf{r};\mathbf{R}\})=E_s(\mathbf{R})\psi_s(\{\mathbf{r};\mathbf{R}\})
	\end{displaymath}
	这里$s=1,2,3,\cdots$
	电子运动\textrm{Hamiltonian}可由本征态表示为
	\begin{displaymath}
		\langle\psi_m(\mathbf{r};\mathbf{R})|\hat H(\mathbf{R})|\psi_n(\mathbf{r};\mathbf{R})\rangle=E_s(\mathbf{R})\delta_{mn}
	\end{displaymath}
	这里利用了电子本征态波函数的正交关系
	\begin{displaymath}
		\langle\psi_m(\mathbf{r};\mathbf{R})|\psi_n(\mathbf{r};\mathbf{R})\rangle=\delta_{mn}
	\end{displaymath}
}

\frame
{
	\frametitle{\textrm{Hellmann-Feynman}定理}
	根据电子波函数的正交关系,电子波函数对参数$\mathbf{R}$改变的响应有
	\begin{displaymath}
		\langle\dfrac{\partial}{\partial{\mathbf{R}_i}}\psi_m(\mathbf{r};\mathbf{R})|\psi_n(\mathbf{r};\mathbf{R})\rangle=-\langle\psi_m(\mathbf{r};\mathbf{R})|\dfrac{\partial}{\partial{\mathbf{R}_i}}\psi_n(\mathbf{r};\mathbf{R})\rangle
	\end{displaymath}
	对$m=n$有
	\textcolor{blue}{
		\begin{displaymath}
			\langle\psi_m(\mathbf{r};\mathbf{R})|\dfrac{\partial}{\partial{\mathbf{R}_i}}\psi_m(\mathbf{r};\mathbf{R})\rangle=\mbox{纯虚数}
		\end{displaymath} }
		类似地,\textrm{Schr\"odinger}方程对参数$\mathbf{R}$改变的响应有
		\begin{displaymath}
			\langle\dfrac{\partial\psi_m}{\partial{\mathbf{R}_i}}|H|\psi_n\rangle+\langle\psi_m|\dfrac{\partial H}{\partial{\mathbf{R}_i}}|\psi_n\rangle+\langle\psi_m|H|\dfrac{\partial\psi_n}{\partial{\mathbf{R}_i}}\rangle=\dfrac{\partial E_n}{\partial\mathbf{R}_i}\delta_{mn}
		\end{displaymath}
		注意到$\psi_m(\mathbf{r};\mathbf{R})$和$\psi_n(\mathbf{r};\mathbf{R})$是$H(\mathbf{R})$的本征态,本征值分别是$E_m(\mathbf{R})$和$E_n(\mathbf{R})$,因此有
		\begin{displaymath}
			\langle\psi_m|\dfrac{\partial H}{\partial{\mathbf{R}_i}}|\psi_n\rangle+[E_m-E_n]\langle\psi_m|H|\dfrac{\partial\psi_n}{\partial{\mathbf{R}_i}}\rangle=\dfrac{\partial E_n}{\partial\mathbf{R}_i}\delta_{mn}
		\end{displaymath}
}

\frame
{
	\frametitle{\textrm{Hellmann-Feynman}定理}
	当$m=n$时有\textrm{Hellmann-Feynman}定理
	\textcolor{purple}{
	\begin{displaymath}
		\langle\psi_m(\mathbf{r};\mathbf{R})|\dfrac{\partial H(\mathbf{R})}{\partial{\mathbf{R}_i}}|\psi_m(\mathbf{r};\mathbf{R})\rangle=\dfrac{\partial E_m(\mathbf{R})}{\partial\mathbf{R}_i}
	\end{displaymath}}
	\textrm{Hellmann-Feynman}定理非常重要,因为它表明,对于确定的势能面,能量对位置导数(广义力)可通过波函数和算符$\partial H(\mathbf{R})/\partial\mathbf{R}_i$的期望值计算得到
	\vskip 5pt
	一般地,当势能面不包含任何简并态时,可以有\textrm{Epstein}广义\textrm{Hellmann-Feynman}定理
	\begin{displaymath}
		\hspace*{-10pt}
		\langle\psi_m(\mathbf{r};\mathbf{R})|\dfrac{\partial}{\partial{\mathbf{R}_i}}\psi_n(\mathbf{r};\mathbf{R})\rangle=\dfrac1{E_n(\mathbf{R})-E_m(\mathbf{R})}\langle\psi_m(\mathbf{r};\mathbf{R})|\dfrac{\partial H(\mathbf{R})}{\partial{\mathbf{R}_i}}|\psi_n(\mathbf{r};\mathbf{R})\rangle
	\end{displaymath}
}

\frame
{
	\frametitle{绝热近似下的原子受力}
	根据\textrm{Hellmann-Feynman}定理,在\textrm{Born-Oppenheimer}近似下,位于$\mathbf{R}_K$处的原子核的受力
	\begin{displaymath}
		\langle\psi(\mathbf{r};\mathbf{R})|\dfrac{\partial}{\partial{\mathbf{R}_K}}H_e(\mathbf{R})|\psi(\mathbf{r};\mathbf{R})\rangle=\dfrac{\partial E(\mathbf{R})}{\partial\mathbf{R}_K}
	\end{displaymath}
	这里\textrm{Hamiltonian}的梯度为
	\begin{displaymath}
		\dfrac{\partial}{\partial{\mathbf{R}_K}}H_e(\mathbf{R})=-\dfrac{\partial}{\partial{\mathbf{R}_K}}\sum_i\dfrac{Z_Ke^2}{\mathbf{r}_i-\mathbf{R}_K}+\dfrac{\partial}{\partial{\mathbf{R}_K}}\sum_{I(\neq K)}\dfrac{Z_IZ_Ke^2}{\mathbf{R}_I-\mathbf{R}_K}
	\end{displaymath}
	由此,原子受力可表示为
	\textcolor{red}{
	\begin{displaymath}
		\vec F_K=-\dfrac{\partial E}{\partial\mathbf{R}_K}=\int n(\mathbf{r};\mathbf{R})\dfrac{\partial}{\partial{\mathbf{R}_K}}\dfrac{Z_Ke^2}{\mathbf{r}_i-\mathbf{R}_K}\mathrm{d}\mathbf{r}-\dfrac{\partial}{\partial{\mathbf{R}_K}}\sum_{I(\neq K)}\dfrac{Z_IZ_Ke^2}{\mathbf{R}_I-\mathbf{R}_K}
	\end{displaymath}}
	原子核受力:~\textcolor{blue}{其余原子核的经典静电排斥}和\textcolor{purple}{电子的电荷密度分布}
}

\frame
{
	\frametitle{绝热近似下的核运动}
	绝热近似下,原子核-电子的波函数可以表示为
	\begin{displaymath}
		\Psi(\{\mathbf{r};\mathbf{R}\})=\sum_i\chi_{si}(\{\mathbf{R}\})\psi_i(\{\mathbf{r}\};\{\mathbf{R}\})
	\end{displaymath}
	原子核$\chi_{si}(\{\mathbf{R}\})$在电子形成的势能面$E_i(\{\mathbf{R}\})$上的运动方程
	\begin{displaymath}
		[T_N+E_i(\{\mathbf{R}\})-E_s]\chi_{si}(\{\mathbf{R}\})=-\sum_{ii^{\prime}}C_{ii^{\prime}}\chi_{si}(\{\mathbf{R}\})
	\end{displaymath}
	这里$T_n=-\frac12(\sum\limits_J\nabla_J^2/M_J)$,矩阵元$C_{ii^{\prime}}=A_{ii^{\prime}}+B_{ii^{\prime}}$
	\begin{displaymath}
		\begin{aligned}
			A_{ii^{\prime}}(\{\mathbf{R}\})=&\sum_J\frac1{M_J}\langle\psi_i(\{\mathbf{r}\};\{\mathbf{R}\})|\nabla_J|\psi_{i^{\prime}}(\{\mathbf{r}\};\{\mathbf{R}\})\rangle\nabla_J\\
			B_{ii^{\prime}}(\{\mathbf{R}\})=&\sum_J\frac1{2M_J}\langle\psi_i(\{\mathbf{r}\};\{\mathbf{R}\})|\nabla_J^2|\psi_{i^{\prime}}(\{\mathbf{r}\};\{\mathbf{R}\})\rangle\\
		\end{aligned}
	\end{displaymath}
	这里$\langle\psi_i(\{\mathbf{r}\};\{\mathbf{R}\})|\hat O|\psi_{i^{\prime}}(\{\mathbf{r}\};\{\mathbf{R}\})\rangle$表示对电子变量$\{\mathbf{r}\}$积分
}

\frame
{
	\frametitle{绝热近似下的核运动}
	绝热近似下,\textcolor{red}{将忽略矩阵$C_{ii^{\prime}}$的全部非对角元},可有
	\begin{itemize}
		\item \textcolor{blue}{电子能及时响应原子核的运动}
		\item \textcolor{blue}{电子由态$i\rightarrow i^{\prime}$的激发,不会影响原子核位置变量${\{\mathbf{R}\}}$}
		\item $A_{ii^{\prime}}=0$(波函数归一化要求)
		\item 核运动的势函数$U_i(\{\mathbf{R}\})=E_i(\{\mathbf{R}\})+B_{ii}(\{\mathbf{R}\})$
	\end{itemize}
	核运动方程
	\begin{displaymath}
		\left[ -\sum_J\frac1{2M_J}\nabla_J^2+U_i(\{\mathbf{R}\})-E_{ni} \right]\chi_{ni}(\{\mathbf{R}\})=0
	\end{displaymath}
这里$n=1,2,3,\cdots$

如果忽略$B_{ii}$的贡献,即冻声子近似(\textrm{frozen phonon})或微扰方法
}

\frame
{
	\frametitle{电-声耦合}
	电子-声子的来源:~\textcolor{blue}{$C_{ii^{\prime}}$的非对角元部分}
	\begin{itemize}
		\item $C_{ii^{\prime}}$的非对角元部分描述了\textcolor{red}{原子核运动(振动)引起电子在不同态间跃迁}
		\item $C_{ii^{\prime}}$的非对角元部分主要来自$A_{ii^{\prime}}$
			\begin{enumerate}
\setlength{\itemsep}{7pt}
				\item 电子波函数$\psi_i(\{\mathbf{r}\};\{\mathbf{R}\})$对原子核位置$\{\mathbf{R}_j\}$的梯度
				\item 梯度算符对原子核波函数$\chi_{si}(\{\mathbf{R}\})$的贡献
			\end{enumerate}
		\item 电子在态$i\rightarrow i^{\prime}$跃迁将会激发或吸收一个声子
	\end{itemize}
	根据\textrm{Epstein}推广的\textrm{Hellmann-Feynman}定理线性近似下有
	\begin{displaymath}
		\hspace*{-15pt}
		\langle\psi_i(\{\mathbf{r}\};\{\mathbf{R}\})|\nabla_J|\psi_i^{\prime}(\{\mathbf{r}\};\{\mathbf{R}\})\rangle=\frac{\langle\psi_i(\{\mathbf{r}\};\{\mathbf{R}\})|\tfrac{\nabla_V}{\nabla_{\mathbf{R}_J}}|\psi_{i^{\prime}}(\{\mathbf{r}\}:\{\mathbf{R}\})\rangle}{E_{i^{\prime}}(\{\mathbf{R}\})-E_i(\{\mathbf{R}\})}
	\end{displaymath}
}

\section{第一原理分子动力学}
\frame
{
	\frametitle{分子动力学\textrm{(MD)}}
	分子动力学\textrm{(Molecular dynamics,~MD)}主要用于各类化学反应、合金与复杂材料状态方程研究,着重关注体系的反应或状态随温度、压力变化规律和动力学性质
\vskip 5pt
\textcolor{blue}{分子动力学模拟的基本框架}
	\begin{itemize}
		\item 结构优化:~根据体系的初始构型\textrm{(initial configuration)},遵从能量最低原理,得到体系基态结构(确定基态时原子的位置)
		\item 原子运动计算:~在一定环境(温度、压力等)条件下,计算各原子的受力,并依据运动方程得到设定时间步长下的原子的运动,进而获得得体系的当前构型
		\item 径迹计算:~在设定的时间范围内,根据原子运动和体系构型的变化,组合成体系随时间演化的径迹\textrm{(the trajectory of time evolution)}
		\item 结果分析:~分析体系的径迹变化规律,得到体系的动力学和热力学性质
	\end{itemize}
}

\frame
{
	\frametitle{原子间相互作用力的表示}
	分子动力学模拟中影响结果最主要因素之一是\textcolor{purple}{原子间相互作用力}的准确度
\vskip 5pt
\begin{itemize}
	\item 经典分子动力学模拟中,原子间相互作用力是根据经验势函数得到的\footnote{\fontsize{6.2pt}{4.2pt}\selectfont{经验势函数也称为力场,是参数化形式给出的原子间相互作用,一般通过对实验数据拟合或小体系的第一原理计算得到}}。构建一套高精度的经验势函数代价很高,而且经验势函数一般不具备可移植性
\end{itemize}
\vskip 5pt
	当动力学过程必须考虑量子效应(如电子影响的贡献不可忽略时),必须采用第一原理分子动力学\textrm{(Ab initio~MD,~AIMD)}
	\begin{itemize}
		\item 所谓第一原理分子动力学,就是在计算原子运动时,将电子结构变化的贡献考虑进来,因此在每一时间步长,体系实时构型下的原子受力计算,都必须伴随电子结构计算
	\end{itemize}
	一般电子结构计算采用\textrm{DFT}计算,不难想见,第一原理分子动力学模拟的代价极高
}

\frame
{
	\frametitle{第一原理分子动力学\textrm{AIMD}}
	\begin{itemize}
		\item \textrm{AIMD}将电子结构与原子和经典轨迹计算在同一基础上完成
		\item 每个原子运动步的受力都是在电子结构计算基础上获得的
%		\item 基于\textrm{B-O}近似,原子运动径迹的每一步上,电子运动都达到收敛
		\item 扩展的\textrm{Lagrangian}方法:~根据体系几何结构构造体系波函数\\
			\textcolor{blue}{\textrm{Car-Parinello}}:\\
			平面波基,构成分子轨道\\
			\textcolor{blue}{\textrm{Atom-centered Density Matrix Progation~(ADMP)}}:\\
			原子中心基,构成密度矩阵
	\end{itemize}
	\textrm{AIMD}计算内容
	\begin{itemize}
		\item 可用于复杂体系的电子结构计算
		\item 几何结构优化(能量最小化)
		\item 描述系统演化
		\item 模拟时长规模$\approx{\mathrm{ps}}(10^{-12}\mathrm{s})$~(经典分子动力学 $\approx\mathrm{ns}(10^{-9}\mathrm{s})$)
	\end{itemize}
}


%\frame
%{
%	\frametitle{第一原理分子动力学的分类}
%\begin{figure}[h!]
%\centering
%\vspace*{-0.05in}
%%\hspace*{-10pt}
%\includegraphics[height=2.3in,width=4.0in,viewport=0 0 440 230,clip]{Figures/Molecular-dynamics_Claaified.png}
%\caption{\fontsize{5.2pt}{4.2pt}\selectfont{\textrm{The major differences between the methods of aiMD simulations.}}}%
%\label{Molecular-dynamics_Claaified}
%\end{figure}
%}
%
\frame
{
	\frametitle{第一原理分子动力学中的近似}
	由于分子动力学模拟的复杂性,必须做出适当的近似。具体到第一原理分子动力学,一般有两类重要的近似:
	\begin{itemize}
		\item \textcolor{blue}{绝热近似\textrm{(adiabatic approximation)}}\\
			{\fontsize{6.2pt}{4.2pt}\selectfont{假设电子-原子核在能量层面上完全分离,彼此间没有能量传递}}
		\item \textcolor{blue}{\textrm{Born-Oppenheimer}近似}\\
			{\fontsize{6.2pt}{4.2pt}\selectfont{假设电子和原子核的运动完全解耦,对应每个时间步长的原子构型,电子可以实时处于基态\footnote{\fontsize{5.2pt}{4.2pt}\selectfont{\textrm{B-O}近似也是一种绝热近似,\textcolor{red}{但\textrm{B-O}近似下的绝热强调电子对核运动的瞬时响应}。讨论电子计算时,\textrm{B-O}近似下假设原子核是固定不动的;~在分子动力学讨论中,绝热近似强调的是电子-核运动在能量上的完整分离,而\textrm{B-O}近似则明确要求电子-核运动彼此完全解耦,且电子实时处于基态}}}}
	\end{itemize}
\begin{figure}[h!]
\centering
\vspace*{-0.25in}
%\hspace*{-10pt}
\includegraphics[height=1.55in,width=2.6in,viewport=0 0 440 230,clip]{Figures/Molecular-dynamics_Claaified.png}
%\caption{\fontsize{5.2pt}{4.2pt}\selectfont{\textrm{The major differences between the methods of aiMD simulations.}}}%
\label{Molecular-dynamics_Classified}
\end{figure}
}

\frame
{
	\frametitle{第一原理分子动力学:~\textrm{BOMD}}
	如果绝热近似和\textrm{Born-Oppenheimer}近似同时满足,称为\textrm{Born-Oppenheimer}分子动力学\textrm{(BOMD)}
	\begin{itemize}
		\item 原子核运动的势函数为$E[\{\psi_i\};\mathbf{R}]$,并且每个时间步长内,势函数对$\{\psi_i(\vec r)\}$取极小值\\
			\begin{displaymath}
				\begin{aligned}
					L_{\mathrm{BO}}(\{\psi_i\};~\mathbf{R},\dot{\mathbf{R}})=&\dfrac12\sum_{I=1}^NM_I\dot{\mathbf{R}}^2_I-\underset{\{\psi_i\}}{\textcolor{red}{\min}}~E[\{\psi_i\};\mathbf{R}]\\
					&+\sum_{ij}\Lambda_{ij}(\langle\psi_i|\psi_j\rangle-\delta_{ij})
				\end{aligned}
			\end{displaymath}
		\item 运动方程\textrm{(Equations of Motion,~EOM)}
			\begin{displaymath}
				\begin{aligned}
					\hspace*{-40pt}
					M_I\ddot{\mathbf{R}}_I=&-\nabla_{\mathbf{R}_I}\bigg[\underset{\{\psi_i\}}{\textcolor{red}{\min}}~E[\{\psi_i\};\mathbf{R}]\bigg|_{\{\langle\psi_i|\psi_j\rangle=\delta_{ij}\}}\bigg]\\
					=&\textcolor{purple}{-\dfrac{\partial E}{\partial\mathbf{R}_I}}\textcolor{blue}{+\sum_{i,j}\Lambda_{ij}\dfrac{\partial}{\partial \mathbf{R}_I}\langle\psi_i|\psi_j\rangle}\textcolor{magenta}{-2\sum_i\dfrac{\partial\langle\psi_i|}{\partial\mathbf{R}_I}\bigg[\dfrac{\delta E}{\delta\langle\psi_i|}-\sum_j\Lambda_{ij}|\psi_j\rangle\bigg]}
				\end{aligned}
			\end{displaymath}
	\end{itemize}
}

\frame
{
	\frametitle{第一原理分子动力学:~\textrm{BOMD}}
	\begin{itemize}
		\item $\textcolor{purple}{\dfrac{\partial E}{\partial\mathbf{R}_I}}$表示\textrm{Hellmann-Feynman}力$\vec F_{\mathrm{HF}}$
		\item $\textcolor{blue}{\sum\limits_{i,j}\Lambda_{ij}\dfrac{\partial}{\partial \mathbf{R}_I}\langle\psi_i|\psi_j\rangle}$是\textrm{Pulay}力$\vec F_{\mathrm{WF}}$\\
			{\fontsize{6.2pt}{4.2pt}\selectfont{源于电子波函数正交要求,且只有当基函数为局域函数(依赖于$\mathbf{R}$时)才有贡献}}
		\item $\textcolor{magenta}{\sum\limits_i\dfrac{\partial\langle\psi_i|}{\partial\mathbf{R}_I}\bigg[\dfrac{\delta E}{\delta\langle\psi_i|}-\sum\limits_j\Lambda_{ij}|\psi_j\rangle\bigg]}$表示非自洽电子态的影响$\vec F_{\mathrm{NSC}}$\\
			{\fontsize{6.2pt}{4.2pt}\selectfont{源自非局域基(如平面波),由于波函数非显式依赖$\mathbf{R}$,因此展开系数$c_{ij}(\mathbf{R})$依赖于原子核位置
			\begin{displaymath}
				\psi_i(\mathbf{R})=\sum\limits_jc_{ij}(\mathbf{R})\phi_i
			\end{displaymath}
			前面\textrm{MOE}中的系数\textrm{2}源于\textrm{K-S}轨道波函数为实数时的简化表示\\
			这一项的贡献比起$F_{\mathrm{HF}}$小很多,\textcolor{blue}{只要当$\psi_i(\mathbf{R})$是体系精确的电子的本征态波函数,该项就会消失}——换言之,只有非完全自洽的电子计算,才需要考虑该项的贡献。显然,所有数值计算中,都将存在不等式
	\begin{displaymath}
			0\leqslant-\dfrac{\delta E}{\delta\langle\psi_i|}+\sum_j\Lambda_{ij}|\psi_j\rangle=-\hat{H_{\mathrm{e}}}\langle\psi_j|+\sum_j\Lambda_{ij}|\psi_j\rangle
		\end{displaymath} }}
	\end{itemize}
}

\frame
{
	\frametitle{第一原理分子动力学:~\textrm{BOMD}}
	另一方面,如果忽略$\vec F_{\mathrm{WF}}$和$\vec F_{\mathrm{NSC}}$的贡献,仅对体系电子的非本征态波函数应用\textrm{Hellmann-Feynman}定理,得到的结果和精确计算的原子受力
	\begin{displaymath}
		\vec F=\vec F_{\mathrm{HF}}+\vec F_{\mathrm{WF}}+\vec F_{\mathrm{NSC}}
	\end{displaymath}
	计算相比,也只有微小的偏差
	\vskip 1.5pt
	这是因为在\textrm{DFT}框架下,能量是电荷密度的非线性函数,因此$H_{\mathrm{e}}$必须通过迭代求解;~而原子受力的误差则随电荷密度线性变化——这也解释了为什么一般\textrm{BOMD}计算的原子受力比体系总能要精确得多

	\vskip 5pt
	在\textrm{BOMD}中,\textrm{Born-Oppenheimer}近似下核与电子的运动完全解耦,在此基础上考虑绝热近似,将不再有对动力学模拟的时间步长限制,相比于其它\textrm{AIMD}方法,\textrm{BOMD}模拟允许的时间步长要长得多
}

\frame
{
	\frametitle{第一原理分子动力学:~\textrm{CPMD}}
	1985年\textrm{Car}和\textrm{Parrinello}在密度泛函理论基础上,将第一原理方法应用到分子动力学研究中,形成\textrm{Car-Parrinello Molecular dynamics~(CPMD)}
	\begin{itemize}
		\item 与\textrm{BOMD}不同,在\textrm{CPMD}中,电子自由度存在于经典\textrm{Lagrangian}中
			\begin{displaymath}
				\begin{aligned}
					L_{\mathrm{CP}}(\{\psi_i\};~\mathbf{R},\dot{\mathbf{R}})=&\textcolor{blue}{\dfrac12\mu\sum_i\langle\dot{\psi}_i|\dot{\psi}_i\rangle}+\dfrac12\sum_{I=1}^NM_I\dot{\mathbf{R}}^2_I\\
					-&\textcolor{red}{E}[\{\psi_i\};\mathbf{R}]\\
					&+\sum_{ij}\Lambda_{ij}(\langle\psi_i|\psi_j\rangle-\delta_{ij})
				\end{aligned}
			\end{displaymath}
			{\fontsize{6.2pt}{4.2pt}\selectfont{在经典\textrm{Lagrangian}中考虑电子自由度,人为地引入了傀电子质量参数$\mu$和傀轨道速度$\dot{\psi}_i$}}
	\end{itemize}
}

\frame
{
	\frametitle{第一原理分子动力学:~\textrm{CPMD}}
	\begin{itemize}
		\item \textrm{CPMD}下的运动方程表示为
			\begin{displaymath}
				\begin{aligned}
					M_I\ddot{\mathbf{R}}_I=&-\nabla_{\mathbf{R}_I}\bigg[E[\{\psi_i\};\mathbf{R}]\bigg|_{\{\langle\psi_i|\psi_j\rangle=\delta_{ij}\}}\bigg]\\
					=&\textcolor{purple}{-\dfrac{\partial E}{\partial\mathbf{R}_I}}\textcolor{blue}{+\sum_{i,j}\Lambda_{ij}\dfrac{\partial}{\partial \mathbf{R}_I}\langle\psi_i|\psi_j\rangle}\\
					\mu\ddot{\psi}_i(\vec r,t)=&-\dfrac{\delta E}{\delta\langle\psi_i|}+\sum_j\Lambda_{ij}|\psi_j\rangle\\
					=&-\hat{H_{\mathrm{e}}}\langle\psi_j|+\sum_j\Lambda_{ij}|\psi_j\rangle
				\end{aligned}
			\end{displaymath}
			{\fontsize{6.2pt}{4.2pt}\selectfont{$-\dfrac{\delta E}{\delta\langle\psi|}$表示经典力学框架下的电子受力,用来描述分子动力学范畴内电子自由度随原子核运动的情况 }}
	\end{itemize}
}

\frame
{
	\frametitle{\textrm{CPMD}计算的特点}
	\begin{itemize}
		\item 计算成本大大节约\\
			{\fontsize{8.2pt}{4.2pt}\selectfont{相比于\textrm{BOMD},\textrm{CPMD}无需在每个分子动力学时间步长执行电子自洽计算}}
		\item 计算时间步长不能太长\\
			{\fontsize{8.2pt}{4.2pt}\selectfont{绝热近似要求电子-核运动能量彼此分离,\textcolor{blue}{声子最高频率}$\omega_{\mathrm I}$必须远小于\textcolor{blue}{傀电子最低振动频率}$\omega_{\mathrm e}$
			\begin{displaymath}
				\omega_{\mathrm e}\propto\sqrt{\dfrac{\Delta E_{\mathrm{gap}}}{\mu}}
			\end{displaymath}
			$\Delta E_{\mathrm{gap}}$是\textrm{K-S}单粒子的带隙,许可最大时间步长$\Delta t_{\mathrm{max}}<1/\omega_{\mathrm{e}}$,大小主要由$\sqrt{\mu}$}}确定
		\item $\mu$物理上没有意义,但通过调节$\mu$可以平衡\textrm{AIMD}的效率和精度,一般选取$\mu$使得$\omega_{\mathrm{I}}<<\omega_{\mathrm{I}}$成立
		\item 对于金属/导体的\textrm{CPMD}计算,由于$\Delta E_{\mathrm{gap}}=0$,必须要求体系通过恒温条件平衡交换能或者泛函允许分数占据
		\end{itemize}
}

\frame
{
	\frametitle{其它\textrm{AIMD}:~\textrm{PIMD}和\textrm{Ehrenfest~MD}}
	\begin{itemize}
		\item \textrm{Path Integral~MD~(PIMD)}\footnote{\fontsize{6.2pt}{4.2pt}\selectfont{基于量子统计的第一原理路径积分称为\textrm{Feynman}路径积分\textrm{(path integrals)}}}\\
			\textrm{PIMD}用量子力学计算电子和原子核运动,因此该方法比\textrm{BOMD}和\textrm{CPMD}方法精确,特别是对于含有轻元素体系——计算量也要大得多
		\item \textrm{Ehrenfest~MD}\\
			电子自由度通过求解含时\textrm{(Time-dependent)~Schr\"odinger}方程得到,当$\Delta t\rightarrow0$,自由度变化对应于电子的幺正传播\textrm{(unitary propagation)}\footnote{\fontsize{6.2pt}{4.2pt}\selectfont{\textrm{Ehrenfest~MD}的幺正变换确保波函数保持正交,但代价是积分时间步长必须极小,因此\textrm{Ehrenfest~MD}模拟时间尺度仅达\textrm{atto}$(10^{-18})$秒尺度}}
	\end{itemize}
	\textcolor{blue}{\textrm{CPMD}结合了\textrm{BOMD}与\textrm{Ehrenfest~MD}的优点}:
	\begin{itemize}
		\item 计算体系受力由总能对粒子位置的求导,并非求电子态的$\langle\Psi_0|\hat H_{\mathrm{e}}|\Psi_0\rangle$极小值
		\item 因为选择平面波基,$\vec F_{\mathrm{NSC}}$自然为0
	\end{itemize}
}

\frame
{
	\frametitle{平衡态统计基础}
	系综(\textrm{Ensembles})是在一定的宏观条件下,由大量微观粒子组成的性质和结构完全相同的、处于各种运动状态的、各自独立的系统整体的集合。简言之,系综是给定宏观条件下,所有微观状态的集合。
%	应用\textrm{Verlet}算法,完成单粒子运动的数值积分,可以得到动力学体系的\textrm{Hamiltonian}对应的能量,进而应用统计力学的统计系综,获得宏观体系的物理量
	\vskip 3pt
	\textcolor{blue}{等概率原理}\textrm{(Principle of equal weights)}:\\
	一个热力学体系有相同的概率到达每个可能经历的微观态。\\
	等概率原理导出\textrm{Boltzmann}分布
	\begin{displaymath}
		P_j=\dfrac{\mathrm{e}^{-\beta\varepsilon_j}}Q
	\end{displaymath}
	这里$Q$称为配分函数\textrm{(partition function)}
	\begin{displaymath}
		\begin{aligned}
			Q=&\sum_i\mathrm{e}^{(-\beta\varepsilon_i)}\\
			\beta=&1/k_{\mathrm{B}}T
		\end{aligned}
	\end{displaymath}
	物理量的系综平均
	\begin{displaymath}
		\langle A\rangle=\sum_jA_j\mathrm{e}^{(-\beta\varepsilon_j)}/Q
	\end{displaymath}
}

\frame
{
	\frametitle{常用统计系综}
\begin{figure}[h!]
\centering
\vspace*{-0.20in}
%\includegraphics[height=1.60in,width=3.85in,viewport=0 0 1420 570,clip]{Figures/Statistical_Ensembles.png}
\includegraphics[height=1.20in,width=3.75in,viewport=0 0 1420 470,clip]{Figures/Statistical_Ensembles.png}
\caption{\tiny \textrm{The Statistical Ensembles.}}%(与文献\cite{EPJB33-47_2003}图1对比)
\label{Statistical_Ensembles}
\end{figure}
\vskip -15pt
\begin{itemize}
		{\fontsize{7.2pt}{1.2pt}\selectfont{
		\item 微正则系综\textrm{(Mircocanonical Ensemble)}%\footnote{\fontsize{4.5pt}{1.2pt}\selectfont{\textrm{canonical},汉译作``正则'',出自《楚辞\textperiodcentered 离骚》``皇揽揆余於初度兮,肇锡余以嘉名;~名余曰\textcolor{red}{正则}兮,字余曰灵均'',《楚辞章句》\upcite{Chucizhangju}:~``正,平也;~则,法也;~灵,神也;~均,调也。言正平可法则者,莫过于天;~养物均调者,莫神于地。高平曰原,故父伯庸名我为平以法天,字我为原以法地。言己上之能安君,下之能养民也。''意思是说``正则''、``灵均''隐喻着某种意义,即平正是天的象征,原均是地的象征。因此正则的含义是``\textcolor{blue}{符合天道}'',与\textrm{canonical}的意思\textrm{of, relating to, or forming a canon}意义一致。}}
			:~\textrm{NVE}皆为常数
		\item 正则系综\textrm{(Canonical Ensemble)}:~\textrm{NVT}皆为常数
		\item 巨正则系综\textrm{(Grandcanonical Ensemble)}:~\textrm{$\mu$VT}皆为常数,粒子数不固定
		\item 等压-等温系综\textrm{(Isobaric-Isothermal Ensemble)}:~\textrm{NPT}皆为常数
		\item 等焓-等压系综\textrm{(Isoenthalpic-Isobaric Ensemble)}:~\textrm{NPH}皆为常数
		\item 等张力-等温系综\textrm{(Isotension-Isothermal Ensemble)}:~容器形状可变 }}
\end{itemize}
}

\frame
{
	\frametitle{常用热力学量}
	\begin{itemize}
{\fontsize{7.8pt}{1.2pt}\selectfont{
		\item 动能 ~$E_{\mathrm{k}}=\bigg\langle\sum\limits_{i=1}^N\dfrac12m_iv_i^2\bigg\rangle$
		\item 势能 ~$E_{\mathrm{p}}=\bigg\langle\sum\limits_{i=1}^NE_{\mathrm{p}i}\bigg\rangle$
		\item 温度 ~$T=\dfrac1{\mathrm{d}Nk_{\mathrm{B}}}\bigg\langle\sum\limits_{i=1}^Nm_iv_i^2\bigg\rangle$ ~~~~ 其中$\mathrm{d}$是空间维度
		\item 压强 ~$p=\dfrac{k_{\mathrm{B}}TN}{V}+\dfrac1{\mathrm{d}V}\bigg\langle\sum\limits_{i<j}\vec f_{ij}\cdot\vec r_{ij}\bigg\rangle$
		\item 焓 ~$H=E+pV$ ~~~~ 相当于\textrm{NPT}下的有效总内能
		\item 熵 ~$S=k_{\mathrm{B}}\ln\Omega(N,V,E)$ ~~~~ $\Omega$是系统的总的微观状态数
		\item \textrm{Helmholtz}自由能:~\textcolor{blue}{\textrm{NVT}下的自由能}
			\begin{displaymath}
				F=E-TS=-k_{\mathrm{B}}T\ln{Q}
			\end{displaymath}
		\item \textrm{Gibbs}自由能:~\textcolor{blue}{\textrm{NPT}下的自由能}
			\begin{displaymath}
				G=F+pV=E-TS+pV
			\end{displaymath}
		\item 化学势 ~$\mu=\dfrac{\partial G}{\partial N}\bigg|_{T,p}=\dfrac{\partial F}{\partial N}\bigg|_{T,V}$}}
	\end{itemize}
}
