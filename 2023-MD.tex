%%%%%%%%%%%%%%%%%%%%%%%%%%%%%%%%%%%%%%%%%%  不使用 authblk 包制作标题  %%%%%%%%%%%%%%%%%%%%%%%%%%%%%%%%%%%%%%%%%%%%%%
%-------------------------------PPT Title-------------------------------------
\title{分子动力学简介}
%-----------------------------------------------------------------------------
%----------------------------Author & Date------------------------------------

%\author[\textrm{Jun\_Jiang}]{姜\;\;骏\inst{}} %[]{} (optional, use only with lots of authors)
%% - Give the names in the same order as the appear in the paper.
%% - Use the \inst{?} command only if the authors have different
%%   affiliation.
\institute[BCC]{\inst{}%
%\institute[Gain~Strong]{\inst{}%
\vskip -20pt 北京市计算中心~智能计算事业部}
%\vskip -20pt {\large 格致斯创~科技}}
\date[\today] % (optional, should be abbreviation of conference name)
{%	{\fontsize{6.2pt}{4.2pt}\selectfont{\textcolor{blue}{E-mail:~}\url{jiangjun@bcc.ac.cn}}}
\vskip 45 pt {\fontsize{8.2pt}{6.2pt}\selectfont{%清华大学\;\;物理系% 报告地点
	\vskip 5 pt \textrm{2023.04}}}
}

%% - Either use conference name or its abbreviation
%% - Not really information to the audience, more for people (including
%%   yourself) who are reading the slides onlin%%   yourself) who are reading the slides onlin%%   yourself) who are reading the slides onlineee
%%%%%%%%%%%%%%%%%%%%%%%%%%%%%%%%%%%%%%%%%%%%%%%%%%%%%%%%%%%%%%%%%%%%%%%%%%%%%%%%%%%%%%%%%%%%%%%%%%%%%%%%%%%%%%%%%%%%%

\subject{}
% This is only inserted into the PDF information catalog. Can be left
% out.
%\maketitle
\frame
{
%	\frametitle{\fontsize{9.5pt}{5.2pt}\selectfont{\textcolor{orange}{“高通量并发式材料计算算法与软件”年度检查}}}
\titlepage
}
%-----------------------------------------------------------------------------

%------------------------------------------------------------------------------列出全文 outline ---------------------------------------------------------------------------------
\section*{}
\frame[allowframebreaks]
{
  \frametitle{Outline}
%  \frametitle{\textcolor{mycolor}{\secname}}
  \tableofcontents%[current,currentsection,currentsubsection]
}
%%在每个section之前列出全部Outline
%%类似的在每个subsection之前列出全部Outline是\AtBeginSubsection[]
%\AtBeginSection[]
%{
%  \frame<handout:0>%[allowframebreaks]
%  {
%    \frametitle{Outline}
%%全部Outline中,本部分加亮
%    \tableofcontents[current,currentsection]
%  }
%}

%-----------------------------------------------PPT main Body------------------------------------------------------------------------------------
\small
\frame
{
	\frametitle{分子模拟}
	分子模拟:~\textcolor{blue}{研究原子分子层面的物性}
	\begin{itemize}
		\item 分子模拟方法分为~\textrm{Monte~Carlo~(MC)}和\textrm{Molecular~Dynamics~(MD)}两大类
		\item 对\textrm{Boltzmann}分布的重要性采样:~\textcolor{blue}{统计物理}是理论基础
	\end{itemize}
	\begin{itemize}
		\item 量子效应只有在粒子波长$\lambda$与原子间距离\textrm{(1-3~\AA)}相当时才变得重要。因此对于原子/分子运动,一般可以用经典力学方程描述\\
			尽管第一性原理方法在功能材料,特别是光电磁相关研究领域中独树一帜,但是大多数与原子/分子运动相关的材料——典型的如结构材料——研究,主要应用\textcolor{blue}{经典力学}预测%特别是对于结构材料,如何利用分子动力学方法来描述材料中的原子运动是该领域的重要主题。
		\item 自\textrm{1950}年代,\textrm{Alder}和\textrm{Wainwright}通过密堆积球出发,成功模拟液-固相变\upcite{JCP27-1208_1957}开始,标志着分子动力学方法成功应用到材料学模拟中
			%自此以后,大量的高效算法和代码以及不断升级的计算能力使得分子动力学在材料领域发挥的作用日益凸显,进入21世纪,\textrm{Kadau}等实现了对3200亿个原子的分子动力学模拟\cite{IJMPC17-1755_2006},显示出该方法的巨大能力。
	\end{itemize}
	分子模拟:~\textcolor{blue}{依然是计算材料研究领域最重要的理论工具之一}
}

\frame
{
	\frametitle{\textrm{Monte~Carlo}模拟}
\begin{minipage}{0.33\textwidth}
%\begin{figure}[h!]
%\centering
%\vspace{-8.0pt}
%\includegraphics[height=1.45in,width=1.50in,viewport=0 0 140 138,clip]{Figures/Unit_Cell_of_Liquid_Water.png}
%\caption{\textrm{\tiny Schematic representation of water can be simulated using Periodic Boundary Conditions.}}
%\label{Water-PBC}
%\end{figure}
\begin{itemize}
\vspace{-8.0pt}
{\fontsize{9.5pt}{6.2pt}\selectfont{
	\item 基于\textrm{Boltzmann}分布,只能模拟平衡态体系
	\item 只计算势能(状态函数),不用计算力(瞬时作用)
\item 模拟步长可以比较大}}
\end{itemize}
\end{minipage}
\begin{minipage}{0.65\textwidth}
\begin{figure}[h!]
\centering
\vspace{-5.0pt}
\includegraphics[height=1.55in,width=3.00in,viewport=0 0 930 475,clip]{Figures/Schematic-representation-of-the-Metropolis-Monte-Carlo-simulation.png}
%\caption{\textrm{Schematic representation of the Metropolis−Monte Carlo simulation.}}
\label{MC-Algorithm-Workflow}
\end{figure}
\end{minipage}
\begin{figure}[h!]
\centering
\vspace{-10.0pt}
\includegraphics[height=1.25in,width=4.15in,viewport=0 0 1470 480,clip]{Figures/Monte-Carlo-development.png}
%\caption{\textrm{Schematic representation of the Metropolis−Monte Carlo simulation.}}
\label{Monte-Carlo-development}
\end{figure}
}

\subsection{经典分子动力学提要}
\frame
{
	\frametitle{}
\begin{figure}[h!]
\centering
\vspace{-7.0pt}
\includegraphics[height=2.85in,width=4.00in,viewport=0 0 410 300,clip]{Figures/Nobel_Prize_Chemistry-2013.jpg}
%\caption{\textrm{Schematic representation of the Metropolis−Monte Carlo simulation.}}
\label{Nobel-Prize-Chemistry_2013}
\end{figure}
}

\frame
{
	\frametitle{经典分子动力学}
\begin{minipage}{0.45\textwidth}
\begin{figure}[h!]
\centering
\vspace{-3.0pt}
\includegraphics[height=2.00in,width=2.00in,viewport=0 0 135 135,clip]{Figures/Unit_Cell_of_Liquid_Water.png}
%\caption{\textrm{Schematic representation of the Metropolis−Monte Carlo simulation.}}
\label{Unit_Cell_of_Liquid_Water}
\end{figure}
\end{minipage}
\begin{minipage}{0.53\textwidth}
	\begin{itemize}
		\item 应用数值方法解析多粒子体系(原子、离子、$\cdots$)运动
		\item 粒子间相互作用可用简洁的解析函数或数值描述
		\item 广泛应用于材料科学、物理化学和生物科学的相关研究
		\item 模拟体系的粒子数规模:~\\$100\sim10^6$
		\item 模拟体系的时间范围:~\\$10~\mathrm{ps}\sim1\mu\mathrm{s}$~(一般是$\mathrm{ns}$范围)
	\end{itemize}
\end{minipage}
\begin{itemize}
	\item 分子动力学模拟可以包括温度和压力效应
	\item 分子动力学可以模拟相对大的体系和相对长的时间
	\item 分子动力学可以得到分子尺度的结构和动力学信息
\end{itemize}
}

\frame
{
	\frametitle{经典分子动力学}
	分子动力学方法的\textcolor{red}{优点}:
	\begin{itemize}
			\setlength{\itemsep}{5pt}
		\item 可以模拟大规模分子系统,原则上能计算系统的微观、宏观物理量,增补实验数据的空缺
		\item 与\textrm{Monte Carlo}方法相比,具备更高的准确度,可以提供更多微观粒子的运动细节
		\item 可以模拟非平衡态的体系以及体系在极端条件下复杂情况下的物理性质和物理状态
	\end{itemize}
	分子动力学方法的\textcolor{red}{缺点}:
	\begin{itemize}
			\setlength{\itemsep}{5pt}
		\item 只能描述原子(核)的运动,但无法反映电子的运动,难以准确模拟化学键
		\item 数值模拟依赖经验力场,化学环境复杂的体系,如化学键断裂,电子结构将由稳态电子结构过渡到过渡态,无法用简单的解析力场描述
	\end{itemize}
}

\frame
{
	\frametitle{相空间和时间平均}
	分子动力学模拟一般在相空间完成\\
	\vskip 3pt
	\textcolor{blue}{相空间}\textrm{(Phase~Space)}是表示系统所有可能所处状态的空间\\
	\vskip 5pt
{\fontsize{8.5pt}{6.2pt}\selectfont{
	\textcolor{blue}{系统中每个可能的状态在相空间中都存在一个确定的对应点}\\
	相空间是一个六维假想空间,其中\textcolor{blue}{动量}和\textcolor{blue}{空间}各占三维}}\\
	\vskip 8pt
	对于包含$N$个粒子的体系,构成$6N$维相空间$(\Gamma_N)$:~包括$3N$个空间$(\vec r)$自由度,$3N$个动量$(\vec p)$自由度
	\vskip 5pt
{\fontsize{8.2pt}{6.2pt}\selectfont{
	如果描述体系性质的状态函数$A$可以用空间和动量坐标表示~$A(\Gamma)$(如体系的总能或瞬时压力),其\textcolor{purple}{时间平均}值为}}
	\begin{displaymath}
		A_{\mathrm{obs}}=\langle A\rangle_{\mathrm{time}}=\langle A(\Gamma(t))\rangle_{\mathrm{time}}=\lim_{\tau\rightarrow\infty}\dfrac1{\tau}\int_{t=0}^{\tau}A(\Gamma(t))\mathrm{d}t
	\end{displaymath}
	\textcolor{magenta}{各态遍历假设}\textrm{(ergodic~hypothesis)}\\
	\vskip 5pt
{\fontsize{9.5pt}{6.2pt}\selectfont{
只要演化时间足够长,孤立体系会等概率地遍历每个可能的微观状态~\\
换言之,\textcolor{blue}{体系足够长时间平均}等价于\textcolor{blue}{体系许可的大量系综的系综平均}
%分子动力学模拟的一个思想是用时间换取空间。我们无法得知系统在某一时刻所有可能的微观状态,但是可以通过一段时间的采样来获取系统大量的微观状态数。模拟的整个过程可以理解为建立了一个系综,而每一个时间点就是组成这个系综的一个系统,它们具有相同的限制条件(例如:粒子总数,温度、压强或体积等),但是具有不同的微观状态。系统的热力学量就是系综平均的结果。
}}
}

\frame
{
	\frametitle{分子动力学基本思想}
	\begin{itemize}
			\setlength{\itemsep}{3pt}
		\item 分子动力学模拟过程中产生的具体数据是\\
			\textcolor{blue}{粒子在相空间中随时间演化的 一系列点}
		\item 分子动力学模拟的直接结果是\\
			\textcolor{blue}{体系中全部粒子随时间变化的径迹\textrm{(trajectory)}}
		\item 体系的时间平均和其它物理性质都可以通过粒子径迹计算
		\item 约化单位\textrm{(reduced unit)}\\
{\fontsize{7.2pt}{6.2pt}\selectfont{
	数值模拟中使用的是内部单位,需要通过换算才能得到实际体系的真实物理单位(国际单位制,\textrm{Syst\`eme International $\mathrm{d}'$Unit\'es, SI})}}
	\begin{enumerate}
{\fontsize{7.2pt}{6.2pt}\selectfont{
		\item 四个基本物理量单位:\\
			长度\textrm{L}、质量\textrm{M}、时间\textrm{t},电荷电量\textrm{Q}
		\item 其它物理量:\\
			能量~$E=M\cdot L^2/t^2$\hskip 15pt温度~$T=E/k_{\mathrm{B}}$\hskip 15pt压力~$P=E/L^3$\\
			质量密度~$\rho=M/L^3$\hskip 8pt数量密度~$n=1/L^3$\hskip 8pt介电常数~$\varepsilon=\dfrac{N_{\mathrm{A}}\cdot Q^2}{L\cdot E}$
			\vskip 4pt
	这里$k_{\mathrm{B}}$是\textrm{Boltzmann}常数,$N_{\mathrm{A}}$是\textrm{Avogadro}常数}}
	\end{enumerate}
	\end{itemize}
}

\frame
{
	\frametitle{分子动力学基本思想}
	在相空间中,体系中粒子的运动由\textrm{Hamiltonian}方程描述
	\begin{displaymath}
		\begin{aligned}
			\dot{\vec r}_i=&\dfrac{\partial\mathbf{H}}{\partial\vec p_i}=\dfrac{\vec p_i}{m_i}\\
			\dot{\vec p}_i=&-\dfrac{\partial\mathbf{H}}{\partial\vec r_i}=\vec f_i
		\end{aligned}
	\end{displaymath}
	\textrm{Hamiltonian}$\mathbf{H}$的定义为
	\begin{displaymath}
		\begin{aligned}
			\mathbf{H}(\vec r,\vec p)=&\mathbf{K}(\vec p)+V(\vec r)\\
			\mathbf{K}(\vec p)=&\sum_i^N\dfrac{\vec p_i^2}{2m_i}
		\end{aligned}
	\end{displaymath}
	由此可得粒子运动遵守的\textrm{Newton}方程为
	\begin{displaymath}
		\begin{aligned}
			\dot{\vec x}=&\dfrac{\partial\mathbf{H}}{\partial\vec p_x}=\dfrac{\vec p_x}m=\vec v_x\\
			\dot{\vec p}_x=&-\dfrac{\partial\mathbf{H}}{\partial{\vec x}}=-\dfrac{\partial V(\vec x)}{\partial\vec x}=\vec F=m\vec a_x
		\end{aligned}
	\end{displaymath}
}

\frame
{
	\frametitle{分子动力学模拟流程}
\begin{figure}[h!]
\centering
\vspace{-10.0pt}
\includegraphics[height=2.70in,width=3.00in,viewport=30 20 750 794,clip]{Figures/The-molecular_dynamcis_algorithm.png}
\caption{\textrm{The steps in performing a molecular dynamcis simulation.}}
\label{The-molecular-dynamics_algorithm}
\end{figure}
}

\frame[allowframebreaks]
{
	\frametitle{力场}
分子动力学中,粒子间相互作用用\textcolor{red}{力场}\textrm{(Force Field)},也就是“\textcolor{blue}{相互作用势}”,描述,力场的形式有很多种,一般将势函数分解为粒子间相互作用,包括双体相互作用、三体相互作用、$\cdots$
	\begin{displaymath}
		V(\vec r)=\sum_{ij}V_{ij}(\vec r_i,\vec r_j)+V_{ijk}(\vec r_i,\vec r_j,\vec r_k)+V_{ijkl}(\vec r_i,\vec r_j,\vec r_k,\vec r_l)+\cdots
	\end{displaymath}
\begin{figure}[h!]
\centering
\vspace{-18.0pt}
\includegraphics[height=1.47in,width=3.40in,viewport=0 0 1520 700,clip]{Figures/Interaction_particles.png}
\caption{\textrm{\tiny The Schematic representation of the interactions between pairs, triplets of particles.}}
\label{Interaction_particles}
\end{figure}

\begin{itemize}
	\item 一般说,分解后的粒子间相互作用中,最重要的贡献来自双体相互作用。对常将多体相互作用截断,仅保留双体相互作用的求和部分
		\begin{displaymath}
			V(\vec r)=\sum_{ij}V_{ij}(\vec r_i,\vec r_j)
		\end{displaymath}
	\item 真实的
\end{itemize}

	典型力场的有
	\begin{itemize}
		\item \textrm{Lennard-Jones}对势
	\begin{displaymath}
		U(r)=4\varepsilon\bigg[\bigg(\dfrac{\sigma}{r}\bigg)^{12}-\bigg(\dfrac{\sigma}{r}\bigg)^6\bigg]
	\end{displaymath}
	{\fontsize{7.2pt}{6.2pt}\selectfont{这里$\varepsilon$和$\sigma$是和原子有关的参数
	\textrm{L-J}势能的最低点在$r_{\min}=2^{(1/6)}\sigma\approx1.12\sigma$,$r<r_{\min}$时为排斥力,$r>r_{\min}$时为吸引力
\begin{figure}[h!]
\centering
\vspace*{-0.30in}
\includegraphics[height=1.00in,width=1.35in,viewport=0 0 340 270,clip]{Figures/Lennard-Jones_potential.png}
\caption{\tiny \textrm{The Lennard-Jones Potential.}}%(与文献\cite{EPJB33-47_2003}图1对比)
\label{Potential-Lennard-Jones}
\end{figure}
\vskip -20pt
	由\textrm{L-J}势改造,可以得到\textrm{WCA}势和\textrm{PHS}势}}
\item \textrm{Morse}势
	\begin{displaymath}
		U(r)=-D_{\mathrm{e}}+D_{\mathrm{e}}\bigg(1-\mathrm{e}^{-a(r-r_{\mathrm{e}})}\bigg)^2
	\end{displaymath}
	{\fontsize{7.2pt}{6.2pt}\selectfont{这里$D_{\mathrm{e}}$是\textrm{Morse}势的势阱深,参数$a$确定势阱宽度,$r_{\mathrm{e}}$是原子处于平衡位置的平衡键长
\begin{figure}[h!]
\centering
\vspace*{-0.15in}
\includegraphics[height=1.30in,width=1.85in,viewport=0 0 3540 2770,clip]{Figures/Morse-potential.png}
\caption{\tiny \textrm{The Morse potential (blue) and harmonic oscillator potential (green).}}%(与文献\cite{EPJB33-47_2003}图1对比)
\label{Potential-Morse}
\end{figure}
}}
\item \textrm{EAM}势\\
	{\fontsize{7.2pt}{6.2pt}\selectfont{对于金属晶体,内能虽可以表示为对相互作用之和,但拟合原子受力非常困难:\footnote{\fontsize{5.2pt}{3.2pt}\selectfont{应用二体势计算金属弹性常数时必须涉及对体积很敏感的能量项,因为涉及缺陷、表面的体积很难确定。}}\\
	\textcolor{red}{从物理上说金属原子处于电子海洋中,电子密度来自多个原子的贡献,这是自由电子气带来的多体效应}}}\\
	\textrm{EAM}将金属中原子的势能表示为二体势和多体势之和
	\begin{displaymath}
		E_i=F_{\alpha}\bigg(\sum_{j\neq i}\rho_{\beta}(r_{ij})\bigg)+\dfrac12\sum_{j\neq i}\phi_{\alpha\beta}(\vec r_{ij})
	\end{displaymath}
	{\fontsize{7.2pt}{6.2pt}\selectfont{$\alpha$和$\beta$分别为位置$i$、$j$处的原子类型\\
		$\phi$是二体势,是原子$\alpha$和$\beta$和原子间距$r_{ij}$的函数\\
		$F$是多体势,是其余原子在位置$i$处的电荷密度与位置$i$处原子$\alpha$的相互作用能,由原子类型$\alpha$和位置$i$处的电子密度确定\\
		位置$j$原子在位置$i$处产生的电荷密度$\rho$只与位置$j$处原子类型$\beta$和原子间距$r_{ij}$有关,与方向无关
	}}\\
	各类\textrm{EAM}势中,$\phi(r)$、$\rho(r)$和$F(\rho)$都不是解析的,以数值形式存储
	\end{itemize}
}

\frame
{
	\frametitle{经典分子动力学}
	装有$N$个经典粒子的$L_1\times L_2\times L_3$容器内,假设粒子间只有简单的二体相互作用\footnote{\fontsize{7.2pt}{6.2pt}\selectfont{二体作用是粒子间多体相互作用的简化,只考虑粒子两两间彼此相互作用。}}$\vec F(r)$,力的大小仅与粒子间间距$r$相关
	\begin{displaymath}
		\vec F(R_i)=\sum_{\substack{j=1\\j\neq i}}^N F(|\vec r_i-\vec r_j|)\hat{\vec r}_{ij}
	\end{displaymath}
	{\fontsize{7.2pt}{6.2pt}\selectfont{这里$R$代表全部原子坐标$\vec r_i$,$\hat{\vec r}_{ij}$是表示粒子$i$指向粒子$j$的矢量($\vec r_j-\vec r_i$)的单位矢量}}

	在经典力学框架下,粒子$i$的受力运动方程是:~
	\begin{displaymath}
		\dfrac{\mathrm{d}^2\vec r_i(t)}{\mathrm{d}t^2}=\dfrac{\vec F_i(R)}{m_i}
	\end{displaymath}
	粒子$i$的质量是$m_i$\\
	\textcolor{purple}{经典分子动力学,就是应用数值模拟对大量粒子求解该方程,基于统计力学原理,研究物质的状态和热力学性质}
}

\frame
{
	\frametitle{经典分子动力学与\textrm{Verlet}算法}
	分子动力学模拟研究的对象是平衡态体系
	\begin{itemize}
		\item 初始化
		\item 开始分子运动模拟,直到模拟体系达到平衡
		\item 继续模拟体系的物理性质,保存计算结果
	\end{itemize}
	\textcolor{blue}{标准\textrm{Verlet}算法:~}求解作用力$\vec F$下单个粒子运动的积分
	\begin{displaymath}
		\vec r(t+h)=2\vec r(t)-\vec r(t-h)+h^2\vec F(\vec r(t))/m
	\end{displaymath}
	{\fontsize{7.2pt}{6.2pt}\selectfont{这里$h$是时间步长,$t=nh$是模拟累积时间,$\vec r(t)$是粒子在时间$t$时的位置\\
	\textcolor{magenta}{每个时间步长的误差为$h^4$,在模拟时间范围内的累积误差是$h^2$}
\vskip 5pt
	{\fontsize{7.2pt}{6.2pt}\selectfont{如果已知模拟粒子的初始速度$\vec v$和时间,取初始态时间$t=0$}}
	\begin{displaymath}
		\vec r(h)=\vec r(0)=h\vec v(0)+\dfrac{h^2}2\vec F[\vec r(t=0)]~\qquad~ (m\equiv1)
	\end{displaymath}
误差为$h^3$,速度随时间变化的函数
\begin{displaymath}
	\vec v(t)=\dfrac{\vec r(t+h)-\vec r(t-h)}{2h}+\mathscr{O}(h^2)
\end{displaymath}
}}
}

\frame
{
	\frametitle{经典分子动力学与\textrm{Verlet}算法}
	\textrm{Verlet}算法有两种被普遍应用的变体形式,相比于标准\textrm{Verlet}算法,这两种方法误差累积效应更小
	\begin{itemize}
		\item \textcolor{blue}{蛙跳(\textrm{Leap-Frog})法}
			\begin{displaymath}
				\begin{aligned}
					\vec v(t+h/2)=&\vec v(t-h/2)+h\vec F[\vec r(t)]\\
					\vec r(t+h)=&\vec r(t)+h\vec v(t+h/2)
				\end{aligned}
			\end{displaymath}
		\item \textcolor{blue}{速度-\textrm{Verlet}算法}
			\begin{displaymath}
				\vec v(t)=\dfrac{\vec r(t+h)-\vec r(t-h)}{2h}
			\end{displaymath}
			\begin{displaymath}
				\begin{aligned}
					\vec r(t+h)=&\vec r(t)+h\vec v(t)+h^2\vec F(t)/2\\
					\vec v(r+h)=&\vec v(t)+h[\vec F(t+h)+\vec F(t)]/2
				\end{aligned}
			\end{displaymath}
			速度-\textrm{Verlet}算法更稳定也更方便,但需要保存$\vec F(t)$和$\vec F(t+h)$两个力的数组
	\end{itemize}
}

\frame
{
	\frametitle{经典分子动力学与\textrm{Verlet}算法}
	以下算法与速度-\textrm{Verlet}算法完全等价,但只需要保留$\vec F(t)$一个数组
	\begin{displaymath}
		\begin{aligned}
			\tilde{\vec v}(t)=&\vec v(t)+h\vec F(t)/2\\
			\vec r(t+h)=&\vec r(t)+h\tilde{\vec v}(t)\\
			\vec v(t+h)=&\tilde{\vec v}(t)+h\vec F(t+h)/2
		\end{aligned}
	\end{displaymath}
	而粒子受力$\vec F(t+h)$则在第二步、第三步之间临时计算
\vskip 5pt
	{\fontsize{6.2pt}{4.2pt}\selectfont{一般地,作用在粒子$i$上的力,是所有与粒子$i$的相互作用的“合成”结果
	\begin{displaymath}
		\vec F_i(R)=-\dfrac{\partial U(\{\vec r_i\})}{\partial \vec r_i}
	\end{displaymath}
	通常总的势能$U(\{\vec r_i\})$拆解为各部分贡献
	\begin{displaymath}
		U(\{\vec r_i\})=\sum_iU_1(\vec r_i)+\sum_i\sum_{j>i}U_2(\vec r_i,\vec r_j)+\sum_i\sum_{j>i}\sum_{k>j}U_3(\vec r_i,\vec r_j,\vec r_k)+\cdots
	\end{displaymath}
	这里$U_1(\vec r_i)$是单体势,一般是单个粒子在外场(如重力场、电场)中的势能,与材料性质无关\\
$U_2(\vec r_i,\vec r_j)$是双体势,$U_3(\vec r_i,\vec r_j,\vec r_k)$是描述粒子间对相互作用的主要函数}}

	在分子动力学计算中,力的计算需要更多的时间,因为其计算耗时步数是$\mathscr{O}(N^2)$,\textcolor{blue}{对于周期体系,这种力的计算尤其需要谨慎}
}

\frame
{
	\frametitle{统计系综}
	系综(\textrm{Ensembles})是在一定的宏观条件下,由大量微观粒子组成的性质和结构完全相同的、处于各种运动状态的、各自独立的系统整体的集合\footnote{\fontsize{4.2pt}{2.2pt}\selectfont{简言之,系综是系统的集合(\textcolor{magenta}{系统}:~宏观相同,微观不同)。}}。\\
	应用\textrm{Verlet}算法,完成单粒子运动的数值积分,可以得到动力学体系的\textrm{Hamiltonian}对应的能量,进而应用统计力学的统计系综,获得宏观体系的物理量
\begin{figure}[h!]
\centering
\vspace*{-0.20in}
%\includegraphics[height=1.60in,width=3.85in,viewport=0 0 1420 570,clip]{Figures/Statistical_Ensembles.png}
\includegraphics[height=1.30in,width=3.35in,viewport=0 0 1420 570,clip]{Figures/Statistical_Ensembles.png}
\caption{\tiny \textrm{The Statistical Ensembles.\footnote{\fontsize{3.5pt}{1.2pt}\selectfont{\textrm{canonical},汉译作“正则”,出自《楚辞\textperiodcentered 离骚》“皇揽揆余於初度兮,肇锡余以嘉名;~名余曰\textcolor{red}{正则}兮,字余曰灵均”,《楚辞章句》\upcite{Chucizhangju}:~“正,平也;~则,法也;~灵,神也;~均,调也。言正平可法则者,莫过于天;~养物均调者,莫过于地。高平曰原,故父伯庸名我为平以法天,字我为原以法地。言己上能安君,下能养民也。”,意思是说“正则”、“灵均”隐喻着某种意义,即平正是天的象征,原均是地的象征。因此正则的含义是“符合天道”,与\textrm{canonical}的意思\textrm{of, relating to, or forming a canon}意义一致。}}}}%(与文献\cite{EPJB33-47_2003}图1对比)
\label{Statistical_Ensembles}
\end{figure}
}
\frame
{
	\frametitle{分子动力学模拟}
	
}
%------------------------------------------------------------------------Reference----------------------------------------------------------------------------------------------
		\frame[allowframebreaks]
{
\frametitle{主要参考文献}
\begin{thebibliography}{99}
{\tiny
	\bibitem{JCP27-1208_1957}\textrm{B.J. Alder, T.E. Wainwright. \textit{J. Chem. Phys.} \textbf{27} (1957), 1208}
}
\end{thebibliography}
%\nocite*{}
}
